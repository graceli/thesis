%  Chemical modifications of the ionic and hydrophobic groups on Congo red were found to affect its binding affinity with fibrils, suggesting that both types of interactions may be important for CR binding.

% Biophysical studies of the binding of CR molecules with the amyloid fibril suggest that CR is most likely to be binding along the long-axis of the fibril.\cite{Frid:2007bo} % Although CR is typically used to detect the presence of fibrillar aggregates, studies suggest that CR can also interact with monomers of alpha-synuclein, and alpha-helical form of poly-L-lysine.\cite{Maltsev:2012kw}

% MD simulation studies have been useful in examining the interaction of CR with amyloid fibrils.

% Moreover, ThT binding is not limited to fibrils, and was found to bind in hydrophobic pockets of human serum albumin, a globular protein, with comparable affinity to many drug-like molecules.\cite{Groenning:2007p3436,Groenning:2007eo} % Not sure what this point will go.

% A comparative MD simulation study of the binding of PIB and ThT with protofibrillar models of \abeta40\ and \abeta42\ derived from SSNMR studies was performed to understand the binding differences between the dyes.\cite{Wu:2011fd}   In agreement with experimental observations of multiple binding sites and binding affinities (this was for Abeta40), both dyes were observed to bind to several sites on the fibrils. REF Charge-charge interactions between ThT and protofibrils did not appear to play an important role in binding, which was provided as the key reason for explaining why the removal of charged groups from ThT made PIB binding more favorable. 
% Energetic analysis show that stabilizing forces are the van der Waals term and surface hydrophobic interaction term, whereas electrostatic interaction terms were unfavorable.

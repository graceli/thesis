% Directly lifted from my Ph.D thesis proposal objectives

Alzheimer's Disease (AD) is a progressive neurodegenerative disease characterized pathologically by the presence of extracellular fibrillar deposits of $\beta$-amyloid (A$\beta$), a 38 to 42 residue protein that is produced normally as part of the cellular metabolism. \emph{Scyllo}-inositol is a promising potential therapeutic compound for AD treatment, which is currently in phase two of clinical trials. Both \emph{scyllo}-inositol and one of its stereoisomers, \emph{epi}-inositol, have been shown to effectively block the accumulation of A$\beta$ oligomeric assemblies and reduce AD-like symptoms in a transgenic mouse model of AD. Furthermore, \emph{in vitro}, \emph{scyllo}-inositol and its stereoisomers \emph{myo}-inositol, \emph{epi}-inositol, and \emph{chiro}-inositol, have been demonstrated to stabilize A$\beta$42 oligomers, disassemble preformed A$\beta$42 fibrils, and reduce A$\beta$42-induced neurotoxicity in a stereochemistry-dependent manner. However, the specific molecular interactions of inositol and its effect on A$\beta$ aggregation at the molecular level are not known. At present, experimental approaches lack the ability to determine the precise mechanistic mode(s) of action of inositol as the molecular structures of various intermediates in the amyloid aggregation pathway are not known. Moreover, observed intermediates morphologically distinct from fibrils are experimentally very difficult to detect and isolate.
 
Molecular dynamics simulations can provide the atomistic level of detail necessary to determine the interaction of inositol with specific intermediates in the pathway. Thus, the primary objective of my research is to use molecular dynamics simulations to elucidate the molecular basis of the structure-activity relationship of inositol by determining its effect on the structure, thermodynamics, and kinetics of the amyloid aggregation pathway. The proposed work will provide insight into developing novel, higher-efficacy derivatives with the ability to effectively prevent the onset and progression of AD.
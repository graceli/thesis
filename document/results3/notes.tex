
Protofibrillar morphology

% Fibril dynamics such as chain unwinding is not caused by ligand binding as the edge chains break apart occasionally in the control simulations.   

% (I mean to say that the protofibril flops around, rather than remaining as a rock solid beta-sheet) in the simulation % [Still vaguely written; need to be more specific;  why is this observation significant? ]. 

%A decrease in the number of hydrogen bonds, when compared to the inner peptide strands, suggesting that peptides at the edges of the protofibril are more mobile than non-edge peptides. 

% (Figure~\ref{fig:protofibril_dynamics}). -- this is not really the right figure for this statement.
% (Note need to be careful here because the end chains have less HBs also because they don't have flanking peptides; The fibril remained aggregated in the simulations.  Note that there could be other rare events, which I haven't looked into too much.  Need to describe and discuss the unravelling in more detail.)

% Note that if the aggregate doesn't get messed up ... it is unlikely that the secondary structure is messed up as well.
% By eye, looks like there is no major differences in water, glycerol, glucose and inositol (NEED more quantitative description)
% Figure: Number of residues in secondary structure versus time.  Y-axis is number of residues in a particular SS conformation.

\subsection{Comparison of binding modes}
% Contact map -- needed? Yes I think I should further quantitate their binding mode differences. Just by looking at the spatial maps, the differences are minor and need a trained eye.  Having a contact map by residue will be more rigorous. Also quantitate binding affinity per binding pocket. This helps define binding sites on the fibril more precisely. Overall I have more binding statistics than all of the other relevant studies out there, I should take advantage of this.

% TODO: create a tracker story
% The SDF indicates that binding to the edges are not even, that is on one side, there is more binding than the other. This is a convergence issue.  May need a section in the discussion discussing the convergence of the results.
% Furthermore, inositol (is it only inositol?) binds to the charged termini of the peptides. 

Discussion
% Outline
% Which residues are most frequently bound by each of the ligands?
% Mostly nonpolar or hydrogen bonding?
% Over all binding modes.  Use the spatial probability maps to demonstrate this
% Are there binding hot spots? If so where and what are they? [in the discussion relate these findings to the mechanism of amyloid formation]
% Do the molecules Cluster? If so, do cluster sizes differ? For example do glycerol and glucose cluster a lot less? If there are differences in this mechanism could have implications in the mechanism of amyloid formation.
% How do the results compare with my previous results [discussion?] \\
% Does any of these molecules target the fibril core? [discussion?] \\
% Is targeting the fibril core a plausible mechanism for breaking up amyloid fibrils? Address this. Previous studies don't really address them.  
% In the discussion, I think it will also help to have a comparisons with previous studies, and experimental studies, if applicable. Again, look into literature for updates.
%% ut-thesis.tex -- document template for graduate theses at UofT
%%
%% Copyright (c) 1998-2012 Francois Pitt <fpitt@cs.utoronto.ca>
%% last updated at 09:43 (EDT) on Fri  1 Jun 2012
%%
%% This work may be distributed and/or modified under the conditions of
%% the LaTeX Project Public License, either version 1.3c of this license
%% or (at your option) any later version.
%% The latest version of this license is in
%%     http://www.latex-project.org/lppl.txt
%% and version 1.3c or later is part of all distributions of LaTeX
%% version 2005/12/01 or later.
%%
%% This work has the LPPL maintenance status "maintained".
%%
%% The Current Maintainer of this work is
%% Francois Pitt <fpitt@cs.utoronto.ca>.
%%
%% This work consists of the files listed in the accompanying README.

%% SUMMARY OF FEATURES:
%%
%% All environments, commands, and options provided by the `ut-thesis'
%% class will be described below, at the point where they should appear
%% in the document.  See the file `ut-thesis.cls' for more details.
%%
%% To explicitly set the pagestyle of any blank page inserted with
%% \cleardoublepage, use one of \clearemptydoublepage,
%% \clearplaindoublepage, \clearthesisdoublepage, or
%% \clearstandarddoublepage (to use the style currently in effect).
%%
%% For single-spaced quotes or quotations, use the `longquote' and
%% `longquotation' environments.


%%%%%%%%%%%%         PREAMBLE         %%%%%%%%%%%%

%%  - Default settings format a final copy (single-sided, normal
%%    margins, one-and-a-half-spaced with single-spaced notes).
%%  - For a rough copy (double-sided, normal margins, double-spaced,
%%    with the word "DRAFT" printed at each corner of every page), use
%%    the `draft' option.
%%  - The default global line spacing can be changed with one of the
%%    options `singlespaced', `onehalfspaced', or `doublespaced'.
%%  - Footnotes and marginal notes are all single-spaced by default, but
%%    can be made to have the same spacing as the rest of the document
%%    by using the option `standardspacednotes'.
%%  - The size of the margins can be changed with one of the options:
%%     . `narrowmargins' (1 1/4" left, 3/4" others),
%%     . `normalmargins' (1 1/4" left, 1" others),
%%     . `widemargins' (1 1/4" all),
%%     . `extrawidemargins' (1 1/2" all).
%%  - The pagestyle of "cleared" pages (empty pages inserted in
%%    two-sided documents to put the next page on the right-hand side)
%%    can be set with one of the options `cleardoublepagestyleempty',
%%    `cleardoublepagestyleplain', or `cleardoublepagestylestandard'.
%%  - Any other standard option for the `report' document class can be
%%    used to override the default or draft settings (such as `10pt',
%%    `11pt', `12pt'), and standard LaTeX packages can be used to
%%    further customize the layout and/or formatting of the document.

%% *** Add any desired options. ***
\documentclass[12pt,onehalfspaced]{ut-thesis}

%% *** Add \usepackage declarations here. ***
%% The standard packages `geometry' and `setspace' are already loaded by
%% `ut-thesis' -- see their documentation for details of the features
%% they provide.  In particular, you may use the \geometry command here
%% to adjust the margins if none of the ut-thesis options are suitable
%% (see the `geometry' package for details).  You may also use the
%% \setstretch command to set the line spacing to a value other than
%% single, one-and-a-half, or double spaced (see the `setspace' package
%% for details).
% \usepackage{chapterbib}
\usepackage{outlines}

\usepackage[numbers,sort&compress]{natbib}
\usepackage{amsmath}
\usepackage[version=3]{mhchem}
\usepackage[mathletters]{ucs}
\usepackage[utf8x]{inputenc}
\usepackage{array}
\usepackage[normalem]{ulem}
\newcommand{\textsubscr}[1]{\ensuremath{_{\scriptsize\textrm{#1}}}}
\usepackage[breaklinks=true,linktocpage,colorlinks]{hyperref}
\usepackage{url}
\usepackage{graphicx}


\degree{Doctor of Philosophy}
\department{Biochemistry}
\gradyear{2013}
\author{Grace Li}
\title{Molecular Mechanism of $\beta$-amyloid Inhibition By Inositol}

%% *** NOTE ***
%% Put here all other formatting commands that belong in the preamble.
%% In particular, you should put all of your \newcommand's,
%% \newenvironment's, \newtheorem's, etc. (in other words, all the
%% global definitions that you will need throughout your thesis) in a
%% separate file and use "\input{filename}" to input it here.


%% *** Adjust the following settings as desired. ***

%% List only down to subsections in the table of contents;
%% 0=chapter, 1=section, 2=subsection, 3=subsubsection, etc.
\setcounter{tocdepth}{3}

%% Make each page fill up the entire page.
\flushbottom

%%%%%%%%%%%%      MAIN  DOCUMENT      %%%%%%%%%%%%
\begin{document}

%% This sets the page style and numbering for preliminary sections.
\begin{preliminary}

%% This generates the title page from the information given above.
\maketitle

%% There should be NOTHING between the title page and abstract.
%% However, if your document is two-sided and you want the abstract
%% _not_ to appear on the back of the title page, then uncomment the
%% following line.
\cleardoublepage

%% This generates the abstract page, with the line spacing adjusted
%% according to SGS guidelines.
\begin{abstract}
%% *** Put your Abstract here. ***
%% (At most 150 words for M.Sc. or 350 words for Ph.D.)
Lorem ipsum dolor sit amet, consectetur adipiscing elit. Nunc faucibus vulputate dui, sed molestie tortor tincidunt eget. Aenean commodo mi cursus purus ornare condimentum. Proin ante dui, hendrerit quis convallis ac, venenatis nec dolor. Class aptent taciti sociosqu ad litora torquent per conubia nostra, per inceptos himenaeos. In hac habitasse platea dictumst. Nam id risus id purus porta porttitor in ut nisi. Class aptent taciti sociosqu ad litora torquent per conubia nostra, per inceptos himenaeos. Fusce eget quam neque. Nulla a sem ante, id ornare diam. Suspendisse viverra aliquet fringilla. Etiam sit amet ipsum nec eros dapibus luctus. Proin felis purus, consequat eget fringilla ut, faucibus vitae nisi.

Nulla facilisi. Duis vitae enim tortor. Nam sollicitudin sapien non tellus ullamcorper ac ornare sapien condimentum. Nunc dignissim nulla orci. Phasellus vel dolor ac sem tristique malesuada. Morbi ullamcorper tincidunt nisi. Etiam purus massa, scelerisque at hendrerit a, fermentum id sapien. Morbi nec nisl vitae dolor molestie venenatis sed sed neque. Pellentesque habitant morbi tristique senectus et netus et malesuada fames ac turpis egestas. Nulla mollis, arcu ut varius dapibus, est lacus pretium ante, et lobortis dolor sapien ac lacus. Vestibulum elit elit, ultricies non sodales quis, porta at metus. Pellentesque condimentum viverra nunc in euismod. Mauris nec vulputate dolor. Sed semper, risus vel viverra aliquet, ante mauris facilisis tortor, a pulvinar nisl mi vel velit.

Ut eu egestas enim. In hac habitasse platea dictumst. Suspendisse in enim magna, nec euismod odio. Morbi nec viverra leo. Mauris eros nibh, faucibus non gravida et, eleifend quis diam. Maecenas nulla lacus, accumsan ut venenatis sit amet, accumsan sagittis urna. Sed a metus ut dolor volutpat faucibus a non ante.

Praesent quam elit, interdum in tempor vel, hendrerit ac nulla. Pellentesque habitant morbi tristique senectus et netus et malesuada fames ac turpis egestas. Vestibulum volutpat mollis nisl eget mollis. Ut pharetra dolor commodo nunc bibendum placerat. Pellentesque vehicula arcu eget urna facilisis bibendum. Proin vestibulum, enim id interdum lacinia, purus quam mollis orci, semper ultrices dolor eros non neque. Etiam sollicitudin vehicula lectus in auctor. Cras fringilla lectus velit. Sed eu pulvinar lectus. Praesent quis.
\end{abstract}

%% Anything placed between the abstract and table of contents will
%% appear on a separate page since the abstract ends with \newpage and
%% the table of contents starts with \clearpage.  Use \cleardoublepage
%% for anything that you want to appear on a right-hand page.

%% This generates a "dedication" section, if needed
%% (uncomment to have it appear in the document).
% \begin{dedication}
% % *** Put your Dedication here. ***
% This thesis is dedicated to my grandparents, and my mother.


% \end{dedication}

%% The `dedication' and `acknowledgements' sections do not create new
%% pages so if you want the two sections to appear on separate pages,
%% you should put an explicit \newpage between them.

%% This generates an "acknowledgements" section, if needed
%% (uncomment to have it appear in the document).
% \begin{acknowledgements}
% % *** Put your Acknowledgements here. ***
% First and foremost, I am grateful to my supervisor, Regis Pomes for his wisdom, insight, and guidance for the past 6 years of my life.

Secondly, I thank the members of my lab, both past and present, for all their moral support, and assistance with my work. 
I would like to thank the early members of the lab Drs. Chris Madill, Chris Neale, Sarah Rauscher and Tomas Rodinger for setting an example for me of how to be an excellent student of science. Not only were they brilliant scientists, they have provided the lab with invaluable insight, and technical expertise.

% advice and support which have had a profound influence on my perspective on life.
Nilu Chakrabarti, David Caplan, John Holyoake, Loan Huynh, Kethika K.

% Thank Mr. Zimmerman

% Thank Dr. Nasir Memon

% Thank SciNet & Team
% The computations presented in this thesis were resource intensive and I could not have accomplished any of it without generous allocations from the center for computational biology (CCB) at Sickkids and the following high-performance computing (HPC) consortia of Compute Canada: SciNet, SHARCNET, CLUMEQ, WestGrid, and RQCHP.
% My use of these computing resources was only possible because of the excellent support staff at each of these consortia. While there are perhaps one hundred people who have contributed to this thesis in this respect, I would like to mention some of those people who have provided exceptional levels of support.

Finally, I thank my mom and my grandparents for all of their support over the years. Without them I would not be here today.


% \end{acknowledgements}

%% This generates the Table of Contents (on a separate page).
\tableofcontents

%% This generates the List of Tables (on a separate page), if needed
%% (uncomment to have it appear in the document).
%\listoftables

%% This generates the List of Figures (on a separate page), if needed
%% (uncomment to have it appear in the document).
\listoffigures

%% You can add commands here to generate any other material that belongs
%% in the head matter (for example, List of Plates, Index of Symbols, or
%% List of Appendices).

%% End of the preliminary sections: reset page style and numbering.
\end{preliminary}


%%%%%%%%%%%%%%%%%%%%%%%%%%%%%%%%%%%%%%%%%%%%%%%%%%%%%%%%%%%%%%%%%%%%%%%%
%%  Put your Chapters here; the easiest way to do this is to keep     %%
%%  each chapter in a separate file and `\include' all the files.     %%
%%  Each chapter file should start with "\chapter{ChapterName}".      %%
%%  Note that using `\include' instead of `\input' will make each     %%
%%  chapter start on a new page, and allow you to format only parts   %%
%%  of your thesis at a time by using `\includeonly'.                 %%
%%%%%%%%%%%%%%%%%%%%%%%%%%%%%%%%%%%%%%%%%%%%%%%%%%%%%%%%%%%%%%%%%%%%%%%%

% A Doctoral thesis is generally organized into the following sections:
% Introduction: Although the length of the introduction is somewhat field-specific, in most cases 40-60 double-spaced pages of text should be sufficient. The last section of the General Introduction in either format should provide a clear rationale for the thesis project.
% Title Pages and Data Attribution: If any one other than the student has contributed data to the thesis the student should clearly state this on the title page of the relevant Results Chapter(s). The student should indicate the nature of the contribution (e.g. technical assistance under the student’s guidance; independent design and interpretation of specific experiments in the Chapter etc.). If the work has been published or submitted, the full citation should also be given on this page.
% The student must obtain any required copyright permissions, which may be needed for work that has been published. Journals often require specifically-worded citations to be included on the first page of the relevant Results Chapter. The student will need to submit the copyright permissions to SGS when submitting the final thesis.
% Methods: Sufficient details of the methods should be given such that the research could be readily reproduced.
% Figures: The quality of halftone figures must allow for unambiguous assessment of the data.
% Bibliography: Most examiners prefer that the bibliography include titles.
% Appendices: Appendices may be included to cover such topics as additional details pertaining to Methods; speculative ideas; projects-in-progress; and/or negative results, particularly if the second format is chosen.

\chapter{Introduction}

% Organizational detail: AD and amyloid (which one should go first?). Center it on Amyloid, and then bring in AD as a central motivation for doing amyloid science.

% I think my focus should still be on Alzheimer's disease ... the practical purpose of my work is to understand how inositol works
% I am not trying to cure all amyloid diseases.
\section{Alzheimer's Disease}
% Here, my intention is to lead into the discussion of amyloid by giving a historical perspective and overview of AD
% And use AD as a motivation for why so much work has gone into studying amyloids.
\begin{outline}[enumerate]
	\1 More than a hundred years have pass since Dr. Alois Alzheimer first presented the connection between the presence of neuronal plaques and the clinical symptoms of presenile dementia characteristic of Alzheimer's disease (AD).

	\1 Today, AD is known to be the most common cause of dementia in persons of age 65 or older. With the increasing longevity of our population, AD is approaching epidemic proportions with no cures or preventative therapy available.\cite{Blennow:2006wd}

	% Elaborate on happened between the step above and the amyloid hypothesis?
	\1 The discovery of amyloid plaque deposits of the brains of deceased dementia patients led to the formulation of the amyloid hypothesis which posits that amyloid aggregates initiates the pathogenesis of AD, whereas the other pathological symptoms such as neurofibrillary tangles are secondary.

% \2 A$\beta$ is produced by intramembrane proteolytic cleavage of the larger amyloid-$\beta$ precursor protein (APP) by $\beta$-secretase, and is produced constitutively as part of the normal cellular metabolism.\{Selkoe, 2002 \#222\} Depending on the position of the cleavage, A$\beta$ peptides of lengths varying from 38 to 43 residues can be produced. However, the peptides spanning residues 1--40 (A$\beta$40) or 1--42 (A$\beta$42) are predominantly found AD-associated plaques.
\end{outline}

\subsection{Other amyloid diseases}
Many diseases have been identified to be linked with the presence of amyloid: Type II diabetes, Prion-related diseases, Parkinson's, Huntington's disease etc.


\section{Amyloid: formation and mechanism of toxicity}
  % In this section I will talk about how amyloid aggregation is thought to work. Introduce the thermodynamic model for understanding fibril formation. I will now broadly introduce to amyloid.  
  \begin{outline}[enumerate]
    \1 Although initial studies of amyloid was focused on understanding the etiology of AD, amyloid science now have grown into its own field.

    % \1 Finding a treatment for AD and other fatal neurodegenerative diseases motivated many biochemical and biophysical studies of the amyloid state.  
    \1 In the laboratory, a variety of proteins and peptides, both folded and disordered, have been shown to form amyloid fibrils via chemical modification of solution conditions and denaturation. It is currently thought that amyloid fibrillar state may be the globally stable folded state for all proteins.

    \1 Amyloid fibrils are formed via a complex aggregation pathway. Initially, monomers aggregate to form oligomers with different morphologies which exists in equilibrium with amyloid fibrils. Some of these oligomers are on-pathway to fibril formation, while others themselves may be end-points of the aggregation pathway. Amyloid fibrils, typically the visible endpoint of aggregation, has a cross-$\beta$ structure.
  \end{outline}

 \subsection{Structure of Fibril}
   \begin{outline}
    % \1 Add details on the definition of the cross-$\beta$ structure and its significance.
    \1 Decades after the initial discovery by Alois Alzheimer, A$\beta$, the central protein component of neuronal plaques, was synthetically produced in the laboratory. In vitro, A$\beta$ was found to precipitate out of solution almost immediately. 
  		\2 Describe the molecular structure of A$\beta$ amyloid fibrils
  			\3 Define the cross-$\beta$ structure [Show a schematic here].
  		\2 Briefly mention the techniques that can be used to obtain structural information of amyloid fibrils.
		\end{outline}

  \subsection{Structure of Non-fibrillar oligomers}
   Because many of the dynamic and disordered nature of oligomeric aggregates, the structure of oligomers have been challenging to obtain via experimental protein structural determination methods.
	
  \subsection{Kinetics}
  Amyloid fibrils have been observed to form via a nucleation-polymerization process. In the nucleation phase, energetic barriers of aggregation must be overcome to form the initial aggregation nucleus or seed.  Following nucleation, free monomers bind to the nucleated aggregates and polymerize into mature fibrils.\cite{Murphy:2002fe}
    
  \subsection{Toxicity}
  \begin{outline}
  	% Key question in the field: What is the toxic species?
  	\1 Multiple lines of research have identified oligomers as a likely causative agent for neuronal cell death. By contrast, the monomeric and fibril forms are thought to be less toxic than oligomers. It is hypothesized that soluble oligomers may cause toxicity by perturbing the integrity of cellular membranes through binding and disrupting the lipid bilayer (perhaps by making them ion permeable). \cite{Walsh:2007fu}
  	\1 Include a paragraph about amyloid formation and lipid membranes (?)
  	% Understanding the toxicity or finding out whether there is a toxic species in part validates the amyloid hypothesis. 
  \end{outline}


\section{Amyloid Inhibition: A promising treatment for amyloid disorders}
% Cure, method of prevention; is there hope?
\begin{outline}
	\1 In this section, I will provide an overview of some of the challenges to overcome when developing a small molecule therapeutic for Alzheimer's disease.  Furthermore, using this information, I will motivate why inositol is an exciting avenue to explore.
	  \2 scyllo-Inositol is able to cross the blood brain barrier. It has high bioavailability. Because it is not broken down in the gut, it can be taken orally.
	  \2 Inositol is not toxic to the human body.  Inositol is used in signaling pathways.

	\1 Briefly mention non-small molecule putative therapies which also acts via amyloid inhibition. The focus of this thesis will be on small-molecule amyloid inhibition.
\end{outline}

\subsection{Molecular mechanisms of amyloid inhibition 
            \\ by small molecules}
\begin{outline}[enumerate]
    \1 Amyloid inhibition as a treatment for Alzheimer's disease and related amyloid disorders. Amyloids are attractive drug targets. Small molecules may be one effective way to develop a treatment for amyloid disorders because they have the potential to be able to treat the underlying disease. Through in vitro screening, many small molecules have been found to effect the amyloid aggregation pathway.  Some were demonstrated to inhibit amyloid fibrils, where as others were shown to arrest or reduce oligomer formation.   
      % Here I can take a cue from Justin Lemkul`'s recent review paper.
      % Talk about the different kinds of small molecules that have been found to inhibition amyloid formation.  Here I will also provide a summary of what people know about the mechanism by which they inhibit amyloid formation.
      
      \2 Pharmacological perspective of the challenge of developing an Alzheimer's drug. In order to effectively treat Alzheimer's and other neurodegenerative diseases, small molecule drug candidates must pass the blood brain barrier at sufficient concentrations for inhibition.  This is difficult to achieve.
      \2 In vitro screening has led to the discovery of a large number of small-molecules which were found to affect the amyloid aggregation pathway. Many of these small molecules are thought to act by directly binding to amyloidogenic peptides and aggregates.
        \3 Thioflavin T and Congo red are dye molecules used to identify the presence of amyloid.  Both molecules bind to mature amyloid fibrils and have been shown to affect fibril formation. (Figure Part X)

        \3 Polyphenols is a large group of natural and synthetic molecules.  (−)-epigallocatechin-3-gallate, curcumin, and a polyphe- nolic grape seed extract are some that was found to inhibit fibril formation. Are also known for their anti-oxidant properties. (Figure Part Y) 
      
      \2 Small molecule inhibitors share common chemical features and groups.  They are typically planar in geometry, have many aromatic rings, and polar functional groups (hydroxyl groups) around the edge of these aromatic rings.
    
    	\2 Mechanism of action. Some small molecules inhibit fibril formation, where as others may prevent oligomerization, but not fibrillation. A high concentration is often required to observe activity (micromolar to millimolar), which suggests that they may be non-specific inhibitors. EGCG, one such polyphenol, is known to have the lowest/highest IC50.
    	% IC50 -- This quantitative measure indicates how much of a particular drug or other substance (inhibitor) is needed to inhibit a given biological process (or component of a process, i.e. an enzyme, cell, cell receptor or microorganism) by half.
      % EC50 -- The term half maximal effective concentration (EC50) refers to the concentration of a drug, antibody or toxicant which induces a response halfway between the baseline and maximum after some specified exposure time.[1] It is commonly used as a measure of drug's potency.
      % Ref: wikipedia
	
	\1 Inositol molecules
	
		\2 The discovery of inositol
		
		\2 Role of inositol in the human body 
		
			\3 \emph{myo}-inositol
			
		\2 Where is inositol found
		
		\2 Role of inositol in amyloid inhibition
		
			\3 in vitro studies
			
			\3 in vivo studies
\end{outline}

\subsection{Analogy to Sugar-protein binding}
% Does this section fit here? Where should I put it?
\subsubsection{Experimental techniques to study sugar-binding modes}
\subsubsection{Sugar Binding modes}


\section{Structure-based Drug Discovery}
\subsection{Forces in protein-ligand interactions}
\begin{outline}
	\1 Protein-ligand non-covalent interactions which are important for binding and recognition.
		\2 Electrostatic interactions: Polar and charge-charge interactions
			\3 Hydrogen bonding
			   % Here, it will benefit me to read Sarah's appendix C carefully.
		\2 Nonpolar (hydrophobic) interactions
		  \3 Van der Waals
\end{outline}

\subsection{Protein-Ligand binding}
% subsection protein_ligand_binding_theory (end)
% Below is a summary of an excerpt from Tom's thesis on structure-based drug discovery.
% Design of antibiotics 
% 1) Target determination (biochemical)
% 2) Structural determination (Xray, NMR, or homology); active site identified; Here would be useful to get the holo structure of the protein
% 3) Screen for inhibitors against a chemical library or in silico docking.
\begin{outline}
	\1 Enzyme and the ligand must bind tightly and specifically, so to avoid large drug doses to inhibit the enzyme, which may have adverse side effects (toxicity) in the human body.
	\1 Binding constant is a measure of the affinity of a ligand to a protein. It is the concentration at which 50\% of the drug is bound to the protein. In experimental studies, $K_d$ is often used to quantitatively identify potential binders. A small $K_d$ suggests that the ligand may bind tightly to the protein.
	\1 Binding free energies
		\2 Absolute binding free energy
		\2 Relative binding free energy
		\2 Binding equilibrium
			\[ protein+ligand <-> (protein-ligand)aq \]
	\1 Experimental techniques for determining $K_d$. 
		\2 What experimental techniques are used to estimate binding affinity? (Study up on this.)
\end{outline}	

\subsection{Role of chirality in drug binding}
Definition of chirality [Add schematic] ... etc

\addcontentsline{toc}{section}{Bibliography}
\bibliographystyle{plain}
\bibliography{chapter1}

% Model peptides
% \include{chapter2}

% % KLVFFAE
% \include{results-chapter-2}
% 
% % Abeta42
% \include{results-chapter-3}
% 
% % PGab
% \include{results-chapter-4}
% 
% % Discussion / Perspectives
% \chapter{Perspectives, Conclusions and Future Directions}
% Example sentences found in teh conclusions and summary chapters of people's thesis 
% The work presented in this thesis has ...

% Because this work represents the first atomistic simulation, to our knowledge, demonstrating that polypeptide chains can form entangled polymer melt-like states, it contributes to an improved understanding of both elastin coacervation, and the more general phenomenon of protein aggregation

% Everyone's conclusions all had a significant amount of Future directions. (~3 pages worth)
% Strategy:  take the conclusion paragraphs of each chapter and then meld it into a conclusions chapter.

In chapter 2, we have performed systematic simulations of simple amyloidogenic peptide models with both active and inactive stereoisomers of inositol to examine the molecular basis of amyloid inhibition. Our results indicate that although peptide backbone dominates the interaction with inositol, the binding affinity is low and remains in the millimolar range. Moreover, this property is independent of stereochemistry and does not appear to be sufficient to impede peptide dimerization through intermolecular backbone hydrogen bonding. Taken together, our results suggest that amyloid inhibition by inositol cannot be accounted for by generic binding to the peptidic backbone alone and is likely to involve sequence-specific interactions with amino-acid side chains as well as binding to specific aggregate morphologies. Accordingly, although the formation of intermolecular hydrogen bonds is the predominant interaction in protein aggregates composed of \gafour, amyloidogenic peptides involved in amyloid diseases are often more hydrophobic and in general, self-aggregation is driven largely by the hydrophobic effect.\cite{Chiti:2006p20} 

% In forthcoming studies, we will examine the role of sequence-specific interactions between inositol and aggregates of pathogenic peptides.

In chapter 3, we have examined the binding of  \emph{scyllo}-inositol, and its inactive stereoisomer, \emph{chiro}-inositol, successively to monomers, disordered oligomers, and $\beta$-sheet aggregates of A$\beta$(16-22), whose sequence is thought to be the core aggregation region in the A$\beta$42 peptide. Notably, the $K_{eq}$ of inositol ($\sim$0.2 - 0.5 mM) for the $\beta$-oligomer is commensurate with the concentration at which inhibition of amyloid formation by A$\beta$42 is observed \emph{in vitro}. Although both \emph{scyllo}- and \emph{chiro}-inositol exhibit similar binding affinities with all peptide states considered, we have uncovered a stereospecific face-to-face stacking stacking mode of \emph{scyllo}-inositol with the Phe side chains and a higher propensity for hydrogen bonding, which together suggests a molecular basis for measured differences in activity.  Cooperative binding modes of inositol at grooves on the surface of the $\beta$-oligomer of A$\beta$(16-22) suggest a possible mechanism of fibril inhibition whereby inositol prevents the lateral association or stacking of protofibrillar $\beta$-sheet oligomers. Furthermore, our results suggest that the fibril core of A$\beta$ amyloid aggregates contains carbohydrate-like binding sites. As such, carbohydrate-based small-molecule derivatives may be a promising avenue to explore for the rational design of novel therapeutics for AD.


\section{Significance for drug development for AD}
%[Thesis significance - part of the summary of what I make of my thesis ] 
This thesis demonstrates the applicability of MD simulations in providing insight into how drugs might bind to amyloidogenic species and intrinsically disordered peptides.  The results in this thesis can be used to map out a pharmacophore for developing a drug for the treatment of Alzheimer's Disease, opening the way to computer-aided design of improved diagnostics and therapeutics. A pharmacophore is an abstract description of molecular features which are necessary for molecular recognition of a ligand by a biological macromolecule - If I have to define this here, then it should have been in the introduction. \textbf{Definition is taken from the wiki.}

% Speculate on the future of drug development for AD in regards to the significance of thesis wrt curing AD. Will blocking aggregation work for AD? - May be this should in the introduction instead.
% (WRITE SOME CONCLUSIONS HERE … RELATING TO HOW A SINGLE SMALL MOLECULE BINDS SPECIFIC MORPHOLOGICAL STRUCTURES -- clearly a key result of my works demonstrate that there is sequence specificity) 

Because AD may be caused by a multitude of pathological changes in the brain, it is likely that a cocktail of compounds, each targeting a different disease pathway, may be required for treating AD. % (Adapted from pharmacophore for AD 2011)

% \section{MD simulations and Rational drug design}
\section{Contribution to sugar-protein binding}
% MD simulation as a tool for probing weak interactions
The work presented in this thesis have demonstrated that the methodology used in this thesis may be generally applicable to understanding carbohydrate-protein interactions. In chapter 4, we used simple sugars glucosamine and GlcNAc to map out a binding surface on PgaB, a protein invovled in the biofilm formation pathway. 

This thesis has demonstrated that MD simulations, and the current force fields can effectively probe weak and transient molecular interactions, which are often not readily detectable using experimental techniques. One prominent example where weak interactions are prominent in protein-ligand binding is protein-carbohydrate interactions.\cite{weak binding review paper}

Understanding protein-sugar interactions is an important endeavor because of antibody binding to proteins.  Viruses often express carbohydrates on their coats.  Inhibiting bacterial action involves knowing how polysaccharrides are expressed, which involves binding proteins.  The results of this thesis presents methods that may be useful for developing antibiotics. \textbf{Rewrite this part to reflect and tie into how my work, methods, and how they are related to solving these problems} 

% The role of simulations in drug discovery to effectively predict binding modes and binding sites

[Thesis significance; more perspectives] Traditionally computation studies probing protein-ligand binding is often carried out with the knowledge of a putative binding site (often determined by X-ray crystallography) and the mechanism is examined employing sophisticated methods using the ligands thought to be able to bind in this specific pocket, while ignoring all other possibilities.  However, with the computing power to extend simulations to a longer time scale, MD simulations may be accurate in discovering new binding sites (even sites that are XXX high in affinity) without prior assumptions about the binding site. Our studies are among those which demonstrate the utility of MD to probe for binding sites that are difficult to obtain via experimental structural determination methods.  A recent MD study have demonstrated the capability of MD in binding site prediction for a folded protein.\cite{Shan:2011bo}

% Shan, 2011 (the DE shaw letter) An emerging challenge in drug discovery concerns the identification of allosteric ligand-binding sites (I’m not entirely convinced yet that this is important -- because I can’t think of any situations where this might be important -- convince myself or just drop this idea), through which drugs can modulate the effects of ligands that bind at the primary site.  More generally, an important limitation of traditional virtual drug screening is that it must start with a well-defined binding site (look into limitations of current drug discovery processes … may give some clue for constructing an argument for how MD is helping …), despite promising recent developments.\cite{Hetenyi, C.; van der Spoel, D. FEBS Lett. 2006, 580, 1447. (11) Davis,I.W.;Raha,K.;Head,M.S.;Baker,D.ProteinSci.2009, 18, 1998. }

\section{Significance for disordered binding}
% [Conclusions - A general interest in the specific work that I have done here ] 
Aside from contributing to the design of future AD therapeutics, our results have also contributed to the understanding of small molecule binding to disordered peptides (IDPs) in general. This is important because disordered binding are ubiquitous in biology, and disordered peptides and proteins and involved in many diseases such as X, Y, Z.  

Understanding how small molecules may interact with intrinsically disordered proteins goes beyond amyloid-related disorders. As IDPs are involved in many signalling pathways, they are also viable drug targets for many diseases.   Hence, targeting these disordered peptides using small molecules may be a possible therapeutic approach. For example, c-Myc is frequently involved in many cancers, \textbf{FILL IN THE GAP ABOUT HOW IT WORKS} and thus disruption of the c-Myc–-Max interaction is a possible anticancer strategy.\cite{Iakoucheva:2002uv,Metallo:2010p6822,Cuchillo:2012bm}

The work presented in chapters in this thesis sheds light on the mechanism of binding disordered peptides, and represents a step forward in understanding how small molecules may prevent protein - protein interactions, which involves identifying the binding interfaces, and understanding how to target these interfaces using small molecules.REFs.

\section{Osmolyte effects, denaturation, and macromolecular crowding}
% Note this is another hairy field … and you might want to stay the hell away from it, despite the fact that your thesis is loosely connected to this field
Another field of interest where weak interactions dominate is cosolvent effects on peptide folding. Simulations are a good use for probing that.  For example, MD simulations have been useful in gaining insight into protein denaturation mechanisms by urea or guanidinium, and the activity of osmolytes. Many of these are still open questions.\cite{http://pubs.acs.org/doi/abs/10.1021/jp200625k -- crowding and protein association.} Where am I going with this?

% These ideas below are  lesser developed ideas … consider cutting or bulk up …
% \section{other areas that are related to my work}
% [Ab-GAG membrane binding] This branches off the fact that I looked at sugar binding with peptides - amyloids when deposited may interact with glycosaminoglycans (part of the extracellular matrix) exposed at cellular surfaces. So what about it? Is my work helping to understand how amyloids are interacting with GAGs? or what role GAGs might play in accelerating amyloid formation?


% \subsection{Relationship to Polypharmacology -- where one drug binds to different targets?}
% Not sure how my thesis relates to this.

% I think this section is too crazy - eliminate.
%\section{In the future - perspectives on computer simulations}
% TODO: Find out why are there so few new drugs being discovered nowadays?
% Expand simulations into the macroscopic level 

% TODO: A good thought experiment - If I had the perfect simulation system? Predictive force field, simulate milliseconds in days.  How could I use this simulation system? Some effective use of the data? 

%Software has the ability to revolutionize drug discovery and take it from bench to personalized medicine. Molecular simulations can play a role in driving experimentation by helping to generate testable hypotheses.
%
%Virtual drug screening method using MD simulations as a component predicting drug toxicity.
%
%Here I'm speculating what the future of MD simulations might be ... A bit like science fiction with a touch of reality.
%
%Most Useful to “somehow” integrate experimental data with simulations results\cite{that nature paper discussing integrating MD and systems biology}
%- What are some success stories?
%what’s the real problem with drugs?
%Can we do without drugs? What are the therapies available?
%Small molecule
%peptide
%antibodies
%gene therapy (viral)
%\cite{Hansen:2012hh}

\section{Future directions}
Better methods? Longer simulation times? Better force fields to detect hydrophobic effect?
Better understanding of the amyloid aggregation mechanism

Simulations can be a good tool when combined with experimental validation by using a variety of techniques SSNMR and other biophysical techniques used to probe amyloid systems. Several studies are beginning to do that to understand the molecular mechanism of small molecule inhibitors ECGC.



% How to separate references?
% Combined bibliography ? 

% Leave this to the last minute
% Combine Introduction -- how ?

%% This adds a line for the Bibliography in the Table of Contents.
\addcontentsline{toc}{chapter}{Bibliography}
%% *** Set the bibliography style. ***
%% (change according to your preference/requirements)
\bibliographystyle{plain}
%% *** Set the bibliography file. ***
%% ("thesis.bib" by default; change as needed)
% \bibliography{thesis}

%% *** NOTE ***
%% If you don't use bibliography files, comment out the previous line
%% and use \begin{thebibliography}...\end{thebibliography}.  (In that
%% case, you should probably put the bibliography in a separate file and
%% `\include' or `\input' it here).

\end{document}

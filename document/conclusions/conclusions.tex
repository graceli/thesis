\chapter{Perspectives, Conclusions and Future Directions}
% Example sentences found in teh conclusions and summary chapters of people's thesis 
% The work presented in this thesis has ...

% Because this work represents the first atomistic simulation, to our knowledge, demonstrating that polypeptide chains can form entangled polymer melt-like states, it contributes to an improved understanding of both elastin coacervation, and the more general phenomenon of protein aggregation

% Everyone's conclusions all had a significant amount of Future directions. (~3 pages worth)
% Strategy:  take the conclusion paragraphs of each chapter and then meld it into a conclusions chapter.

% Make a clear and concise statement of the original contribution to knowledge found in your thesis.

%What has this work led to?
%This has led to the understanding of mechanism of amyloid formation.
%An exploration of carbohydrate binding.


% One sentence summarizing AD and the medical challenges that it poses.  Summarize why it is difficult to design a drug for AD. 
% Summary of the core technique that is used in this thesis
Alzheimer?s Disease (AD) is a devastating neurodegenerative disease that is the most common cause of dementia in persons of age 65 or older. Currently there is no cure or method of treatment that targets the underlying disease.  As the world's population is living ages, AD will reach epidemic levels and will pose a tremendous medical burden for society. The overall goal of this thesis work is to investigate the molecular basis of amyloid inhibition by inositol, a small-molecule putative therapeutic for the treatment of Alzheimer's disease. By utilizing molecular dynamics simulations, a computational technique based on classical physics which allows us to simulate the motions of physical systems at the atomistic-level, we systematically examine model amyloidogenic peptides and aggregates to more complex systems involving the full-length A$\beta$42 peptide. The work presented in this thesis contributed to 2 peer-reviewed articles and 2 manuscripts in preparation which form the basis of Chapters 3 to 6.  The key results from each of these chapters are summarized below.

Beginning in chapter 2, I performed systematic simulations of simple amyloidogenic peptide models with scyllo- and chiro-inositol, stereoisomers of inositol, to examine the role of backbone binding on amyloid inhibition. My results indicated that although peptide backbone dominates the interaction with inositol, the binding affinity is low and remains in the millimolar range. Moreover, this property is independent of stereochemistry and does not appear to be sufficient to impede peptide dimerization through intermolecular backbone hydrogen bonding. Taken together, my results in this chapter suggest that amyloid inhibition by inositol cannot be accounted for by generic binding to the peptidic backbone alone. Rather, it is likely to involve sequence-specific interactions with amino-acid side chains as well as binding to specific aggregate morphologies.
% Accordingly, although the formation of intermolecular hydrogen bonds is the predominant interaction in protein aggregates composed of \gafour, amyloidogenic peptides involved in amyloid diseases are often more hydrophobic and in general, self-aggregation is driven largely by the hydrophobic effect.\cite{Chiti:2006p20}

To further investigate the role of sequence-specific interactions between inositol and aggregates of pathogenic peptides, I examined the binding of  inositol stereoisomers, successively to monomers, disordered oligomers, and $\beta$-sheet aggregates of A$\beta$(16-22), whose sequence is thought to be the core aggregation region in the A$\beta$42 peptide (chapter 4). A key finding of this study was that the $K_{eq}$ of inositol ($\sim$0.2 - 0.5 mM) for the $\beta$-oligomer is commensurate with the concentration at which inhibition of amyloid formation by A$\beta$42 is observed \emph{in vitro}. Although both \emph{scyllo}- and \emph{chiro}-inositol exhibit similar binding affinities with all peptide states considered, my simulations have uncovered a stereospecific face-to-face stacking stacking mode of \emph{scyllo}-inositol with the Phe side chains and a higher propensity for hydrogen bonding, which together suggests a molecular basis for measured differences in activity. \textbf{Cooperative binding modes of inositol at grooves on the surface of the $\beta$-oligomer of A$\beta$(16-22) suggest a possible mechanism of fibril inhibition whereby inositol prevents the lateral association or stacking of protofibrillar $\beta$-sheet oligomers.} Furthermore, my results suggest that the fibril core of A$\beta$ amyloid aggregates contains carbohydrate-like binding sites.  \textbf{As such, carbohydrate-based small-molecule derivatives may be a promising avenue to explore for the rational design of novel therapeutics for AD.}

In chapter 4, I investigate the binding of inositol to the protofibrillar form of A$\beta$42, the full-length peptide. Here, I found that there were no differences in the fibril conformations with and without inositol in either low or high molar ratio. 
% Does glucose bind more or less? If it does bind less, then it tells us that we might be onto something with scyllo-inositol even though the structure is subtly different.  
% Not sure what I meant here
Glucose does not necessarily bind ``less", but it does not bind on the right face ie. the KLVFFAE face Scyllo-inositol appears to preferentially bind the KLVFFAE face, more than glucose and chiro-inositol.  Furthermore, scyllo-inositol does not bind in a hydrated tunnel formed in the protofibrillar aggregate, where as both chiro-inositol and glucose Binding mode differences between active and inactive inhibitors of Abeta suggest a mechanism of inhibition.

In chapter 5, I demonstrated the generality of the methodology that was developed to study the binding mechanism of inositol by using a similar approach to study carbohydrate-protein binding. Specifically, ....  

\section{Significance for drug development for AD}
%[Thesis significance - part of the summary of what I make of my thesis ] 

There are a multitude of challenges in understanding the mechanism of a drug for treating neurodegeneration

This thesis demonstrates the applicability of MD simulations in providing insight into how drugs might bind to amyloidogenic species and intrinsically disordered peptides.  The results in this thesis can be used to map out a pharmacophore for developing a drug for the treatment of Alzheimer's Disease, opening the way to computer-aided design of improved diagnostics and therapeutics. A pharmacophore is an abstract description of molecular features which are necessary for molecular recognition of a ligand by a biological macromolecule - If I have to define this here, then it should have been in the introduction. \textbf{Definition is taken from the wiki.}

% Speculate on the future of drug development for AD in regards to the significance of thesis wrt curing AD. Will blocking aggregation work for AD? - May be this should in the introduction instead.
% (WRITE SOME CONCLUSIONS HERE … RELATING TO HOW A SINGLE SMALL MOLECULE BINDS SPECIFIC MORPHOLOGICAL STRUCTURES -- clearly a key result of my works demonstrate that there is sequence specificity) 

Because AD may be caused by a multitude of pathological changes in the brain, it is likely that a cocktail of compounds, each targeting a different disease pathway, may be required for treating AD. % (Adapted from pharmacophore for AD 2011)

% \section{MD simulations and Rational drug design}
\section{Contribution to sugar-protein binding}
% MD simulation as a tool for probing weak interactions
The work presented in this thesis have demonstrated that the methodology used in this thesis may be generally applicable to understanding carbohydrate-protein interactions. In chapter 4, we used simple sugars glucosamine and GlcNAc to map out a binding surface on PgaB, a protein invovled in the biofilm formation pathway. 

My work has demonstrated that MD simulations in combination with the use of current force fields can be effectively used to probe weak and transient molecular interactions, which are often not readily detectable using experimental techniques. One prominent example where weak interactions are prominent in protein-ligand binding is protein-carbohydrate interactions.\cite{weak binding review paper}

Understanding protein-sugar interactions is an important endeavor because of antibody binding to proteins.  Viruses often express carbohydrates on their coats.  Inhibiting bacterial action involves knowing how polysaccharrides are expressed, which involves binding proteins.  The results of this thesis presents methods that may be useful for developing antibiotics. \textbf{Rewrite this part to reflect and tie into how my work, methods, and how they are related to solving these problems} 

% The role of simulations in drug discovery to effectively predict binding modes and binding sites

[Thesis significance; more perspectives] Traditionally computation studies probing protein-ligand binding is often carried out with the knowledge of a putative binding site (often determined by X-ray crystallography) and the mechanism is examined employing sophisticated methods using the ligands thought to be able to bind in this specific pocket, while ignoring all other possibilities.  However, with the computing power to extend simulations to a longer time scale, MD simulations may be accurate in discovering new binding sites (even sites that are XXX high in affinity) without prior assumptions about the binding site. Our studies are among those which demonstrate the utility of MD to probe for binding sites that are difficult to obtain via experimental structural determination methods.  A recent MD study have demonstrated the capability of MD in binding site prediction for a folded protein.\cite{Shan:2011bo}

% Shan, 2011 (the DE shaw letter) An emerging challenge in drug discovery concerns the identification of allosteric ligand-binding sites (I’m not entirely convinced yet that this is important -- because I can’t think of any situations where this might be important -- convince myself or just drop this idea), through which drugs can modulate the effects of ligands that bind at the primary site.  More generally, an important limitation of traditional virtual drug screening is that it must start with a well-defined binding site (look into limitations of current drug discovery processes … may give some clue for constructing an argument for how MD is helping …), despite promising recent developments.\cite{Hetenyi, C.; van der Spoel, D. FEBS Lett. 2006, 580, 1447. (11) Davis,I.W.;Raha,K.;Head,M.S.;Baker,D.ProteinSci.2009, 18, 1998. }

\section{Significance for disordered binding}
% [Conclusions - A general interest in the specific work that I have done here ] 
Aside from contributing to the design of future AD therapeutics, our results have also contributed to the understanding of small molecule binding to disordered peptides (IDPs) in general. This is important because disordered binding are ubiquitous in biology, and disordered peptides and proteins and involved in many diseases such as X, Y, Z.  

Understanding how small molecules may interact with intrinsically disordered proteins goes beyond amyloid-related disorders. As IDPs are involved in many signalling pathways, they are also viable drug targets for many diseases.   Hence, targeting these disordered peptides using small molecules may be a possible therapeutic approach. For example, c-Myc is frequently involved in many cancers, \textbf{FILL IN THE GAP ABOUT HOW IT WORKS} and thus disruption of the c-Myc–-Max interaction is a possible anticancer strategy.\cite{Iakoucheva:2002uv,Metallo:2010p6822,Cuchillo:2012bm}

The work presented in chapters in this thesis sheds light on the mechanism of binding disordered peptides, and represents a step forward in understanding how small molecules may prevent protein - protein interactions, which involves identifying the binding interfaces, and understanding how to target these interfaces using small molecules.REFs.

\section{Osmolyte effects, denaturation, and macromolecular crowding}
% Note this is another hairy field … and you might want to stay the hell away from it, despite the fact that your thesis is loosely connected to this field
Another field of interest where weak interactions dominate is cosolvent effects on peptide folding. Simulations are a good use for probing that.  For example, MD simulations have been useful in gaining insight into protein denaturation mechanisms by urea or guanidinium, and the activity of osmolytes. Many of these are still open questions.\cite{http://pubs.acs.org/doi/abs/10.1021/jp200625k -- crowding and protein association.} Where am I going with this?

% These ideas below are  lesser developed ideas … consider cutting or bulk up …
% \section{other areas that are related to my work}
% [Ab-GAG membrane binding] This branches off the fact that I looked at sugar binding with peptides - amyloids when deposited may interact with glycosaminoglycans (part of the extracellular matrix) exposed at cellular surfaces. So what about it? Is my work helping to understand how amyloids are interacting with GAGs? or what role GAGs might play in accelerating amyloid formation?


% \subsection{Relationship to Polypharmacology -- where one drug binds to different targets?}
% Not sure how my thesis relates to this.

% I think this section is too crazy - eliminate.
%\section{In the future - perspectives on computer simulations}
% TODO: Find out why are there so few new drugs being discovered nowadays?
% Expand simulations into the macroscopic level 

% TODO: A good thought experiment - If I had the perfect simulation system? Predictive force field, simulate milliseconds in days.  How could I use this simulation system? Some effective use of the data? 

%Software has the ability to revolutionize drug discovery and take it from bench to personalized medicine. Molecular simulations can play a role in driving experimentation by helping to generate testable hypotheses.
%
%Virtual drug screening method using MD simulations as a component predicting drug toxicity.
%
%Here I'm speculating what the future of MD simulations might be ... A bit like science fiction with a touch of reality.
%
%Most Useful to “somehow” integrate experimental data with simulations results\cite{that nature paper discussing integrating MD and systems biology}
%- What are some success stories?
%what’s the real problem with drugs?
%Can we do without drugs? What are the therapies available?
%Small molecule
%peptide
%antibodies
%gene therapy (viral)
%\cite{Hansen:2012hh}

\section{Future directions}
% Better methods? Longer simulation times? Better force fields to detect hydrophobic effect?
% Better understanding of the amyloid aggregation mechanism will lead to a better understanding of the inhibition mechanism.

These are significance points.  Tie them better Studying different aggregate forms is useful.  One major conclusion of this thesis is that doing comparative systematic simulations is needed and useful for understanding the molecular basis of amyloid inhibitors how work.
[sketchy and would require a fast compute cycle -- look into more literature to formulate this idea better] This thesis shows that MD may be useful in the rational design of amyloid inhibitors from a template inhibitor. Molecular information obtained from simulations can be used to help design new derivatives, and simulations can be run for these chemically-modified forms of derivatives. 

\subsection{Simulations for other peptides involved in amyloid disorders}
Hard disk space and cpu power can be easily secured.  It would be a minor point to get that done.  Since there are many amyloid-associated disorders. In the future, simulations can be performed with these peptides while using a similar methodology. 

\subsection{Effect of inositol in the presence of lipid membranes}
Effect of the drug in the presence of lipid membranes => How does it affect aggregation in the presence of membranes. Does it prevent peptides from binding to the membrane?

\subsection{Apply enhanced sampling methods to study amyloid inhibition}
Extend simulations. The goal is to be able to simulate the entire aggregating process.  But this is not feasible because the timescale of fibril formation can take up to days. But can simulate the early stages of amyloid formation.  Use of enhanced sampling methods for aggregates in the presence of the drug.  These methods can enhanced Replica-exchange.  These methods enhance sampling and can speed up beta-sheet formation. With longer simulation times, can reach beta-sheet formation for shorter peptides such as KLVFFAE, we can add inositol (or some other drug) to understand their activity in early amyloid formation.


\section{Limitations}
Length of the peptides

\textbf{Lack of experimental data.} Finally, it would important to combine simulations with experimental studies.  This is a deficiency of this study. Simulations can be a good tool when combined with experimental validation by using a variety of techniques SSNMR and other biophysical techniques used to probe amyloid systems. Several studies are beginning to do that to understand the molecular mechanism of small molecule inhibitors ECGC.  In recent years several studies have begun to do this with some successes.




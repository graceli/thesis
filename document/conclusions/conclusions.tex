\chapter{Perspectives, Conclusions and Future Directions}
% Example sentences found in teh conclusions and summary chapters of people's thesis 
% The work presented in this thesis has ...

% Because this work represents the first atomistic simulation, to our knowledge, demonstrating that polypeptide chains can form entangled polymer melt-like states, it contributes to an improved understanding of both elastin coacervation, and the more general phenomenon of protein aggregation

% Everyone's conclusions all had a significant amount of Future directions. (~3 pages worth)

\section{Significance for drug development for AD}
%[Thesis significance - part of the summary of what I make of my thesis ] 
This thesis demonstrates the applicability of MD simulations in providing insight into how drugs might bind to amyloidogenic species and intrinsically disordered peptides.  The results in this thesis can be used to map out a pharmacophore for developing a drug for the treatment of Alzheimer's Disease, opening the way to computer-aided design of improved diagnostics and therapeutics. A pharmacophore is an abstract description of molecular features which are necessary for molecular recognition of a ligand by a biological macromolecule - If I have to define this here, then it should have been in the introduction. \textbf{Definition is taken from the wiki.}

% Speculate on the future of drug development for AD in regards to the significance of thesis wrt curing AD. Will blocking aggregation work for AD? - May be this should in the introduction instead.
% (WRITE SOME CONCLUSIONS HERE … RELATING TO HOW A SINGLE SMALL MOLECULE BINDS SPECIFIC MORPHOLOGICAL STRUCTURES -- clearly a key result of my works demonstrate that there is sequence specificity) 

Because AD may be caused by a multitude of pathological changes in the brain, it is likely that a cocktail of compounds, each targeting a different disease pathway, may be required for treating AD. % (Adapted from pharmacophore for AD 2011)

% \section{MD simulations and Rational drug design}
\section{Contribution to sugar-protein binding}
% MD simulation as a tool for probing weak interactions
This thesis has demonstrated that MD simulations, and the current force fields can effectively probe weak and transient molecular interactions, which are often not readily detectable using experimental techniques. One prominent example where weak interactions are prominent in protein-ligand binding is protein-carbohydrate interactions.\cite{weak binding review paper}

Understanding protein-sugar interactions is an important endeavor because of antibody binding to proteins.  Viruses often express carbohydrates on their coats.  Inhibiting bacterial action involves knowing how polysaccharrides are expressed, which involves binding proteins.  The results of this thesis presents methods that may be useful for developing antibiotics. \textbf{Rewrite this part to reflect and tie into how my work, methods, and how they are related to solving these problems} 

% The role of simulations in drug discovery to effectively predict binding modes and binding sites

[Thesis significance; more perspectives] Traditionally computation studies probing protein-ligand binding is often carried out with the knowledge of a putative binding site (often determined by X-ray crystallography) and the mechanism is examined employing complicated methods using the ligands thought to be able to bind in this specific pocket, while ignoring all other possibilities.  However, with enough computing power to extend simulations to a long time, longer time scale MD simulations can be accurate in discovering new binding sites (even sites that are XXX high in affinity) without having to make any prior assumptions about the binding site. Our studies are among those which demonstrate the utility of MD to probe for binding sites that are not possible to be obtained via experimental structural determination methods.  A recent MD study have demonstrated the capability of MD in binding site prediction for a folded protein...\cite{Shan:2011bo}

% Shan, 2011 (the DE shaw letter) An emerging challenge in drug discovery concerns the identification of allosteric ligand-binding sites (I’m not entirely convinced yet that this is important -- because I can’t think of any situations where this might be important -- convince myself or just drop this idea), through which drugs can modulate the effects of ligands that bind at the primary site.  More generally, an important limitation of traditional virtual drug screening is that it must start with a well-defined binding site (look into limitations of current drug discovery processes … may give some clue for constructing an argument for how MD is helping …), despite promising recent developments.\cite{Hetenyi, C.; van der Spoel, D. FEBS Lett. 2006, 580, 1447. (11) Davis,I.W.;Raha,K.;Head,M.S.;Baker,D.ProteinSci.2009, 18, 1998. }

\section{Significance for disordered binding}
% [Conclusions - A general interest in the specific work that I have done here ] 
Aside from contributing to the design of future AD therapeutics, our results have also contributed to the understanding of small molecule binding to disordered peptides (IDPs) in general. This is important because disordered binding are ubiquitous in biology, and disordered peptides and proteins and involved in many diseases such as X, Y, Z.  

Understanding how small molecules may interact with intrinsically disordered proteins goes beyond amyloid-related disorders. As IDPs are involved in many signalling pathways, they are also viable drug targets for many diseases.   Hence, targeting these disordered peptides using small molecules may be a possible therapeutic approach. For example, c-Myc is frequently involved in many cancers, \textbf{FILL IN THE GAP ABOUT HOW IT WORKS} and thus disruption of the c-Myc–-Max interaction is a possible anticancer strategy.\cite{Iakoucheva:2002uv,Metallo:2010p6822,Cuchillo:2012bm}

The work presented in chapters in this thesis sheds light on the mechanism of binding disordered peptides, and represents a step forward in understanding how small molecules may prevent protein - protein interactions, which involves identifying the binding interfaces, and understanding how to target these interfaces using small molecules.REFs.

\section{Osmolyte effects, denaturation, and macromolecular crowding}
% Note this is another hairy field … and you might want to stay the hell away from it, despite the fact that your thesis is loosely connected to this field

Another field of interest where weak interactions dominates is understanding cosolvent effects on peptide folding. Simulations are a good use for probing that.  For example, MD simulations have been useful in gaining insight into protein denaturation mechanisms by urea or guanidinium, and the activity of osmolytes. Many of these are still open questions.\cite{http://pubs.acs.org/doi/abs/10.1021/jp200625k -- crowding and protein association.} Where am I going with this?

% These ideas below are  lesser developed ideas … consider cutting or bulk up …
% \section{other areas that are related to my work}
% [Ab-GAG membrane binding] This branches off the fact that I looked at sugar binding with peptides - amyloids when deposited may interact with glycosaminoglycans (part of the extracellular matrix) exposed at cellular surfaces. So what about it? Is my work helping to understand how amyloids are interacting with GAGs? or what role GAGs might play in accelerating amyloid formation?


% \subsection{Relationship to Polypharmacology -- where one drug binds to different targets?}
% Not sure my thesis relates to this.


\section{In the future - perspectives on computer simulations}
% TODO: Find out why are there so few new drugs being discovered nowadays?
% Expand simulations into the macroscopic level 

% TODO: A good thought experiment - If I had the perfect simulation system? Predictive force field, simulate milliseconds in days.  How could I use this simulation system? Some effective use of the data? 


Software has the ability to revolutionize drug discovery and take it from bench to personalized medicine.  

Role of simulations in rational drug design: can drive experiments and generate testable hypotheses.

Virtual drug screening method using MD simulations as a component predicting drug toxicity.

Here I'm speculating what the future of MD simulations might be ... A bit like science fiction with a touch of reality.

Most Useful to “somehow” integrate experimental data with simulations results\cite{that nature paper discussing integrating MD and systems biology}
- What are some success stories?
what’s the real problem with drugs?
Can we do without drugs? What are the therapies available?
Small molecule
peptide
antibodies
gene therapy (viral)
\cite{Hansen:2012hh}


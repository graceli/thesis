% Tense
% What's unclear right now, what I should add
% What do I think is the most important thing to come out of this work ??

% This chapter can just be 3 - 4 pages.  It doesn't have to be very long.

\chapter{Conclusions and Future Directions}
\section{Conclusions}
% Go from narrow to broad: Alzheimer's Disease (AD) is a devastating neurodegenerative disease that is the most common cause of dementia in persons of age 65 or older. Currently there is no cure or method of treatment that targets the underlying disease.  As the population of the world is living longer, if left untreated, AD will become an epidemic that poses a tremendous medical burden to society.

% I examined the interaction of inositol stereoisomers with various aggregate species in the amyloid aggregation pathway.
The primary objective of my research was to elucidate the molecular basis of the structure-activity relationship of inositol, a putative therapeutic for the treatment of Alzheimer's Disease (AD). To the best of my knowledge, this thesis represents the most complete molecular dynamics simulation study of a small-molecule amyloid inhibitor (\textit{scyllo}-inositol) to date. My work provides insight into designing novel and higher-efficacy derivatives of putative drugs which may prevent the onset and progression of AD and amyloid-related disorders.  Furthermore, these results are relevant for understanding the binding mechanism of small molecules to intrinsically disordered proteins in general.

Throughout this thesis, a systematic comparative approach was utilized as a central methodological framework in each of my studies. Simulations of amyloidogenic peptides and their aggregates of increasing sequence and composition complexity were successively carried out in the presence and absence of \emph{scyllo}-inositol, and its inactive isomer, \emph{chiro}-inositol (see Chapter 1.XXX for details on rationale and study design). Importantly, this approach was instrumental in the investigation of the binding mechanism of inositol. Utilizing such a comparative methodology will likely be important for understanding the molecular basis of the mechanism of action of other small-molecule amyloid inhibitors. My work suggests broader strategies for probing weak protein-ligand binding in general: the applicability of the methods utilized in my inositol studies (Chapters 3-5) were applicable for examining protein-carbohydrate binding (see Chapter 6).
% significance -> I felt that the systematic comparative study approach was important to the overall success of the thesis because ...; this type of studies is important because ...
% I was able elucidate the binding mechanism of \textit{scyllo}-inositol (see chapters 3-5). 
 %This approach was crucial for deriving the hypothesis that the protofibril is the most likely binding partner of inositol.

Beginning in chapter 2, I performed systematic simulations of simple amyloidogenic peptide models with \textit{scyllo}- and \textit{chiro}-inositol to examine the role of backbone binding on amyloid inhibition. My results indicated that although inositol predominately interacts with the peptide backbone, the binding affinity is low and remains in the millimolar range. Moreover, backbone binding was independent of stereochemistry and did not appear to be sufficient to impede peptide dimerization. Taken together, my results in this chapter suggest that amyloid inhibition by inositol cannot be accounted for by generic binding to the peptidic backbone alone. In this study, it was hypothesized that amyloid inhibition by inositol is likely to involve sequence-specific interactions with amino-acid side chains as well as binding to specific aggregate morphologies.
% Accordingly, although the formation of intermolecular hydrogen bonds is the predominant interaction in protein aggregates composed of \gafour, amyloidogenic peptides involved in amyloid diseases are often more hydrophobic and in general, self-aggregation is driven largely by the hydrophobic effect.\cite{Chiti:2006p20}

To investigate the role of sequence-specific interactions between inositol and aggregates of pathogenic peptides, in chapter 4, I examined the binding of inositol stereoisomers, successively, to monomers, disordered oligomers, and $\beta$-sheet aggregates of A$\beta$(16-22). A key result of this study was that the $K_{eq}$ of inositol ($\sim$0.2 - 0.5 mM) for the $\beta$-oligomer is commensurate with the concentration at which inhibition of amyloid formation by A$\beta$42 is observed \emph{in vitro}. Both \emph{scyllo}- and \emph{chiro}-inositol exhibited similar binding affinities with all peptide states considered. However,  \textit{scyllo}-inositol was found to possess a stereospecific face-to-face stacking stacking mode with the Phe side chains and a higher propensity for hydrogen bonding, which together suggests a molecular basis for measured differences in activity.  Cooperative binding modes of inositol at grooves on the surface of the $\beta$-oligomer of A$\beta$(16-22) suggest a possible mechanism of fibril inhibition whereby inositol prevents the lateral association or stacking of protofibrillar $\beta$-sheet oligomers. Because inositol was found to adopt carbohydrate-like binding modes at the fibril core of A$\beta$ amyloid, carbohydrate-based small-molecule derivatives may be promising for the development of novel therapeutics for AD.

The above results (presented in chapters 2 and 3) have led me to hypothesize that \textit{scyllo}-inositol is likely to act on the protofibrillar form of A$\beta$42.  In chapter 4, I examined the binding of \textit{scyllo}-inositol to the protofibrillar form of A$\beta$42. chiro-inositol, glucose and glycerol, which were shown not to inhibit \abeta42\ amyloid formation, were used as negative controls in my study. There was no difference in the protofibril conformation in the presence of cosolutes. However, differences in stereochemistry between the molecules led to differential binding modes and propensities. Notably, we found that \textit{scyllo}-inositol displays the highest binding specificity to the region of the surface containing the sequence Leu-Val-Phe-Phe-Ala-Glu, the central hydrophobic core of A$\beta$ fibrils.  On the basis of the ability of scyllo-inositol to bind by forming hydrogen bonds and its binding specificity for the CHC of A$\beta$42 suggest a mechanism of inhibition whereby \textit{scyllo}-inositol prevents the growth of protofibril oligomers into mature amyloid fibrils by disrupting the lateral stacking of these oligomers.

Taken together, the results in Chapters 3 to 5 have demonstrated the applicability of MD simulations in providing insight into how drugs might bind to amyloidogenic species and intrinsically disordered peptides. 

More generally, my work has demonstrated that MD simulations and currently available force fields can be effectively applied to probe weak and transient molecular interactions that are difficult to study using experimental techniques. Understanding the molecular basis of weak interactions are particularly important for protein-carbohydrate binding.\cite{Canchi:2011cg,Fadda:2010p5889}
% \textbf{Elaborate on weak binding} 
In Chapter 6, we performed MD simulations of PgaB, a key protein in the biofilm formation pathway, in the presence of monosacharides of GlcNAc and glucosamine (monomeric components of the polymeric substrate). A binding surface and binding mode for the polymeric substrate (PNAG) of PgaB were predicted from my simulations. Notably, results were consistent with electron densities derived from X-ray crystallography studies. This work indicates that MD simulations combined with the methodology employed this thesis is applicable for examining carbohydrate-protein interactions.
% \textbf{Biological statement of what this work shows!} 
% Together, the morphology of the overall binding density of both GlcNAc and \glucosamine\ paints a molecular picture of the putative binding mechanism and export of PNAG. 


\section{Future directions}
% What are some things that are limiting ... there are lots
% focus on how the technique is limited getting to the desired goal -- a future direction is to combine experimental work with MD simulations.
% Inositol does not necessarily act on the shorter peptide models and their aggregates.
% \textbf{Lack of experimental data.}

% Regis: Note that I don't have to be very detailed here because I already have an application chapter.  Can just describe ideas in high-level.  Examiners would be interested in hearing what the candidate thinks in terms of what would be natural extensions of the work / project.

\subsection{Preventing the growth of amyloid formation by Seeding simulations}
Our work here led to the hypothesis that binding to the surface of protofibrils may lead to inhibition. We can test this hypothesis by directly performing simulations in the presence of both free small molecules and free peptides. Currently running seeding simulations of a protofibril with bound inositol and dispersed monomers in solution. Here, the idea is to test whether the presence of bound inositol on the protofibril alters the binding equilibrium of monomers to the protofibril.
% Preliminary analysis shows that monomers can form disordered oligomers, which then goes on to bind the protofibril, in the presence of inositol.
Analyses such as computing the spatial distribution of the bound monomers around the protofibril will suggest how peptides partition around a protofibril and whether the presence of inositol alters the binding equilibrium of monomers to pre-existing $\beta$-sheets. Moreover, it will be useful to compute the secondary structure of the peptides bound to the protofibril.  This will indicate whether the dispersed peptides are able to form $\beta$-strands after binding to the protofibril.

\subsection{Effect of inositol in the presence of lipid membranes}
% Effect of the drug in the presence of lipid membranes => Does it prevent peptides from binding to the membrane?
As discussed in the introduction (chapter 1), it is currently hypothesized that amyloid may disrupt the integrity of cellular membranes by interacting with them. An interesting question is whether amyloid inhibitors such as inositol affect the aggregation of A$\beta$ peptides in the presence of lipid membranes. Inositol was shown to neutralize the toxicity of A$\beta$ oligomers.\cite{McLaurin:2000bq}  Hence, it is possible that inositol may prevent the partitioning of KLVFFAE monomers and aggregates onto surfaces or alternatively prevent the formation of $\beta$-sheets at the surfaces of membranes.\cite{references}

Using a similar systematic approach that was carried out in this thesis, the effect of inositol on the binding of KLVFFAE peptide aggregates in the presence of (1) the membrane-mimetic, octane slab, and (2) lipid bilayers of differing composition. Previous simulation study in our lab have shown that $\beta$-sheet formation of amyloidogenic peptides is catalyzed by the presence of a hydrophobic surface.\cite{Nikolic:2010go} Furthermore, simulations of lipid bilayers and amyloid-forming peptide fragments have been performed by Dr. Loan Huynh to ascertain how the bilayer affects peptide structure and aggregation and, concurrently, how the peptides perturb the lipid bilayer. With the addition of inositol, molecular simulations of these systems will provide insight into the  molecular basis of inhibition of amyloid toxicity. These studies can be combined with enhanced sampling methods,\cite{TVREX, STDR} which are computationally expensive but will be necessary to obtain statistically converged properties of interest.

\subsection{Application of enhanced sampling methods to probe amyloid inhibition by small molecule}
% Regis: Why do you say that you need enhanced sampling. Justify enhanced sampling => it's mainly used to observed beta-sheet formation. Increase the number of monomer conformations.
% Formation of beta-sheets -- not able to reach the timescales on which I observe beta-sheet formation.
% The one key thing that I don't have information on is the timescales required to observe the direct effect of inos on beta sheet formation
A limitation in my study is that the monomeric peptides examined in this study were either model peptides or shorter fragments of longer amyloidogenic peptides. Small molecule inhibitors may have differing effects on the peptide conformation and aggregation of longer amyloidogenic peptides such as the full-length A$\beta$40 or A$\beta$42 peptides. To address this possibility, it will be useful to perform simulations to examine (1) the properties of monomeric peptides in the presence of inositol, and (2) the binding mechanism of \textit{chiro}- and \textit{scyllo}-inositol with the full-length A$\beta$. However with longer peptides, additional methods such as enhanced sampling may be necessary to obtain statistically converged on longer peptides in order to draw conclusions.

Extend simulations. The goal is to be able to simulate the entire aggregating process.  But this is not feasible because the timescale of fibril formation can take up to days. But can simulate the early stages of amyloid formation.  Use of enhanced sampling methods for aggregates in the presence of the drug.  These methods can enhanced Replica-exchange.  These methods enhance sampling and can speed up beta-sheet formation. With longer simulation times, can reach beta-sheet formation for shorter peptides such as KLVFFAE, I can add inositol (or some other drug) to understand their activity in early amyloid formation.

\subsection{Future Experimental Work}
My work involving molecular dynamics simulations of amyloid-inhibitor interactions provide insight at the atomistic level which is not feasible to obtain using experimental approaches, and has led to testable hypothesis relating to amyloid inhibition. Ongoing collaborative experimental work is being carried out in the laboratory of Dr. Simon Sharpe to test the effect of binding to Phe on the kinetics of amyloid formation. Furthermore, targeting the CHC to arrest amyloid formation can be tested as well. % -- Did Pat mention any of this stuff in his thesis?} 
New ideas for synthesizing novel derivatives based on inositol. My work indicated that inositol molecules likely cluster together when bound, which suggests that derivatives of inositol where multiple monomers of \textit{scyllo}-inositol are linked together may result in a more efficacious inhibitor of amyloid formation. Ultimately, it will be beneficial to cross-validate the results obtained in this thesis experimentally by using solid-state NMR and other biophysical techniques. In the future, synergistic studies combining simulations and experimental characterization will accelerate the progress towards an understanding of the mechanism of amyloid inhibition.   
% What about in the general direction of protein - carbohydrate binding? Short blurb here?  Maybe ask Regis for advice here?
% What did Dustin say here again? <Add blurb from Dustin>

% \subsection{Comparative simulations of Abeta42 and Abeta40 protofibrils}
% [ Controversial topic - I want to leave this out but this would also seed a question during the defense]
% We have uncovered that different stereochemistry of compounds binds have differing specificity of binding to surfaces of beta-sheet fibrils, and their binding modes are sensitive to the surface properties.  A natural extension of this work would be to perform a systematic comparative study of scyllo- and chiro-inositol binding with Abeta40.  Binding modes with Abeta40 may inform the differences in activity with respect to Abeta42 and Abeta40.

%\subsection{Simulations for other peptides involved in amyloid disorders}
%Hard disk space and cpu power can be easily secured.  It would be a minor point to get that done.  Since there are many amyloid-associated disorders. In the future, simulations can be performed with these peptides while using a similar methodology. 



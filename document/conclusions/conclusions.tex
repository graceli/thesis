% Tense
% What's unclear right now, what I should add
% What do I think is the most important thing to come out of this work ??

% This chapter can just be 3 - 4 pages.  It doesn't have to be very long.

\chapter{Conclusions and Future Directions}
\section{Conclusions and perspectives}
Alzheimer's Disease (AD) is a devastating neurodegenerative disease that is the most common cause of dementia in persons of age 65 or older. Currently there is no cure or method of treatment that targets the underlying disease.  As the population of the world is living longer, if left untreated, AD will become an epidemic that poses a tremendous medical burden to society. 

The primary objective of my research was to elucidate the molecular basis of the structure-activity relationship of inositol, a small-molecule putative therapeutic for the treatment of Alzheimer's disease.  To characterize the molecular mechanism of action of inositol, I examined the interaction of inositol with various aggregate species in the amyloid aggregation pathway. Systematic comparative simulation studies of amyloidogenic peptides and their aggregates of increasing sequence and composition complexity were carried out in the presence and absence of \emph{scyllo}-inositol, and its inactive isomer, \emph{chiro}-inositol.

To the best of my knowledge, the work presented in this thesis represents the most complete molecular dynamics simulation study of a small-molecule amyloid inhibitor to date. My work provides insight into designing novel, higher-efficacy derivatives of putative drugs which may prevent the onset and progression of Alzheimer's disease.  Moreover, these results are relevant for understanding the binding mechanism of small molecules to intrinsically disordered proteins. Finally, the applicability of our methodology for examining binding mechanisms of weak protein-ligand interactions such as protein-carbohydrate binding is demonstrated. In summary, the work presented in this thesis contributed to two peer-reviewed articles and two manuscripts in preparation (see Chapters 3 to 6).

Beginning in chapter 2, I performed systematic simulations of simple amyloidogenic peptide models with \textit{scyllo}- and \textit{chiro}-inositol to examine the role of backbone binding on amyloid inhibition. My results indicated that although inositol predominately interacts with the peptide backbone, the binding affinity is low and remains in the millimolar range. Moreover, backbone binding was independent of stereochemistry and did not appear to be sufficient to impede peptide dimerization. Taken together, my results in this chapter suggest that amyloid inhibition by inositol cannot be accounted for by generic binding to the peptidic backbone alone. In this study, it was hypothesized that amyloid inhibition by inositol is likely to involve sequence-specific interactions with amino-acid side chains as well as binding to specific aggregate morphologies.
% Accordingly, although the formation of intermolecular hydrogen bonds is the predominant interaction in protein aggregates composed of \gafour, amyloidogenic peptides involved in amyloid diseases are often more hydrophobic and in general, self-aggregation is driven largely by the hydrophobic effect.\cite{Chiti:2006p20}

To investigate the role of sequence-specific interactions between inositol and aggregates of pathogenic peptides, in chapter 4, I examined the binding of inositol stereoisomers, successively, to monomers, disordered oligomers, and $\beta$-sheet aggregates of A$\beta$(16-22). A key result of this study was that the $K_{eq}$ of inositol ($\sim$0.2 - 0.5 mM) for the $\beta$-oligomer is commensurate with the concentration at which inhibition of amyloid formation by A$\beta$42 is observed \emph{in vitro}. Both \emph{scyllo}- and \emph{chiro}-inositol exhibited similar binding affinities with all peptide states considered. However,  \textit{scyllo}-inositol was found to possess a stereospecific face-to-face stacking stacking mode with the Phe side chains and a higher propensity for hydrogen bonding, which together suggests a molecular basis for measured differences in activity.  Cooperative binding modes of inositol at grooves on the surface of the $\beta$-oligomer of A$\beta$(16-22) suggest a possible mechanism of fibril inhibition whereby inositol prevents the lateral association or stacking of protofibrillar $\beta$-sheet oligomers. Because inositol was found to adopt carbohydrate-like binding modes at the fibril core of A$\beta$ amyloid, carbohydrate-based small-molecule derivatives may be promising for the development of novel therapeutics for AD.

The above results (presented in chapters 2 and 3) have led me to hypothesize that \textit{scyllo}-inositol is likely to act on the protofibrillar form of A$\beta$42.  In chapter 4, I examined the binding of \textit{scyllo}-inositol to the protofibrillar form of A$\beta$42. In this study, chiro-inositol along with gluce, which does not inhibit \abeta42\ amyloid formation, were used as negative controls in our studies.


 I found that there were no differences in the fibril conformations with and without inositol in either low or high molar ratio.
% Does glucose bind more or less? If it does bind less, then it tells us that I might be onto something with scyllo-inositol even though the structure is subtly different.  
% Not sure what I meant here
Glucose does not necessarily bind ``less", but it does not bind on the right face ie. the KLVFFAE face \textit{scyllo}-inositol appears to preferentially bind the KLVFFAE face, more than glucose and \textit{chiro}-inositol.  

Furthermore, \textit{scyllo}-inositol, due to its stereochemistry, did not bind in a hydrated tunnel formed in the protofibrillar aggregate, where as both \textit{chiro}-inositol and glucose did bind in this tunnel.

These differences between active and inactive inhibitors of A$\beta$42 suggest a mechanism of inhibition.

Taken together, the results in chapters 3-5 have demonstrated the applicability of MD simulations in providing insight into how drugs might bind to amyloidogenic species and intrinsically disordered peptides.  Furthermore, our simulations helped define the pharmacophore for a compound with the potential to inhibition amyloid formation by A$\beta$. More generally, my work has demonstrated that MD simulations and currently available force fields can be effectively applied to probe weak and transient molecular interactions, which are often difficult to probe using experimental techniques.  The ability to probe weak interactions is important for understanding protein-carbohydrate interactions.\cite{weak binding review paper}

% The role of simulations in drug discovery to effectively predict binding modes and binding sites
% \textbf{[Thesis significance; more perspectives] Traditionally computation studies probing protein-ligand binding is often carried out with the knowledge of a putative binding site (often determined by X-ray crystallography) and the mechanism is examined employing sophisticated methods using the ligands thought to be able to bind in this specific pocket, while ignoring all other possibilities.  However, with the computing power to extend simulations to a longer time scale, MD simulations may be accurate in discovering new binding sites (even sites that are XXX high in affinity) without prior assumptions about the binding site. Our studies are among those which demonstrate the utility of MD to probe for binding sites that are difficult to obtain via experimental structural determination methods.  A recent MD study have demonstrated the capability of MD in binding site prediction for a folded protein.\cite{Shan:2011bo}}

In chapter 5, I demonstrated the generality of the methodology by using similar approaches to examine carbohydrate binding to the protein PgaB. Specifically, in this study, I demonstrated that by using a high concentration of monosacharides GlcNAc and glucosamine, MD simulations can be used successfully to map out a binding surface relevant for understanding , a key protein in the biofilm formation pathway. The predicted binding sites and binding modes were consistent with electron densities derived from X-ray crystallography studies.  The work presented in this chapter suggests that the methodology developed this thesis is generally applicable for examining carbohydrate-protein interactions.

\section{Limitations of this study}
% What are some things that are limiting ... there are lots
% focus on how the technique is limited getting to the desired goal -- a future direction is to combine experimental work with MD simulations.  Nevertheless 
% Inositol does not necessarily act on the shorter peptide models and their aggregates.
% \textbf{Lack of experimental data.} 

One limitation of this study is that the monomeric peptides examined in this study were either model peptides or shorter fragments of longer amyloidogenic peptides. The effect of small molecule inhibitors on the peptide conformation and aggregation may differ for longer amyloidogenic peptides such the full-length A$\beta$40 or A$\beta$42. To address this possibility, it will be useful to perform simulations to comparatively examine the interaction of \textit{chiro}- and \textit{scyllo}-inositol with monomers of the A$\beta$ peptide.
\textbf{Our work presently has led to new testable hypothesis regarding amyloid inhibition. Ongoing collaborative experimental work is being carried out in the laboratory of Dr. Simon Sharpe to test the effect of binding F on amyloid formation. -- Did Pat mention any of this stuff in his thesis?} Also, studies suggests that inositol can be linked together inositol, which suggests novel ways to synthesize derivatives of inositol (Mark).

Molecular dynamics simulations of amyloid-inhibitor interactions provide insight at the atomistic level which is not feasible to obtain using experimental approaches.  Ultimately, it will be beneficial to cross-validate the results obtained in this thesis experimentally by using solid-state NMR and other biophysical techniques. In the future, synergistic studies combining simulations and experimental characterization will accelerate the progress towards an understanding of the mechanism of amyloid inhibition.   
%Several studies are beginning to do that to understand the molecular mechanism of small molecule inhibitors ECGC.  In recent years several studies have begun to do this with some successes.

% Formation of beta-sheets -- not able to reach the timescales on which I observe beta-sheet formation.
% The one key thing that I don't have information on is the timescales required to observe the direct effect of inos on beta sheet formation

% significance -> I felt that the systematic comparative study approach was important to the overall success of the thesis because ...; this type of studies is important because ...

\section{Future directions}
% Regis: Note that I don't have to be very detailed here because I already have an application chapter.  Can just describe ideas in high-level.  Examiners would be interested in hearing what the candidate thinks in terms of what would be natural extensions of the work / project.

% Regis: Why do you say that you need enhanced sampling. Justify enhanced sampling => it's mainly used to observed beta-sheet formation. Increase the number of monomer conformations.
\subsection{Application of enhanced sampling methods to probe amyloid inhibition by small molecule}
Extend simulations. The goal is to be able to simulate the entire aggregating process.  But this is not feasible because the timescale of fibril formation can take up to days. But can simulate the early stages of amyloid formation.  Use of enhanced sampling methods for aggregates in the presence of the drug.  These methods can enhanced Replica-exchange.  These methods enhance sampling and can speed up beta-sheet formation. With longer simulation times, can reach beta-sheet formation for shorter peptides such as KLVFFAE, I can add inositol (or some other drug) to understand their activity in early amyloid formation.

\subsection{Preventing the growth of amyloid formation by Seeding simulations}
Currently running seeding simulations of a protofibril with bound inositol and dispersed monomers in solution. Here, the idea is to test whether the presence of bound inositol on the protofibril alters the binding equilibrium of monomers to the protofibril. Preliminary analysis shows that monomers can form disordered oligomers, which then goes on to bind the protofibril, in the presence of inositol.  Further analyses, such as computing the spatial distribution of the bound monomers around the protofibril will suggest how peptides partition around a protofibril and whether the presence of inositol alters the binding equilibrium of monomers to pre-existing $\beta$-sheets. Moreover, it will be useful to compute the secondary structure of the peptides bound to the protofibril.  This will indicate whether the dispersed peptides are able to form $\beta$-strands after binding to the protofibril.

\subsection{Effect of inositol in the presence of lipid membranes}
% Effect of the drug in the presence of lipid membranes => Does it prevent peptides from binding to the membrane?
As discussed in the introduction (chapter 1), it is currently hypothesized that amyloid may disrupt the integrity of cellular membranes by interacting with them. An interesting question is whether amyloid inhibitors such as inositol affect the aggregation of A$\beta$ peptides in the presence of lipid membranes. Inositol was shown to neutralize the toxicity of A$\beta$ oligomers.\cite{McLaurin:2000bq}  Hence, it is possible that inositol may prevent the partitioning of KLVFFAE monomers and aggregates onto surfaces or alternatively prevent the formation of $\beta$-sheets at the surfaces of membranes.\cite{references}

Using a similar systematic approach that was carried out in this thesis, the effect of inositol on the binding of KLVFFAE peptide aggregates in the presence of (1) the membrane-mimetic, octane slab, and (2) lipid bilayers of differing composition. Previous simulation study in our lab have shown that $\beta$-sheet formation of amyloidogenic peptides is catalyzed by the presence of a hydrophobic surface.\cite{Nikolic:2010go} Furthermore, simulations of lipid bilayers and amyloid-forming peptide fragments have been performed by Dr. Loan Huynh to ascertain how the bilayer affects peptide structure and aggregation and, concurrently, how the peptides perturb the lipid bilayer. With the addition of inositol, molecular simulations of these systems will provide insight into the  molecular basis of inhibition of amyloid toxicity. These studies can be combined with enhanced sampling methods,\cite{TVREX, STDR} which are computationally expensive but will be necessary to obtain statistically converged properties of interest.

% What about in the general direction of protein - carbohydrate binding? Short blurb here?  Maybe ask Regis for advice here?
% What did Dustin say here again? <Add blurb from Dustin>

\subsection{Simulations for other peptides involved in amyloid disorders}
Hard disk space and cpu power can be easily secured.  It would be a minor point to get that done.  Since there are many amyloid-associated disorders. In the future, simulations can be performed with these peptides while using a similar methodology. 



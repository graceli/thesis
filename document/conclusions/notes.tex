% Notes

% Example sentences found in the conclusions and summary chapters of people's thesis 
% The work presented in this thesis has ...

% Sarah's thesis example:
% Because this work represents the first atomistic simulation, to our knowledge, demonstrating that polypeptide chains can form entangled polymer melt-like states, it contributes to an improved understanding of both elastin coacervation, and the more general phenomenon of protein aggregation

% Everyone's conclusions all had a significant amount of Future directions. (~3 pages worth)
% Strategy:  take the conclusion paragraphs of each chapter and then meld it into a conclusions chapter.

% Make a clear and concise statement of the original contribution to knowledge found in your thesis.

%What has this work led to?
%This has led to the understanding of mechanism of amyloid formation.
%An exploration of carbohydrate binding.

% One sentence summarizing AD and the medical challenges that it poses.  Summarize why it is difficult to design a drug for AD.
% Summary of the core technique that is used in this thesis

% By utilizing molecular dynamics simulations, a computational technique based on classical physics which allows us to simulate the motions of biomolecular systems at the atomistic-level, \textbf{I have systematically examined model amyloidogenic peptides and aggregates to more complex systems involving the full-length A$\beta$42 peptide. -- unclear I need to summarize the study design and rationale a bit better here} 


This is the last part of the conclusions, most of which are notes:

% Some of the things that i can do to alleviate these limitations.
\section{Significance for disordered binding}
% [Conclusions - A general interest in the specific work that I have done here ] 
Aside from contributing to the design of future AD therapeutics, our results have also contributed to the understanding of small molecule binding to disordered peptides (IDPs) in general. This is important because disordered binding are ubiquitous in biology, and disordered peptides and proteins and involved in many diseases such as X, Y, Z.  

Understanding how small molecules may interact with intrinsically disordered proteins goes beyond amyloid-related disorders. As IDPs are involved in many signalling pathways, they are also viable drug targets for many diseases.   Hence, targeting these disordered peptides using small molecules may be a possible therapeutic approach. For example, c-Myc is frequently involved in many cancers, \textbf{FILL IN THE GAP ABOUT HOW IT WORKS} and thus disruption of the c-Myc–-Max interaction is a possible anticancer strategy.\cite{Iakoucheva:2002uv,Metallo:2010p6822,Cuchillo:2012bm}

The work presented in chapters in this thesis sheds light on the mechanism of binding disordered peptides, and represents a step forward in understanding how small molecules may prevent protein - protein interactions, which involves identifying the binding interfaces, and understanding how to target these interfaces using small molecules.REFs.

\section{Osmolyte effects, denaturation, and macromolecular crowding}
% Note this is another hairy field … and you might want to stay the hell away from it, despite the fact that your thesis is loosely connected to this field
Another field of interest where weak interactions dominate is cosolvent effects on peptide folding. Simulations are a good use for probing that.  For example, MD simulations have been useful in gaining insight into protein denaturation mechanisms by urea or guanidinium, and the activity of osmolytes. Many of these are still open questions.\cite{http://pubs.acs.org/doi/abs/10.1021/jp200625k -- crowding and protein association.} Where am I going with this?

% Better methods? Longer simulation times? Better force fields to detect hydrophobic effect?
% Better understanding of the amyloid aggregation mechanism will lead to a better understanding of the inhibition mechanism.
These are significance points.  Tie them better Studying different aggregate forms is useful.  One major conclusion of this thesis is that doing comparative systematic simulations is needed and useful for understanding the molecular basis of amyloid inhibitors how work.
[sketchy and would require a fast compute cycle -- look into more literature to formulate this idea better] This thesis shows that MD may be useful in the rational design of amyloid inhibitors from a template inhibitor. Molecular information obtained from simulations can be used to help design new derivatives, and simulations can be run for these chemically-modified forms of derivatives.

% Not sure if this is polished enough to go into the conclusions chapter -- Understanding protein-sugar interactions is an important endeavor because of antibody binding to proteins.  Viruses often express carbohydrates on their coats.  Inhibiting bacterial action involves knowing how polysaccharrides are expressed, which involves binding proteins.  The results of this thesis presents methods that may be useful for developing antibiotics. \textbf{Rewrite this part to reflect and tie into how my work, methods, and how they are related to solving these problems} 

% Significance for drug development for AD
% Thesis significance - part of the summary of what I make of my thesis As outlined in this thesis, there are a multitude of challenges in understanding the mechanism of a drug for treating neurodegeneration.

% opening the way to computer-aided design of improved diagnostics and therapeutics. A pharmacophore is an abstract description of molecular features which are necessary for molecular recognition of a ligand by a biological macromolecule - If I have to define this here, then it should have been in the introduction. \textbf{Definition is taken from the wiki.}

% Speculate on the future of drug development for AD in regards to the significance of thesis wrt curing AD. Will blocking aggregation work for AD? - May be this should in the introduction instead.
% (WRITE SOME CONCLUSIONS HERE … RELATING TO HOW A SINGLE SMALL MOLECULE BINDS SPECIFIC MORPHOLOGICAL STRUCTURES -- clearly a key result of my works demonstrate that there is sequence specificity) 

% Not sure how I will incorporate this yet -- Because AD may be caused by a multitude of pathological changes in the brain, it is likely that a cocktail of compounds, each targeting a different disease pathway, may be required for treating AD. % (Adapted from pharmacophore for AD 2011)

% \section{MD simulations and Rational drug design}
% \section{Contribution to sugar-protein binding}
% MD simulation as a tool for probing weak interactions

% Shan, 2011 (the DE shaw letter) An emerging challenge in drug discovery concerns the identification of allosteric ligand-binding sites (I’m not entirely convinced yet that this is important -- because I can’t think of any situations where this might be important -- convince myself or just drop this idea), through which drugs can modulate the effects of ligands that bind at the primary site.  More generally, an important limitation of traditional virtual drug screening is that it must start with a well-defined binding site (look into limitations of current drug discovery processes … may give some clue for constructing an argument for how MD is helping …), despite promising recent developments.\cite{Hetenyi, C.; van der Spoel, D. FEBS Lett. 2006, 580, 1447. (11) Davis,I.W.;Raha,K.;Head,M.S.;Baker,D.ProteinSci.2009, 18, 1998. }

% These ideas below are  lesser developed ideas … consider cutting or bulk up …
% \section{other areas that are related to my work}
% [Ab-GAG membrane binding] This branches off the fact that I looked at sugar binding with peptides - amyloids when deposited may interact with glycosaminoglycans (part of the extracellular matrix) exposed at cellular surfaces. So what about it? Is my work helping to understand how amyloids are interacting with GAGs? or what role GAGs might play in accelerating amyloid formation?


% \subsection{Relationship to Polypharmacology -- where one drug binds to different targets?}
% Not sure how my thesis relates to this.

% I think this section is too crazy - eliminate.
%\section{In the future - perspectives on computer simulations}
% TODO: Find out why are there so few new drugs being discovered nowadays?
% Expand simulations into the macroscopic level 

% TODO: A good thought experiment - If I had the perfect simulation system? Predictive force field, simulate milliseconds in days.  How could I use this simulation system? Some effective use of the data? 

%Software has the ability to revolutionize drug discovery and take it from bench to personalized medicine. Molecular simulations can play a role in driving experimentation by helping to generate testable hypotheses.
%
%Virtual drug screening method using MD simulations as a component predicting drug toxicity.
%
%Here I'm speculating what the future of MD simulations might be ... A bit like science fiction with a touch of reality.
%
%Most Useful to “somehow” integrate experimental data with simulations results\cite{that nature paper discussing integrating MD and systems biology}
%- What are some success stories?
%what’s the real problem with drugs?
%Can I do without drugs? What are the therapies available?
%Small molecule
%peptide
%antibodies
%gene therapy (viral)
%\cite{Hansen:2012hh}

\chapter{Binding of inositol to monomers and aggregates of A$\beta$(16-22)}

% The contents of this section were adapted from an article published in the \emph{Journal of Physical Chemistry}.
% \\
% \\
% \emph{Reference}:
% Li, G., Rauscher, S., Baud, S., & Pomès, R. (2012). Binding of Inositol Stereoisomers To Model Amyloidogenic Peptides. Journal of Physical Chemistry B, 116(3), 1111–1119.
% \\
% \\
\emph{Contributions}:
Grace Li conducted the research and wrote the section. Régis Pomès provided editorial input and guidance.

\section{Summary}
	Alzheimer's disease (AD) is a severe neurodegenerative disease with no cure. Currently, one method of targeting the underlying disease is to prevent or reverse the amyloid formation of A$\beta$42, a key pathological hallmark of AD. Scyllo-inositol is a polyol that exhibits stereochemistry-dependent inhibition of the formation of A$\beta$ fibrils \emph{in vitro}. We present molecular dynamics simulations of the monomeric, disordered and protofibrillar states of A$\beta$(16-22), an amyloid-forming peptide fragment found in the $\beta$-sheet core of full-length A$\beta$, successively with and without \emph{scyllo}-inositol and its inactive stereoisomer \emph{chiro}-inositol. Inositol binds monomers and disordered aggregates of A$\beta$(16-22) with similar affinities, whereas binding to $\beta$-sheet containing oligomers ($\beta$-oligomers) yield affinities in the low millimolar range commensurate with \emph{\emph{in vitro}} inhibitory concentrations of inositol. Furthermore, inositol adopts carbohydrate-like binding modes, where stereochemistry modulates the nonpolar binding specificity of inositol to glutamate and phenylalanine side chains. Our results suggest that \emph{scyllo}-inositol inhibits amyloid formation by coating the surface of protofibrillar aggregates of A$\beta$ and disrupting their lateral stacking into fibrils.


\section{Introduction}

One in eight people over the age of 65 has Alzheimer's Disease (AD), a progressive neurodegenerative disease that currently has no cure.\cite{Citron:2010p214} The amyloid cascade hypothesis states that the extracellular neuronal deposition of A$\beta$ amyloid plaque plays a central role in the pathogensis of AD.\cite{Solomon:2010p177} A$\beta$ is a peptide proteolytically cleaved from the amyloid precursor protein (APP) and is produced as two common alloforms, A$\beta$40 or A$\beta$42, which are 40 and 42 residues in lengths, respectively. In the diseased state, A$\beta$42 levels are elevated, and the peptides deposit as extracellular A$\beta$ plaques.\cite{Haass:2007p226,Citron:1997p228}

A$\beta$40 and A$\beta$42 are intrinsically disordered peptides that self-aggregate \emph{in vitro} to form amyloid fibrils. Amyloid fibrils are protein aggregates with a characteristic cross-$\beta$ structure, which consists of in-register $\beta$-sheets with backbone hydrogen bonds running parallel to the long axis of the fibril.\cite{Petkova:2002p192} Moreover, smaller fragments of the full length A$\beta$ sequence are also found to form amyloid in vitro.\cite{Balbach:2000p49,Sawaya:2007p11} In particular, one of the shortest amyloid-forming peptides structurally characterized using solid-state NMR is KLVFFAE or A$\beta$(16-22).\cite{Balbach:2000p49} The residues LVFFA are believed to form the central hydrophobic core critical for the initiation of aggregation and fibril formation in the full length A$\beta$ peptide.\cite{Wood:1995p190} Furthermore, single-point mutations in this region greatly affect the aggregation propensity of A$\beta$: known familial mutations E22Q, E22K, and E22G, known as ``Dutch'', ``Italian'' and ``Arctic'' mutations, respectively, significantly accelerate fibril formation,\cite{Kim:2008ef} whereas the mutation F19T abolishes the formation of fibrils \emph{in vitro}.\cite{Esler:1996p288}

% check the different mutations, might be missing one
% TODO use more connectives to draw the links between the concepts introduced

Amyloid fibril formation follows a complex pathway: Prior to the appearance of fibrils in vitro, amyloidogenic monomers self-aggregate into a variety of pre-fibrillar intermediate morphologies. While the fibril is an important state implicated in AD, recent research has shown that soluble oligomers as small as dimers and tetramers play a role in neurotoxicity.\cite{Bernstein:2009p165} In recent years, drug development and research efforts have been directed towards the development of therapeutic agents to prevent the self-aggregation and amyloid formation of A$\beta$, a promising treatment approach to target the underlying disease.\cite{Masters:2006p183,Citron:2010p214,Dasilva:2010p25} As a result, many different types of \emph{in vitro} amyloid inhibitors have been discovered, including peptide molecules,\cite{EsterasChopo:2008p219,Sciarretta:2006p181,Chalifour:2003p161,Scrocchi:2002p178} immunotherapies,\cite{Janus:2000p198,Solomon:2010p177} polyphenolic molecules,\cite{Masuda:2009p205,Berhanu:2010p230,Ehrnhoefer:2008p8} and other small molecules.\cite{Hawkes:2009p189,Masuda:2009p205,Necula:2007p227,Nitz:2008p13} These approaches have been reviewed in detail elsewhere.\cite{Citron:2010p214,Dasilva:2010p25} 
% Despite experimental progress made in recent years, still lack a mechanistic understanding how these small molecules work. These approaches have been reviewed in detail elsewhere.\cite{Citron:2010p214,Dasilva:2010p25}

\emph{scyllo}-Inositol is a small-molecule A$\beta$ amyloid inhibitor developed for the treatment of AD.\cite{Dasilva:2010p25,Hawkes:2009p189,McLaurin:2000p64,Nitz:2008p13,Sun:2008p208} Inositol is a class of polyols, of which eight out of nine stereoisomers are commonly found in nature (Fig. 1). \emph{Myo}-inositol, the most common isomer, is found at high concentrations in the tissues of the human central nervous system (CNS).\cite{Fisher:2002p62} Like \emph{myo}-inositol, \emph{scyllo}-inositol is also present in the brain and can be passively and actively transported across the blood-brain barrier.\cite{Fenili:2007p182} Importantly, \emph{scyllo}-inositol was demonstrated to prevent and reverse AD-like symptoms in a transgenic mouse model of AD.\cite{McLaurin:2006p29} Because of the positive CNS bioavailability and favorable \emph{in vivo} toxicity profile of inositol, both of which are rare and essential properties of putative AD drug candidates, inositol-based therapies represent unique and promising approach for the treatment of AD. In 2007, \emph{scyllo}-inositol (ELN0005) was fast-tracked by the United States Food and Drug Administration into phase II of clinical trials in North America.

\emph{In vitro}, inositol displays stereochemistry-dependent inhibition of A$\beta$42 fibrils: \emph{myo}-, \emph{epi}- and \emph{scyllo}-inositol were shown to inhibit A$\beta$42 fibrillation at concentrations of 1 - 5 mM,\cite{McLaurin:2000p64} whereas \emph{chiro}-inositol is inactive below molar concentrations.\cite{Janus:2000p198} Moreover, upon incubation of monomeric A$\beta$42 with \emph{scyllo}-inositol, circular dichroism spectroscopy indicated the formation of $\beta$-sheet structure at an inositol:peptide molar ratio of 25:1.\cite{McLaurin:1998p176} Although inositol stereoisomers have been proposed to inhibit amyloid formation by directly interacting with either monomers or non-fibrillar aggregates to ``cap off'' fibril growth,\cite{Janus:2000p198} the molecular basis of the effect of \emph{scyllo}-inositol and its stereoisomers on A$\beta$ amyloid formation is currently not understood.

Thus far, experimental efforts to characterize the molecular structure of non-fibrillar oligomers are impeded because of their transient and disordered nature. In turn, the lack of information on the molecular structure of amyloid oligomers hampers experimental determination of the modes of action of inositol. Molecular dynamics (MD) simulations, by contrast, are well-suited for studies of disordered proteins and can provide atomic-level insight into the mechanism of peptide self-aggregation.\cite{Nikolic:2011p185,Rauscher:2006p43,Li:2012p853,Rauscher:2010p5682}

MD simulations were previously employed to examine the binding mechanism of other small molecules inhibitors such as polyphenols,\cite{Lemkul:2010p23,Wang:2010p204} non-steroidal anti-inflammatory drugs\cite{Raman:2009p47,Takeda:2010p34}, and the well-known amyloid dye thioflavin T\cite{Wu:2008ds,Wu:2011fd} to monomers\cite{Liu:2009p213}and/or fibrillar aggregates of A$\beta$. Because of the existence of multiple aggregation states, small molecule inhibitors may have multiple modes of action and can act by either binding to monomers\cite{Ehrnhoefer:2008fd} non-fibrillar or fibrillar oligomers\cite{Buell:2010p9457} in the fibrillation pathway. Furthermore, their inhibitory activity may be also affected by both concentrations of the ligand and ligand:peptide molar ratios. For example, small molecules (-)-epigallochatechin gallate (EGCG)\cite{Wang:2010p204} and ibuprofen\cite{LeVine:2005cv} have been shown to have activities modulated by ligand:peptide molar ratio.  However, thus far, few MD simulation studies have comparatively examined the effect of ligand concentration on different relevant aggregation states along the amyloid fibrillation pathway. 

% INTRO - rationale - develop
In our previous study on inositol,\cite{Li:2012p853} we systematically examined the role of backbone binding in amyloid inhibition by inositol stereoisomers, at low molar ratios, with model peptides alanine dipeptide and (GA)$_4$, a $\beta$-sheet forming peptide. Weak binding, with dissociation constants commensurate with those of osmolytes, were found for inositol with all peptides and aggregates considered, indicating that backbone binding alone is likely to be insufficient for amyloid inhibition. However, we have uncovered stereochemistry-dependent binding modes with nonpolar groups on surfaces of (GA)$_4$ fibril-like aggregates, which suggests that both aggregate morphology and sequence-specific interactions may play an important role in A$\beta$ amyloid inhibition by inositol.  

In this paper, we elucidate the role of sequence-specific interactions of inositol by examining its binding to A$\beta$(16-22), an amyloidogenic peptide that is part of the central hydrophobic core of fibrillar A$\beta$42, in three aggregation states, monomer, disordered oligomer and protofibrillar-like aggregates ($\beta$-oligomers).  Using a systematic approach, comparative MD simulation studies of each of the aforementioned states were successively carried out successively in the presence and absence of \emph{scyllo}- and its inactive stereoisomer, \emph{chiro}-inositol.  Moreover, we examine the differential effects of varying inositol:peptide molar ratios on the binding equilibria of inositol and morphologies of monomers and aggregates of A$\beta$(16-22). From our microsecond time scale simulations, we compute binding constants (K$_{d}$) and successively characterize binding modes of inositol with the peptide aggregation states considered. The results of our study have implications for both the mechanism of amyloid inhibition by small molecules and the rational design of more efficacious putative therapeutics for AD and related amyloid disorders.

% TODO should add a sentence here on why we choose the peptide KLVFFAE.  Better yet, rearrange the first reference to KLVFFAE to here and only talk about KLVFFAE in the context of full length fibril formation. ie. it is the CHC of Abeta.

% CHANGED Have I clearly explained what was done?
% Need to convey: (1) Why we picked these different states (better explain hypothesis/rationale) (2) That we copmute binding affinity
% Need to disentangle this sentence a bit more.
% Why three different types of aggregates
% Why different molar ratios, concentrations
% Binding affinities to gauge activity
% See previous writings on for a good wording so Im not rewriting the wheel again

\section{Material and Methods} % (fold)
\label{sec:material_and_methods}

\subsection{Simulation Parameters and Protocol}

To eliminate terminal charge effects, the A$\beta$(16-22) peptide was acetylated and amidated at the N- and C-termini respectively. The peptide was represented by the OPLS-AA/L forcefield.\cite{Jorgensen:1996p19} The extended OPLS-AA force field for carbohydrates\cite{Damm:1997p36} was used to model inositol stereoisomers. The TIP3P water model\cite{Jorgensen:1983p40} was used to represent the solvent. All MD simulations were performed using the GROMACS simulation package,\cite{Hess:2008p264,VanDerSpoel:2005p56} versions 3.3.x and 4.0.x. Unless otherwise noted, the following parameters were used for all simulations in this study. The leapfrog Verlet integration algorithm was used with an integration timestep of 2 femtoseconds. Long-range electrostatic interactions were calculated using Particle Mesh Ewald (PME) summation with a Fourier grid spacing of 0.15 nm and a real-space cutoff of 1.3 nm.\cite{Darden:1993p266} The short-range nonbonded van der Waals interactions were switched to zero from 1.1 nm to 1.2 nm. The temperature was controlled at 300 K using the Berendsen thermostat in the NpT ensemble.\cite{Berendsen:1984p26} Pressure was controlled by the Berendsen thermostat at 1 atm with a coupling constant of 1.0 ps.\cite{Berendsen:1984p26} The SHAKE algorithm was used to constrain covalent bonds that contain hydrogens.\cite{Ryckaert:1977p30} In all simulations, a cubic box was used with periodic boundary conditions. Prior to data collection, 500 steps of energy minimization were first performed using the conjugate gradient algorithm, followed by equilibration with isotropic pressure coupling. The center of mass (COM) rotation and translation were removed at every step. Additional details of simulation setup and total sampling time for all systems investigated in this study are listed in Table 1.
% TODO add time to isotropic pressure coupling

Molecular simulations of monomeric A$\beta$(16-22) were performed using the simulated tempering distributed replica sampling algorithm (STDR).\cite{Rauscher:2009p166} STDR is a generalized-ensemble simulation method that allows each replica in the simulation to undergo a random walk in temperature to enhance conformational sampling.\cite{Rauscher:2009p166,Rodinger:2006p78} The STDR simulation was performed using 33 replicas undergoing canonical sampling (NVT ensemble) at exponentially-spaced temperatures ranging from 280 K to 694 K. 108 ns of simulation at each temperature were generated using Langevin dynamics (implemented by the stochastic dynamics integrator in GROMACS 3.3.x), for a total simulation time of 3.564 $\mu$s.
% TODO Might have to look up the details of the NVT simulation for KLVFFAE monomer as that could differ from the simulation parameters above

A set of 1117 structures was drawn randomly from STDR simulations such that the probability distribution of the end to end distance of these peptides closely approximated that of the equilibrium ensemble of KLVFFAE at 296 K. These structures were used as starting points for constant-temperature MD simulations (NpT ensemble) in the presence of 123 mM inositol at an inositol:peptide molar ratio of 2:1. Short 5 ns MD simulations were performed for each structure in the presence and absence of inositol at T=300 K. In addition, 15 ns of simulation in the presence of  \emph{scyllo}- or  \emph{chiro}-inositol molecules at inositol:peptide molar ratios of 15:1 were performed using each of 550 structures drawn randomly from the larger set of 1117 structures.

For the aggregate states of A$\beta$(16-22), total sampling times of 1.44 $\mu$s and 1.5 $\mu$s were generated for disordered aggregates and $\beta$-oligomers of A$\beta$(16-22), respectively. Each of the disordered aggregate simulations was initialized with four peptides drawn at random from the pool of structures obtained at $T$=296 K from the STDR simulation of the monomer. Peptides were initially mono-dispersed and placed approximately equidistant from each other in the simulation box. 2, 15 and 45 molecules of inositol were added at inositol:peptide molar ratios of 1:2, 4:1 and 10:1 respectively.

The A$\beta$(16-22) $\beta$-oligomer consists of two eight-stranded antiparallel $\beta$-sheets stacked in a ``face-to-face'' and antiparallel manner and was constructed based on solid-state NMR evidence\cite{Balbach:2000p49} using a method similar to that described in our previous study.\cite{Li:2012p853} Consistent with the experimental study, the $\beta$-sheets were stacked so that charged side chains (lysine and glutamate) are located on the solvent-exposed faces of the $\beta$-oligomer. 
% TODO define face to face

Simulations were performed in the presence of  \emph{scyllo}- and  \emph{chiro}-inositol at inositol:peptide molar ratios of 1:4 and 4:1 using A$\beta$(16-22) $\beta$-oligomer structures taken from every 10th frame of a 100-ns long trajectory in the absence of inositol. For the higher molar ratio, two separate sets of simulations were performed, one at a concentration of 62 mM and the other at 208 mM, corresponding respectively to 15 and 64 molecules of inositol in the simulation cell.

A set of five independent 100-ns simulations was performed for both  \emph{scyllo}- and  \emph{chiro}-inositol. For comparison, these simulations were repeated at a lower inositol:peptide molar ratio of 1:4 with a similar concentration, using initial snapshots extracted from a single 100 ns trajectory of the $\beta$-oligomer in absence of inositol. Independent simulations in the presence of four molecules of inositol (either  \emph{scyllo}- or  \emph{chiro}-inositol) at a concentration of 40 mM were performed for a total of 500 ns per stereoisomer. See Table 1 for a summary of all simulations performed in this study.

\subsection{Analysis Protocol} % (fold)
\label{sub:analysis}

	The DSSP hydrogen-bonding criteria were used to determine the presence of a hydrogen bond: (1) the distance between donor-acceptor and hydrogen-acceptor atoms is less than 0.35 nm; (2) the distance between the acceptor hydrogen and the acceptor is less than 0.25 nm; and (3) the angle formed by the donor, acceptor hydrogen, and acceptor is greater than 120\mathdeg.\cite{Kabsch:1983p31} Nonpolar contacts between inositol and the peptide were defined by the distance between the center of mass of inositol and the carbon atoms of side chains. The total number of intermolecular nonpolar contacts was calculated by considering all side chain carbon atom pairs within 0.45 nm. Time series were smoothed using a running average with a window of length 500 ps.

	Dissociation constants for inositol K$_d$, were calculated based on the presence of intermolecular contacts (either hydrogen bonding or nonpolar) as defined above. Defining the binding reaction of inositol by
% Equations used in the KLVFFAE paper
\[ \left[ Protein\cdot Inositol \right] \rightleftharpoons \left[ Protein \right] +\left[ Inositol \right]. \]
The associated equilibrium constant for the dissociation is
\[ K_{d} = f_{ub}\frac{\left[ Protein \right]\left[ Inositol \right]}{\left[Protein \cdot Inositol\right]} \]
where f$_{ub}$ denotes the fraction of unbound protein.	

The potential of mean force (PMF) for the binding of  \emph{scyllo}-inositol and  \emph{chiro}-inositol to the phenylalanine side chain was computed using two reaction coordinates: (1) the distance between the center of geometry of inositol and the F side chain (excluding the C$_{\beta}$ atom), $r$; and (2) the angle between the mean plane of the cyclohexane ring of inositol and that of the benzene ring of F, $\theta$. The PMF is given by $\mathit{W}=-RT\ln\rho\left(r,\theta\right)$, where $\rho\left(r,\theta\right)$ is the probability distribution of $r$ and $\theta$. All error bars were estimated using block averaging or by computing the standard deviation in the mean of the property of interest over all independent simulations.
% TODO Redo this calculation for the PMF.  It's not really a PMF right now. See discussion with Chris today regarding the properly way of doing this - look at his calculations in his recent JCTC papers
  
The DSSP algorithm was used for the analysis of secondary structure of the disordered oligomer with the N- and C-termini of the peptides excluded. The distance between the first and last C$_{\alpha}$ atoms defines the end-to-end distance. The spatial probability density of inositol was computed using the VolMap tool from the Visual Molecular Dynamics (VMD) software package.\cite{Humphrey:1996p850}

\section{Results}

In the sections below, we successively characterize the binding equilibrium of inositol and its effect on the morphologies of monomers, disordered and protofibrillar oligomers of A$\beta$(16-22).  

\subsection{Monomer}
% TODO Throughout this section, you are not showing “binding” strength. You are showing propensity to form hydrophobic contacts and hydrogen bonds. Make that distinction clear, then show the data. Finally, recap the take away. 

We performed simulations of an A$\beta$(16-22) monomer successively in pure water and in the presence of \emph{scyllo}- and \emph{chiro}-inositol at inositol:peptide molar ratios of 2:1 and 15:1.  It is important to note that inositol:peptide molar ratios were chosen such that the corresponding inositol:residue ratios are above (2:1) and below (<1:1) the inositol:residue molar ratio where inhibition of A$\beta$42 fibrils was observed in vitro.\cite{McLaurin:1998p176}
%Although ligand:peptide ratios in our simulations were under the 25:1 molar ratio where inhibition of A$\beta$42 fibrils was observed in the in vitro study\cite{McLaurin:1998p176}
% TODO Look at RP's comments on the first draft

Independently of the presence of inositol, A$\beta$(16-22) is a disordered peptide in solution (Fig. 2A), and is able to adopt both collapsed and extended states over the timescales of our simulation. The conformational equilibria of A$\beta$(16-22), as measured by peptide end-to-end distance distributions, was preserved in the presence of inositol at both inositol:peptide molar ratios considered (Fig. 2A).  

Inositol molecules bound weakly and reversibly to the monomer of A$\beta$(16-22). Dissociation constants K$_d$(\emph{scyllo}) = 216 $\pm$ 2 mM, K$_d$(\emph{chiro}) = 222 $\pm$ 1 mM were obtained at a molar ratio of 2:1, and K$_d$(\emph{scyllo}) = 179 $\pm$ 1 mM, K$_d$(\emph{chiro}) = 168 $\pm$ 2 mM at a molar ratio of 15:1. Increasing the molar ratio of inositol:peptide by more than 7-fold only decreased the K$_d$ by a factor of 1.2, suggesting that inositol does not bind cooperatively to the peptide monomer.

% Because of the similarity between the results of the systems at 2:1 and 15:1 molar ratios, in the rest of the monomer results section we will only refer to the data for the 15:1 inositol:peptide molar ratio.
% TODO Regis says to show results at both ratios

The predominant binding mode of inositol with monomers involved only hydrogen-bonding interactions: $\sim$77\% of bound \emph{scyllo}-inositol and $\sim$81\% of bound \emph{chiro}-inositol molecules formed at least one hydrogen bond with the monomer (Fig. 2B). Representative examples of \emph{scyllo}- and \emph{chiro}-inositol binding are depicted in Fig. 2C.  Inositol binds not only to the peptidic backbone of A$\beta$(16-22) (Fig. 2C), but also to the charged side chains of glutamic acid (E) and lysine (K) residues. Both stereoisomers have similar hydrogen bonding propensities to each of the residues in the peptide. In particular, inositol bound most favorably to E (Fig. 2D,E), where its interaction was dominated by hydrogen bonding to the carboxylate group (Fig. 2E). Furthermore, we found an equal fraction of monodentate and bidentate binding to the carboxylate group of E (data not shown). In contrast, less than 1\% of inositol molecules bound to K formed involved multiple hydrogen bonds to the ammonium group (Fig. 2A,B).
% TODO IGNORE Expand on the plots and results here

Nonpolar contacts also played a significant role in inositol binding: as shown in Fig. 2D, $\sim$11\% of \emph{scyllo}- and $\sim$9\% of \emph{chiro}-inositol molecules formed nonpolar contacts with the monomer. Furthermore, consistent with our previous study on (GA)$_4$, stereochemistry appears to modulate nonpolar binding, but not hydrogen bonding: \emph{scyllo}-inositol was found to bind preferentially to nonpolar groups on phenylalanine and glutamate over the other aliphatic nonpolar residues (L, V and A). By contrast, \emph{chiro}-inositol made nonpolar contacts to F18, F19 and E22 with the same probability as with A21 (Fig. 2C).
% ADD the results on the hydrophobic contacts to Glu and explain why \emph{scyllo}- makes a lot more contacts with Glu than \emph{chiro}- (may require more analysis)
	
To characterize the binding geometry of inositol to F in detail, we performed simulations of phe dipeptide in the presence of \emph{scyllo}- or \emph{chiro}-inositol. The preferential nonpolar binding of \emph{scyllo}-inositol is explained by the existence of a face-to-face stacking mode specific to the stereochemistry of \emph{scyllo}-inositol (Fig. 2C). This binding mode has an approximate free energy of binding of -0.5 kcal/mol and appears on the potential of mean force (PMF) for \emph{scyllo}-inositol as a minimum of the distance between the center of inositol and phenyl rings, $r$ = 0.45 nm, and angle between the planes of the rings, $\theta = 10$\mathdeg (Fig. 2D). By contrast, this binding mode was unstable for \emph{chiro}-inositol (Fig 2D), which lacks planar nonpolar faces because of its adjacent axial hydroxyl groups (Fig 2C).
% TODO More details needed for how the FE of the binding mode was calculated 


\subsection{Disordered oligomer}

To probe the effect of inositol on the early aggregation stages of A$\beta$(16-22), we performed multiple sets of independent MD simulations with four initially disperse A$\beta$(16-22) monomers with inositol:peptide molar ratios of 1:2, 4:1, and 10:1, corresponding to inositol concentrations of 52 mM, 70 mM and 209 mM respectively (see Table 1). In each of our simulation studies, the peptides spontaneously aggregated with one another over the course of approximately 40 ns, through both hydrogen bonding and nonpolar contacts, to form a disordered oligomer (Fig. S1, Fig. 5A). A significant fraction of the residues in the aggregate were in the coil conformation, with only a small fraction of $\beta$-sheet residues occurring in some of the systems within the 180-ns simulations (Fig. 4D). Importantly, the distribution of the overall secondary structure of the oligomer in each of the systems was not affected by the presence of inositol, regardless of inositol:peptide molar ratio and inositol concentration (Fig. 4D).

We further characterized the molecular organization of the aggregate by quantifying peptide inter- and intramolecular hydrogen-bonding and nonpolar contacts as measures of the extent of aggregation (Fig. 5). For simulations at both 4:1 and 10:1 molar ratios, the hydrophobic packing was not affected by the presence of inositol: the equilibrium number of inter-peptide hydrophobic contacts formed per peptide remained approximately 1.5 (Fig. 5A, Fig. S2). The number of intermolecular peptide-peptide hydrogen bonds per chain was approximately the same as the number of intramolecular hydrogen bonds (1.5 vs. 1) (Fig. 5B,C). Overall, the presence of inositol had no significant effect on the aggregation kinetics or on the morphology of A$\beta$(16-22) oligomers as measured by intermolecular and intramolecular contacts.

The dissociation constants of inositol with the disordered oligomer at molar ratios 1:2 and 4:1 ranged from 30 to 40 mM (see Table 2). Although this is much smaller than K$_d$ of the monomer, when normalized by the number of peptides in the system, K$_d$ (oligomer) $\times$ 4 = 170 mM = K$_d$(monomer), indicating that inositol does not bind small oligomeric aggregates cooperatively.
% TODO CN again brought this part up again.  Put the cooperativity stuff with the Kds in a table as I had done in my first paper and refer to that table. Also add in the numbers from the first paper for comparisons

Similar binding propensities to polar and nonpolar groups of the peptide were found at both higher and lower inositol concentrations. Furthermore, there were no differences between the distribution of bound \emph{chiro}- and \emph{scyllo}-inositol along the peptide sequence (Fig. 4C). As shown by the fraction of nonpolar contacts to each residue depicted in Fig. 4C, at the lower inositol:peptide molar ratio of 4:1, the nonpolar contact patterns of \emph{scyllo}- are similar to those of \emph{chiro}-inositol, with the exception of a slight propensity for higher nonpolar contacts to E. However, at a molar ratio of 10:1, \emph{scyllo}-inositol made significantly more nonpolar contacts to F and E side chains than \emph{chiro}-inositol.


\subsection{$\beta$-oligomer}

Finally, we examine the binding of inositol to an ordered protofibrillar-like aggregate henceforth referred to as the $\beta$-oligomer. In the absence of inositol, rectangularly-stacked sheets (Fig. SI 3A-C) spontaneously evolved into a twisted $\beta$-sheet structure with significant inter-strand twisting along the long-axis of the fibril and a slight inter-sheet twist (Fig. 6A, Fig. SI 3D). The resulting structure has an average inter-strand twist angle of approximately 25\mathdeg for the top sheet and 15\mathdeg for the bottom sheet. 

The $\beta$-oligomer is comprised of two faces and four edges (Fig 6A,B), each of which contains a hydrophobic shallow groove surrounded by polar or charged groups (Fig. 7A,B).  In particular, the grooves on the faces are formed by solvent-exposed phenylalanine, valine and alanine residues and are surrounded on either side by charged side chains of lysine and glutamate. The grooves at the edges, on the other hand, do not have solvent-exposed aromatic residues, and are instead surrounded by polar groups (exposed peptide backbone or termini).

The spatial probability densities of bound inositol in Fig. 6B show that, overall, inositol predominantly binds at the faces. Both stereoisomers have similar affinities with $\beta$-oligomers: \emph{scyllo}- and \emph{chiro}-Inositol have respective K$_d$s of 15 $\pm$ 4 mM and  21 $\pm$ 9 at a inositol:peptide molar ratio of 1:4, and K$_{d}$s of 5 $\pm$ 3 mM and 1 $\pm$ 1 at molar ratios of 4:1 (Table 2).   Furthermore, K$_d$s of both stereoisomers decreased significantly (corresponding to an increase in affinities) with an increase of inositol:peptide molar ratio (Fig. 7), indicating that inositol binds $\beta$-oligomers cooperatively.  Examples of such cooperative binding modes are depicted in Fig. 7, where inositol molecules are clustered together on shallow grooves on the faces of the $\beta$-oligomer.

Consistent with these global binding modes, inositol has the highest binding propensity to nonpolar groups of F and E, and charged groups of lysine and glutamate, all of which are located at the faces of the $\beta$-oligomer (Fig. SI 3C).  In contrast, inositol did not penetrate the hydrophobic core of the oligomer: The fraction of hydrogen bonds to each residue depicted in Fig. 6E show that little or no hydrogen bonds were made with the residues in the central hydrophobic core region. Although inositol molecules sometimes intercalated between strands, these rare events did not lead to the disaggregation of the preformed $\beta$-oligomer in any of our simulations.

Finally, a key difference between binding modes of scyllo- and chiro-inositol is that their binding propensity for hydrophobic group is modulated by both ligand:peptide molar ratio and stereochemistry. This is a result not previously observed for the monomer and disordered oligomer of A$\beta$(16-22). Specifically, at similar effective concentrations, the increase of inositol:peptide molar ratio from 1:4 to 4:1, significantly shifted the propensity to bind hydrophobic groups for \emph{scyllo}-inositol, but not for \emph{chiro}-inositol: the fraction of inositol molecules bound by nonpolar contacts is 22 $\pm$ 3\% for \emph{scyllo}- versus 6 $\pm$ 1\% for \emph{chiro}-inositol (Fig. 6C).

% April 30th 2012
% This is a key result ... I feel like I could say more. Also this paragraph feel more right in the discussion.  See what Regis has to say.  
% Why does scyllo bind more at higher molar ratios?
% Also missing - I think is a comparisons of binding modes (per residue) at different molar ratios.
% Note that Figure 5D show that the only diff bn sc and ch is the nonpolar binding at glutamate and not phe.  phe propensity is similar.  Face to face stacking => more nonpolar contacts (if counting) atoms than chiro. These two results together imply that perhaps the face to face stacking mode doesn't happen all that often for sc binding here.  Actually, it might be that because these are nonpolar contacts made for each residue, inositol binding modes are not separated by NP, P, and nonpolar & polar, and they are all lumped into here. 
% TODO would be good to look at the occurrence of this face to face stacking ...here ... does it happen for scyllo more than chiro. 
% TODO which resides are scyloo preferentially binding to in that 22%? is it all glu? or something else?


\section{Discussion} % (fold)
% Our results are consistent with previous studies of/on
% Together, these results suggest/indicate ...
% Quantitative agreement
% Qualitatively agrees

% There are several trains of thought that I'm trying to convey below (not entirely linear) ... so have a preamble to make that clear. 
% I actually think its a lot cleaner to keep the cooperativity in the results and have a table showing the numbers which makes this obvious.

% Have I put too much of the results discussion towards the end to draw analogy with sugar binding? They should be shifted up, and their relationship to inhibition should be expounded upon. Check for when reading the entire discussion.

% TODO ADD/Find EVIDENCE FOR KLVFFAE IMPORTANT FOR STACKING -- I could be just bullshitting here, in any case, stacking interface mediated by KLVFFAE is not the only reason why binding to this segment is useful .. LVFFA is also at the fibril core .. disrupting packing here disrupts amyloid formation of Abeta42]. 
% \cite{Takeda:2009es} -- Caflish and Derreumaux computation evidence that 12-22 is the "aggregation interface"

% TODO Look into this further: My results are exactly consistent with the mechanism proposed by Porat for polyphenol inhibition! planar + equatorial OHs => target amyloidgenic core! Except I have data to prove it.
%EGCG, an effective inhibitor of Abeta42 amyloid formation, has this structure and was shown to bind fibrillar forms of Abeta42.\cite{Bieschke:2010ju}
\label{sec:discussion}

In the above analysis, we have systematically characterized the binding of \emph{scyllo}-inositol and its inactive stereoisomer, \emph{chiro}-inositol, with monomer and aggregates of A$\beta$(16-22). In the sections below, we consider in detail, the implications of our findings for the activity of inositol in the A$\beta$42 amyloid aggregation pathway.

Consistent with our results on the binding equilibrium of inositol with model amyloidogenic peptides,\cite{Li:2012p853} both \emph{scyllo}- and \emph{chiro}-inositol bound weakly, with similar binding affinities, to the peptide states of A$\beta$(16-22) considered. However, because of sequence-specific binding modes, the range of dissociation constants computed in this study is about an order of magnitude less than that of the previous study: K$_{d}$s of inositol were measured to be between 0.005 - 0.200 M for A$\beta$(16-22) versus 0.04 - 1 M for model peptides.

Both \emph{scyllo}- and \emph{chiro}-inositol bound the weakest to monomers, with K$_d$s of 179 $\pm$ 1 and 168 $\pm$ 2, respectively, at the highest molar ratio (Table 2). Because binding to the monomer is not cooperative, a predicted K$_d$ of monomeric A$\beta$42 can be obtained by linearly scaling the K$_d$ of inositol for monomeric A$\beta$(16-22) with the ratio of peptide lengths of A$\beta$(16-22) to A$\beta$(1-42).  Taking the K$_{d}$ of inositol at the highest molar ratio, this value would be 170 mM/6 = 28 mM, which is an order of magnitude higher than the highest concentration (1 mM) at which inhibition was observed in vitro.\cite{McLaurin:2000p64}  Moreover, our results indicate that the conformational equilibria of monomeric A$\beta$(16-22) is not displaced in the presence of inositol (Figure 2A). Taken together, we speculate that inositol is unlikely to act as a drug by binding to and displacing the conformational equilibria of monomers of A$\beta$42.

Similar to monomeric A$\beta$(16-22), inositol bound to small disordered oligomers weakly, with K$_d$s in the range of 30 - 40 mM for both \emph{scyllo}- and \emph{chiro}-inositol (Table 2). Moreover, monomeric A$\beta$(16-22) peptides, independently of the presence of inositol, aggregated to form a morphologically similar state with only a small amount of secondary structure. This aggregate is predominantly formed from intermolecular nonpolar contacts (Fig. 5A,B), indicating that hydrophobic association is the primary driving force for the self-assembly of monomeric A$\beta$(16-22) peptides in solution. However, inositol binds both monomers and small oligomers of A$\beta$(16-22) predominantly via hydrogen bonding interactions (Figs. 2B,4B,6C), suggesting that it is unlikely to disrupt the hydrophobic association of nonpolar groups. On the basis of these results, we speculate that inositol is unlikely to prevent early oligomer formation in the A$\beta$42 fibrillation pathway by binding to A$\beta$(16-22).

%In contrast to its binding affinities for monomers and disordered aggregates of A$\beta$(16-22), 
Inositol displays a much higher binding affinity for $\beta$-oligomers, with respective K$_d$s of 1 mM $\pm$ 1 and 5 mM $\pm$ 3 for \emph{chiro}- and \emph{scyllo}-inositol, respectively. Notably, these K$_{d}$ values are in quantitative agreement with experimental concentrations (0.5 - 1 mM) sufficient for the inhibition of A$\beta$42 fibrillation in vitro,\cite{McLaurin:2000p64} suggesting that $\beta$-oligomers may be an in vitro binding partner of inositol. Furthermore, these results suggest that morphology-specific binding modes, in addition to the presence of sequence-specific interactions, may play a role in amyloid inhibition by inositol.

The increase in inositol's binding affinity for the $\beta$-oligomer from its affinities for monomers and disordered oligomers may be explained by structural features present on the former, but not on the latter species. First, the $\beta$-oligomer has a much larger effective surface area, which can accommodate multiple bound inositol molecules (Fig. 7). Second, as a direct consequence of its morphology,  $\beta$-oligomers present grooves on its surfaces that collocate the residues (i.e. F and E) capable of high affinity interactions with inositol. Supporting our results, recent simulation studies of A$\beta$40 fibrillar fragments and NSAIDs ibuprofen and naproxen\cite{Takeda:2010p34,Raman:2009p47} suggested that their inhibitory activities may be related to their ability to bind cooperatively and to form clusters on the surface of A$\beta$40 fibrillar aggregates.

A key finding of this study is that the stereospecificity of binding by inositol stereoisomers is not differentiated by their K$_{d}$s, but instead by their binding modes with nonpolar groups of side chains with specific geometries. In particular, due to the presence of planar hydrophobic faces, \emph{scyllo}-inositol, unlike \emph{chiro}-inositol, can bind phenylalanine side chains in a planar face-to-face stacking mode (Fig. 2D). However, this difference in binding mode between \emph{scyllo}- and \emph{chiro}-inositol does not appear to influence their binding equilibria with monomers and small disordered aggregates of A$\beta$(16-22).

% TODO check that it was due to molar ratio that increased the nonpolar contacts
Instead, the stereochemistry difference between \emph{scyllo}- and \emph{chiro}-inositol modulates their binding specificity to hydrophobic surfaces on $\beta$-oligomers. Binding to $\beta$-oligomers is driven more by the hydrophobic effect, at higher molar ratios, for \emph{scyllo}- than \emph{chiro}-inositol: \emph{scyllo}-inositol is 4 times more likely than \emph{chiro}-inositol to bind via only nonpolar contacts (Fig. 5C). In support of these results, \emph{scyllo}-inositol is reported to be much less soluble experimentally than \emph{chiro}-inositol.\cite{Husson:1998wj} Because of the larger hydrophobic surface area on the $\beta$-oligomer in combination with \emph{scyllo}-inositol's higher binding specificity for phenylalanine,  \emph{scyllo}-'s propensity to adsorb onto hydrophobic groups was dramatically shifted at higher molar ratios.  By contrast, an equal fraction of \emph{chiro}-inositol was bound via only nonpolar contacts to all states considered (Figs. 2B, 4B, 6C), indicating that \emph{chiro}- binds hydrophobic groups in a nonspecific manner independently of aggregate morphology.
 
% ie. like sugars? face stacking and the glu coordination provides specificity for sc but not chiro?
Similar to our findings, a recent combined MD-simulation and biophysical study\cite{Wang:2010p204} on the polyphenolic inhibitor EGCG,\cite{Ehrnhoefer:2008p8,Wang:2010p204} showed that an increase of EGCG:A$\beta$42 molar ratio shifted the predominant binding interaction of EGCG from hydrogen-bonding to hydrophobic interactions.
% TODO ... and is likely the explanation for why at a higher molar ratio EGCG works better?? check this paper.
Taken together, our results suggest that the activity of inositol is likely to involve binding to $\beta$-sheet oligomers of A$\beta$42, where geometries of nonpolar groups (ie. aromatic versus aliphatic) of solvent-exposed side chains modulate the stereospecificity of binding.


\subsection{Similarity to Carbohydrate Binding}

A striking result of our study is the characteristically sugar-like \cite{Taroni:2000p195} binding affinities and modes of inositol. Similar to inositol, monosaccharides exhibit millimolar binding affinities for lectins, a class of sugar-binding proteins.\cite{Wohlert:2010p201,Geisler:2010p188} Furthermore, sugar binding usually involves a combination of hydrogen bonds between hydroxyl groups and charged side chains (D or E) and nonpolar stacking of aromatic moieties, residues important for the recognition and selectivity of sugar enantiomers by lectins.\cite{Sharon:2001p215}. Consistent with these observations, our results indicate that inositol have the highest binding propensities to F and E (Figure 6D-E). Moreover, from our simulations, the free energy of the binding mode with phenyl ring of F is approximately -0.5 kcal/mol, and is in agreement with that of glucose binding to the indole group of tryptophan obtained from a recent MD simulation\cite{Wohlert:2010p201} and NMR studies\cite{Kiehna:2007p163}. Finally, the shallow amphiphilic grooves found at the surface of $\beta$-oligomers is strikingly analogously to binding sites located at the surface of carbohydrate binding domains.\cite{Taroni:2000p195,Kulharia:2009p212,Weis:1996p225}  Taken together, our above results suggest that inositol may bind in carbohydrate-like binding sites on $\beta$-sheet surfaces involving A$\beta$(16-22).


\subsection{Proposed mechanism of A$\beta$ amyloid inhibition by inositol}

Mature amyloid fibrils are thought to form either by $\beta$-strand addition along the long axis of the fiber (elongation) or by lateral face-to-face association with other protofibrils.\cite{Straub:2011p174} It follows then that small molecules that can disrupt either of these interactions may inhibit fibrillation. In particular for A$\beta$42, multiple experimental studies have shown that A$\beta$(16-22) is part of the $\beta$-sheet core of fibrils of A$\beta$42\cite{Hilbich:1992vy,Gordon:2001tj,Soto:2007tm,Watanabe:2002ti,Inouye:1993ku} and is suggested to be involved in the interface which mediates the stacking of constituent protofilaments of mature fibrils.\cite{Petkova:2002p192,Petkova:2006p48,Paravastu:2006p218,Luhrs:2005p229} Consistent with these observations, the fibrillar structure of full length A$\beta$42 from a solid-state NMR study show that the protofibril has two different $\beta$-sheet faces, one of which is formed by the A$\beta$(17-22) peptide segment.\cite{Luhrs:2005p229}

% Note need to tie together the last three pieces together a bit more. The drive home point is that the affinity + cooperativity + binding in grooves to glutamate & F => binding is exactly carbohydrate like, and this is the most likely binding site for inositol.
% Stereochemistry modulate binding specificities of sugar molecules - need examples

Our results suggest that A$\beta$(16-22) is a likely binding site for inositol in the full length A$\beta$42.  We hypothesize that \emph{scyllo}-, but not \emph{chiro}-inositol, binds at the fibrillar core of A$\beta$42 because of its higher binding specificity to nonpolar groups (particular to aromatic residues) on $\beta$-sheet surfaces involving A$\beta$(16-22). Hence, we propose that \emph{scyllo}-inositol, by binding to and coating the $\beta$-sheet surfaces of protofibrils involving A$\beta$(16-22), disrupts the lateral stacking of these oligomers, which ultimately leads to the inhibition of fibril formation.

Furthermore, on the basis of our results, we hypothesize that planar nonpolar faces, with multiple hydroxyl groups in equatorial positions around the ring confer binding specificity to small molecules for the amyloidogenic core of A$\beta$, and thus are key features for their activity. Consistent with this hypothesis, in vitro studies on small molecule derivatives of \emph{scyllo}-inositol showed that the substitution of a single hydroxyl by a ketone group resulted in the loss of activity (ie. fibrils were formed).\cite{Nitz:2008p13,Sun:2008p208,McLaurin:2000p64} Furthermore, polyphenols, a class of small molecules many of which are strong in vitro inhibitors of amyloid formation, all possess planar nonpolar faces with hydroxyl groups arranged equatorially. On the basis of their structure-activity relationships, a similar hypothesis was recently put forth as a possible explanation for their effectiveness in inhibiting amyloid formation. \cite{Porat:2006p33}

There are many differences between the $\beta$-oligomer of A$\beta$(16-22) and protofibrils of  the full-length A$\beta$ peptides. Our results indicate that inositol binding depends on both the morphology, ie. fibrillar versus non-fibrillar, and the surface physiochemical properties of the aggregate in consideration. Thus, alternative binding modes and binding sites of inositol may exist on aggregate forms of the full length A$\beta$42 peptide, which cannot be ruled out from the results of this study. As part of our future directions, we will perform systematic MD simulation studies of \emph{scyllo}- and \emph{chiro}-inositol and comparatively examine the binding mechanisms specific to aggregates of the full length A$\beta$ to rationally design more efficacious therapeutics for the treatment of AD.


\section{Conclusions} % (fold)
\label{sec:conclusions}

In this study, we have examined the binding of a small molecule inhibitor \emph{scyllo}-inositol, and its inactive stereoisomer \emph{chiro}-inositol to monomers, disordered and $\beta$-sheet aggregates of A$\beta$(16-22), the peptide thought to be the core aggregation region in the A$\beta$42 peptide. Notably, the K$_{d}$ of inositol ($\sim$ 1-5 mM) with $\beta$-oligomer is commensurate with the concentration at which inhibition of amyloid formation by A$\beta$42 is observed in vitro. Although both \emph{scyllo}- and \emph{chiro}-inositol, have similar binding affinities with all peptide states considered, we have uncovered a stereospecific face-to-face stacking stacking mode of \emph{scyllo}- with the phenylalanine side chains, which suggests a molecular basis for measured differences in activity.  Cooperative binding modes of inositol at grooves on the surface of the $\beta$-oligomer of A$\beta$(16-22) suggest a possible mechanism of fibril inhibition where by inositol prevents the lateral association of protofibrillar $\beta$-sheet oligomers. Furthermore, our results suggest that the fibril core of A$\beta$ amyloid aggregates contains carbohydrate-like binding sites. As such, carbohydrate-based (e.g. mono-, di- or trisaccharride) small molecules derivatives may be a promising avenue to explore for the rational design of novel therapeutics for AD.
% section conclusions (end)

\section*{Acknowledgements}
This work was made possible by the Centre for Computational Biology at the Hospital for Sick Children, the facilities of the Shared Hierarachical Academic Research Computing Network(SHARCNET, www.sharcnet.ca), the GPC supercomputer at the SciNet HPC Consortium and Compute/Calcul Canada. This work was supported in parts by the Canadian Institutes of Health Research (Grant No. MOP84496). R.P. is a CRCP chairholder.

\begin{singlespace}
\addcontentsline{toc}{section}{Bibliography}
\bibliographystyle{elsart-num}
\bibliography{/Users/grace/github/thesis/document/results2/results2}
\end{singlespace}

\section{Results}

In the sections below, we successively characterize the binding equilibrium of inositol and its effect on the morphologies of monomers, disordered and protofibrillar oligomers of A$\beta$(16-22).  


\subsection{Monomer}
% TODO Throughout this section, you are not showing “binding” strength. You are showing propensity to form hydrophobic contacts and hydrogen bonds. Make that distinction clear, then show the data. Finally, recap the take away. 

We performed simulations of an A$\beta$(16-22) monomer successively in pure water and in the presence of \emph{scyllo}- and \emph{chiro}-inositol at inositol:peptide molar ratios of 2:1 and 15:1.  It is important to note that inositol:peptide molar ratios were chosen such that the corresponding inositol:residue ratios are above (2:1) and below (<1:1) the inositol:residue molar ratio where inhibition of A$\beta$42 fibrils was observed in vitro.\cite{McLaurin:1998p176}
%Although ligand:peptide ratios in our simulations were under the 25:1 molar ratio where inhibition of A$\beta$42 fibrils was observed in the in vitro study\cite{McLaurin:1998p176}
% TODO Look at RP's comments on the first draft

Independently of the presence of inositol, A$\beta$(16-22) is a disordered peptide in solution (Fig. 2A), and is able to adopt both collapsed and extended states over the timescales of our simulation. The conformational equilibria of A$\beta$(16-22), as measured by peptide end-to-end distance distributions, was preserved in the presence of inositol at both inositol:peptide molar ratios considered (Fig. 2A).  

Inositol molecules bound weakly and reversibly to the monomer of A$\beta$(16-22). Dissociation constants K$_d$(\emph{scyllo}) = 216 $\pm$ 2 mM, K$_d$(\emph{chiro}) = 222 $\pm$ 1 mM were obtained at a molar ratio of 2:1, and K$_d$(\emph{scyllo}) = 179 $\pm$ 1 mM, K$_d$(\emph{chiro}) = 168 $\pm$ 2 mM at a molar ratio of 15:1. Increasing the molar ratio of inositol:peptide by more than 7-fold only decreased the K$_d$ by a factor of 1.2, suggesting that inositol does not bind cooperatively to the peptide monomer.

% Because of the similarity between the results of the systems at 2:1 and 15:1 molar ratios, in the rest of the monomer results section we will only refer to the data for the 15:1 inositol:peptide molar ratio.
% TODO Regis says to show results at both ratios

The predominant binding mode of inositol with monomers involved only hydrogen-bonding interactions: $\sim$77\% of bound \emph{scyllo}-inositol and $\sim$81\% of bound \emph{chiro}-inositol molecules formed at least one hydrogen bond with the monomer (Fig. 2B). Representative examples of \emph{scyllo}- and \emph{chiro}-inositol binding are depicted in Fig. 2C.  Inositol binds not only to the peptidic backbone of A$\beta$(16-22) (Fig. 2C), but also to the charged side chains of glutamic acid (E) and lysine (K) residues. Both stereoisomers have similar hydrogen bonding propensities to each of the residues in the peptide. In particular, inositol bound most favorably to E (Fig. 2D,E), where its interaction was dominated by hydrogen bonding to the carboxylate group (Fig. 2E). Furthermore, we found an equal fraction of monodentate and bidentate binding to the carboxylate group of E (data not shown). In contrast, less than 1\% of inositol molecules bound to K formed involved multiple hydrogen bonds to the ammonium group (Fig. 2A,B).
% TODO IGNORE Expand on the plots and results here

Nonpolar contacts also played a significant role in inositol binding: as shown in Fig. 2D, $\sim$11\% of \emph{scyllo}- and $\sim$9\% of \emph{chiro}-inositol molecules formed nonpolar contacts with the monomer. Furthermore, consistent with our previous study on (GA)$_4$, stereochemistry appears to modulate nonpolar binding, but not hydrogen bonding: \emph{scyllo}-inositol was found to bind preferentially to nonpolar groups on phenylalanine and glutamate over the other aliphatic nonpolar residues (L, V and A). By contrast, \emph{chiro}-inositol made nonpolar contacts to F18, F19 and E22 with the same probability as with A21 (Fig. 2C).
% ADD the results on the hydrophobic contacts to Glu and explain why \emph{scyllo}- makes a lot more contacts with Glu than \emph{chiro}- (may require more analysis)
	
To characterize the binding geometry of inositol to F in detail, we performed simulations of phe dipeptide in the presence of \emph{scyllo}- or \emph{chiro}-inositol. The preferential nonpolar binding of \emph{scyllo}-inositol is explained by the existence of a face-to-face stacking mode specific to the stereochemistry of \emph{scyllo}-inositol (Fig. 2C). This binding mode has an approximate free energy of binding of -0.5 kcal/mol and appears on the potential of mean force (PMF) for \emph{scyllo}-inositol as a minimum of the distance between the center of inositol and phenyl rings, $r$ = 0.45 nm, and angle between the planes of the rings, $\theta$ = 10$\degree$ (Fig. 2D). By contrast, this binding mode was unstable for \emph{chiro}-inositol (Fig 2D), which lacks planar nonpolar faces because of its adjacent axial hydroxyl groups (Fig 2C).
% TODO More details needed for how the FE of the binding mode was calculated 


\subsection{Disordered oligomer}

To probe the effect of inositol on the early aggregation stages of A$\beta$(16-22), we performed multiple sets of independent MD simulations with four initially disperse A$\beta$(16-22) monomers with inositol:peptide molar ratios of 1:2, 4:1, and 10:1, corresponding to inositol concentrations of 52 mM, 70 mM and 209 mM respectively (see Table 1). In each of our simulation studies, the peptides spontaneously aggregated with one another over the course of approximately 40 ns, through both hydrogen bonding and nonpolar contacts, to form a disordered oligomer (Fig. S1, Fig. 5A). A significant fraction of the residues in the aggregate were in the coil conformation, with only a small fraction of $\beta$-sheet residues occurring in some of the systems within the 180-ns simulations (Fig. 4D). Importantly, the distribution of the overall secondary structure of the oligomer in each of the systems was not affected by the presence of inositol, regardless of inositol:peptide molar ratio and inositol concentration (Fig. 4D).

We further characterized the molecular organization of the aggregate by quantifying peptide inter- and intramolecular hydrogen-bonding and nonpolar contacts as measures of the extent of aggregation (Fig. 5). For simulations at both 4:1 and 10:1 molar ratios, the hydrophobic packing was not affected by the presence of inositol: the equilibrium number of inter-peptide hydrophobic contacts formed per peptide remained approximately 1.5 (Fig. 5A, Fig. S2). The number of intermolecular peptide-peptide hydrogen bonds per chain was approximately the same as the number of intramolecular hydrogen bonds (1.5 vs. 1) (Fig. 5B,C). Overall, the presence of inositol had no significant effect on the aggregation kinetics or on the morphology of A$\beta$(16-22) oligomers as measured by intermolecular and intramolecular contacts.

The dissociation constants of inositol with the disordered oligomer at molar ratios 1:2 and 4:1 ranged from 30 to 40 mM (see Table 2). Although this is much smaller than K$_d$ of the monomer, when normalized by the number of peptides in the system, K$_d$ (oligomer) $\times$ 4 = 170 mM = K$_d$(monomer), indicating that inositol does not bind small oligomeric aggregates cooperatively.
% TODO CN again brought this part up again.  Put the cooperativity stuff with the Kds in a table as I had done in my first paper and refer to that table. Also add in the numbers from the first paper for comparisons

Similar binding propensities to polar and nonpolar groups of the peptide were found at both higher and lower inositol concentrations. Furthermore, there were no differences between the distribution of bound \emph{chiro}- and \emph{scyllo}-inositol along the peptide sequence (Fig. 4C). As shown by the fraction of nonpolar contacts to each residue depicted in Fig. 4C, at the lower inositol:peptide molar ratio of 4:1, the nonpolar contact patterns of \emph{scyllo}- are similar to those of \emph{chiro}-inositol, with the exception of a slight propensity for higher nonpolar contacts to E. However, at a molar ratio of 10:1, \emph{scyllo}-inositol made significantly more nonpolar contacts to F and E side chains than \emph{chiro}-inositol.


\subsection{$\beta$-oligomer}

Finally, we examine the binding of inositol to an ordered protofibrillar-like aggregate henceforth referred to as the $\beta$-oligomer. In the absence of inositol, rectangularly-stacked sheets (Fig. SI 3A-C) spontaneously evolved into a twisted $\beta$-sheet structure with significant inter-strand twisting along the long-axis of the fibril and a slight inter-sheet twist (Fig. 6A, Fig. SI 3D). The resulting structure has an average inter-strand twist angle of approximately 25$\degree$ for the top sheet and 15$\degree$ for the bottom sheet. 

The $\beta$-oligomer is comprised of two faces and four edges (Fig 6A,B), each of which contains a hydrophobic shallow groove surrounded by polar or charged groups (Fig. 7A,B).  In particular, the grooves on the faces are formed by solvent-exposed phenylalanine, valine and alanine residues and are surrounded on either side by charged side chains of lysine and glutamate. The grooves at the edges, on the other hand, do not have solvent-exposed aromatic residues, and are instead surrounded by polar groups (exposed peptide backbone or termini).

The spatial probability densities of bound inositol in Fig. 6B show that, overall, inositol predominantly binds at the faces. Both stereoisomers have similar affinities with $\beta$-oligomers: \emph{scyllo}- and \emph{chiro}-Inositol have respective K$_d$s of 15 $\pm$ 4 mM and  21 $\pm$ 9 at a inositol:peptide molar ratio of 1:4, and K$_{d}$s of 5 $\pm$ 3 mM and 1 $\pm$ 1 at molar ratios of 4:1 (Table 2).   Furthermore, K$_d$s of both stereoisomers decreased significantly (corresponding to an increase in affinities) with an increase of inositol:peptide molar ratio (Fig. 7), indicating that inositol binds $\beta$-oligomers cooperatively.  Examples of such cooperative binding modes are depicted in Fig. 7, where inositol molecules are clustered together on shallow grooves on the faces of the $\beta$-oligomer.

Consistent with these global binding modes, inositol has the highest binding propensity to nonpolar groups of F and E, and charged groups of lysine and glutamate, all of which are located at the faces of the $\beta$-oligomer (Fig. SI 3C).  In contrast, inositol did not penetrate the hydrophobic core of the oligomer: The fraction of hydrogen bonds to each residue depicted in Fig. 6E show that little or no hydrogen bonds were made with the residues in the central hydrophobic core region. Although inositol molecules sometimes intercalated between strands, these rare events did not lead to the disaggregation of the preformed $\beta$-oligomer in any of our simulations.

Finally, a key difference between binding modes of scyllo- and chiro-inositol is that their binding propensity for hydrophobic group is modulated by both ligand:peptide molar ratio and stereochemistry. This is a result not previously observed for the monomer and disordered oligomer of A$\beta$(16-22). Specifically, at similar effective concentrations, the increase of inositol:peptide molar ratio from 1:4 to 4:1, significantly shifted the propensity to bind hydrophobic groups for \emph{scyllo}-inositol, but not for \emph{chiro}-inositol: the fraction of inositol molecules bound by nonpolar contacts is 22 $\pm$ 3\% for \emph{scyllo}- versus 6 $\pm$ 1\% for \emph{chiro}-inositol (Fig. 6C).

% April 30th 2012
% This is a key result ... I feel like I could say more. Also this paragraph feel more right in the discussion.  See what Regis has to say.  
% Why does scyllo bind more at higher molar ratios?
% Also missing - I think is a comparisons of binding modes (per residue) at different molar ratios.
% Note that Figure 5D show that the only diff bn sc and ch is the nonpolar binding at glutamate and not phe.  phe propensity is similar.  Face to face stacking => more nonpolar contacts (if counting) atoms than chiro. These two results together imply that perhaps the face to face stacking mode doesn't happen all that often for sc binding here.  Actually, it might be that because these are nonpolar contacts made for each residue, inositol binding modes are not separated by NP, P, and nonpolar & polar, and they are all lumped into here. 
% TODO would be good to look at the occurrence of this face to face stacking ...here ... does it happen for scyllo more than chiro. 
% TODO which resides are scyloo preferentially binding to in that 22%? is it all glu? or something else?

\begin{abstract}
	Alzheimer's disease (AD) is a severe neurodegenerative disease with no cure. Currently, one method of targeting the underlying disease is to prevent or reverse the amyloid formation of A$\beta$42, a key pathological hallmark of AD. Scyllo-inositol is a polyol that exhibits stereochemistry-dependent inhibition of the formation of A$\beta$ fibrils \emph{in vitro}. We present molecular dynamics simulations of the monomeric, disordered and protofibrillar states of A$\beta$(16-22), an amyloid-forming peptide fragment found in the $\beta$-sheet core of full-length A$\beta$, successively with and without \emph{scyllo}-inositol and its inactive stereoisomer \emph{chiro}-inositol. Inositol binds monomers and disordered aggregates of A$\beta$(16-22) with similar affinities, whereas binding to $\beta$-sheet containing oligomers ($\beta$-oligomers) yield affinities in the low millimolar range commensurate with \emph{\emph{in vitro}} inhibitory concentrations of inositol. Furthermore, inositol adopts carbohydrate-like binding modes, where stereochemistry modulates the nonpolar binding specificity of inositol to glutamate and phenylalanine side chains. Our results suggest that \emph{scyllo}-inositol inhibits amyloid formation by coating the surface of protofibrillar aggregates of A$\beta$ and disrupting their lateral stacking into fibrils.
	
  % should say scyllo-inositol specific binding modes with glu and phe
  
  \textbf{Keywords}:
  amyloid
  A$\beta$42
  inhibition
  carbohydrate
  stacking
  oligomer
  protofibril
  stereochemistry
  
\end{abstract}
\section{Material and Methods} % (fold)
\label{sec:material_and_methods}

\subsection{Simulation Parameters and Protocol}

To eliminate terminal charge effects, the A$\beta$(16-22) peptide was acetylated and amidated at the N- and C-termini respectively. The peptide was represented by the OPLS-AA/L forcefield.\cite{Jorgensen:1996p19} The extended OPLS-AA force field for carbohydrates\cite{Damm:1997p36} was used to model inositol stereoisomers. The TIP3P water model\cite{Jorgensen:1983p40} was used to represent the solvent. All MD simulations were performed using the GROMACS simulation package,\cite{Hess:2008p264,VanDerSpoel:2005p56} versions 3.3.x and 4.0.x. Unless otherwise noted, the following parameters were used for all simulations in this study. The leapfrog Verlet integration algorithm was used with an integration timestep of 2 femtoseconds. Long-range electrostatic interactions were calculated using Particle Mesh Ewald (PME) summation with a Fourier grid spacing of 0.15 nm and a real-space cutoff of 1.3 nm.\cite{Darden:1993p266} The short-range nonbonded van der Waals interactions were switched to zero from 1.1 nm to 1.2 nm. The temperature was controlled at 300 K using the Berendsen thermostat in the NpT ensemble.\cite{Berendsen:1984p26} Pressure was controlled by the Berendsen thermostat at 1 atm with a coupling constant of 1.0 ps.\cite{Berendsen:1984p26} The SHAKE algorithm was used to constrain covalent bonds that contain hydrogens.\cite{Ryckaert:1977p30} In all simulations, a cubic box was used with periodic boundary conditions. Prior to data collection, 500 steps of energy minimization were first performed using the conjugate gradient algorithm, followed by equilibration with isotropic pressure coupling. The center of mass (COM) rotation and translation were removed at every step. Additional details of simulation setup and total sampling time for all systems investigated in this study are listed in Table 1.
% TODO add time to isotropic pressure coupling

Molecular simulations of monomeric A$\beta$(16-22) were performed using the simulated tempering distributed replica sampling algorithm (STDR).\cite{Rauscher:2009p166} STDR is a generalized-ensemble simulation method that allows each replica in the simulation to undergo a random walk in temperature to enhance conformational sampling.\cite{Rauscher:2009p166,Rodinger:2006p78} The STDR simulation was performed using 33 replicas undergoing canonical sampling (NVT ensemble) at exponentially-spaced temperatures ranging from 280 K to 694 K. 108 ns of simulation at each temperature were generated using Langevin dynamics (implemented by the stochastic dynamics integrator in GROMACS 3.3.x), for a total simulation time of 3.564 $\mu$s.
% TODO Might have to look up the details of the NVT simulation for KLVFFAE monomer as that could differ from the simulation parameters above

A set of 1117 structures was drawn randomly from STDR simulations such that the probability distribution of the end to end distance of these peptides closely approximated that of the equilibrium ensemble of KLVFFAE at 296 K. These structures were used as starting points for constant-temperature MD simulations (NpT ensemble) in the presence of 123 mM inositol at an inositol:peptide molar ratio of 2:1. Short 5 ns MD simulations were performed for each structure in the presence and absence of inositol at T=300 K. In addition, 15 ns of simulation in the presence of  \emph{scyllo}- or  \emph{chiro}-inositol molecules at inositol:peptide molar ratios of 15:1 were performed using each of 550 structures drawn randomly from the larger set of 1117 structures.

For the aggregate states of A$\beta$(16-22), total sampling times of 1.44 $\mu$s and 1.5 $\mu$s were generated for disordered aggregates and $\beta$-oligomers of A$\beta$(16-22), respectively. Each of the disordered aggregate simulations was initialized with four peptides drawn at random from the pool of structures obtained at $T$=296 K from the STDR simulation of the monomer. Peptides were initially mono-dispersed and placed approximately equidistant from each other in the simulation box. 2, 15 and 45 molecules of inositol were added at inositol:peptide molar ratios of 1:2, 4:1 and 10:1 respectively.

The A$\beta$(16-22) $\beta$-oligomer consists of two eight-stranded antiparallel $\beta$-sheets stacked in a ``face-to-face'' and antiparallel manner and was constructed based on solid-state NMR evidence\cite{Balbach:2000p49} using a method similar to that described in our previous study.\cite{Li:2012p853} Consistent with the experimental study, the $\beta$-sheets were stacked so that charged side chains (lysine and glutamate) are located on the solvent-exposed faces of the $\beta$-oligomer. 
% TODO define face to face

Simulations were performed in the presence of  \emph{scyllo}- and  \emph{chiro}-inositol at inositol:peptide molar ratios of 1:4 and 4:1 using A$\beta$(16-22) $\beta$-oligomer structures taken from every 10th frame of a 100-ns long trajectory in the absence of inositol. For the higher molar ratio, two separate sets of simulations were performed, one at a concentration of 62 mM and the other at 208 mM, corresponding respectively to 15 and 64 molecules of inositol in the simulation cell.

A set of five independent 100-ns simulations was performed for both  \emph{scyllo}- and  \emph{chiro}-inositol. For comparison, these simulations were repeated at a lower inositol:peptide molar ratio of 1:4 with a similar concentration, using initial snapshots extracted from a single 100 ns trajectory of the $\beta$-oligomer in absence of inositol. Independent simulations in the presence of four molecules of inositol (either  \emph{scyllo}- or  \emph{chiro}-inositol) at a concentration of 40 mM were performed for a total of 500 ns per stereoisomer. See Table 1 for a summary of all simulations performed in this study.

\subsection{Analysis Protocol} % (fold)
\label{sub:analysis}

	The DSSP hydrogen-bonding criteria were used to determine the presence of a hydrogen bond: (1) the distance between donor-acceptor and hydrogen-acceptor atoms is less than 0.35 nm; (2) the distance between the acceptor hydrogen and the acceptor is less than 0.25 nm; and (3) the angle formed by the donor, acceptor hydrogen, and acceptor is greater than 120$\degree$.\cite{Kabsch:1983p31} Nonpolar contacts between inositol and the peptide were defined by the distance between the center of mass of inositol and the carbon atoms of side chains. The total number of intermolecular nonpolar contacts was calculated by considering all side chain carbon atom pairs within 0.45 nm. Time series were smoothed using a running average with a window of length 500 ps.

	Dissociation constants for inositol K$_d$, were calculated based on the presence of intermolecular contacts (either hydrogen bonding or nonpolar) as defined above. Defining the binding reaction of inositol by
% Equations used in the KLVFFAE paper
\[ \left[ Protein\cdot Inositol \right] \rightleftharpoons \left[ Protein \right] +\left[ Inositol \right]. \]
The associated equilibrium constant for the dissociation is
\[ K_{d} = f_{ub}\frac{\left[ Protein \right]\left[ Inositol \right]}{\left[Protein \cdot Inositol\right]} \]
where f$_{ub}$ denotes the fraction of unbound protein.	

The potential of mean force (PMF) for the binding of  \emph{scyllo}-inositol and  \emph{chiro}-inositol to the phenylalanine side chain was computed using two reaction coordinates: (1) the distance between the center of geometry of inositol and the F side chain (excluding the C$_{\beta}$ atom), $r$; and (2) the angle between the mean plane of the cyclohexane ring of inositol and that of the benzene ring of F, $\theta$. The PMF is given by $\mathit{W}=-RT\ln\rho\left(r,\theta\right)$, where $\rho\left(r,\theta\right)$ is the probability distribution of $r$ and $\theta$. All error bars were estimated using block averaging or by computing the standard deviation in the mean of the property of interest over all independent simulations.
% TODO Redo this calculation for the PMF.  It's not really a PMF right now. See discussion with Chris today regarding the properly way of doing this - look at his calculations in his recent JCTC papers
  
The DSSP algorithm was used for the analysis of secondary structure of the disordered oligomer with the N- and C-termini of the peptides excluded. The distance between the first and last C$_{\alpha}$ atoms defines the end-to-end distance. The spatial probability density of inositol was computed using the VolMap tool from the Visual Molecular Dynamics (VMD) software package.\cite{Humphrey:1996p850}

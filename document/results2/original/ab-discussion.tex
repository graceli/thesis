% Our results are consistent with previous studies of/on
% Together, these results suggest/indicate ...
% Quantitative agreement
% Qualitatively agrees

% There are several trains of thought that I'm trying to convey below (not entirely linear) ... so have a preamble to make that clear. 
% I actually think its a lot cleaner to keep the cooperativity in the results and have a table showing the numbers which makes this obvious.

% Have I put too much of the results discussion towards the end to draw analogy with sugar binding? They should be shifted up, and their relationship to inhibition should be expounded upon. Check for when reading the entire discussion.

% TODO ADD/Find EVIDENCE FOR KLVFFAE IMPORTANT FOR STACKING -- I could be just bullshitting here, in any case, stacking interface mediated by KLVFFAE is not the only reason why binding to this segment is useful .. LVFFA is also at the fibril core .. disrupting packing here disrupts amyloid formation of Abeta42]. 
% \cite{Takeda:2009es} -- Caflish and Derreumaux computation evidence that 12-22 is the "aggregation interface"

% TODO Look into this further: My results are exactly consistent with the mechanism proposed by Porat for polyphenol inhibition! planar + equatorial OHs => target amyloidgenic core! Except I have data to prove it.
%EGCG, an effective inhibitor of Abeta42 amyloid formation, has this structure and was shown to bind fibrillar forms of Abeta42.\cite{Bieschke:2010ju}


\section{Discussion} % (fold)
\label{sec:discussion}

In the above analysis, we have systematically characterized the binding of \emph{scyllo}-inositol and its inactive stereoisomer, \emph{chiro}-inositol, with monomer and aggregates of A$\beta$(16-22). In the sections below, we consider in detail, the implications of our findings for the activity of inositol in the A$\beta$42 amyloid aggregation pathway.

Consistent with our results on the binding equilibrium of inositol with model amyloidogenic peptides,\cite{Li:2012p853} both \emph{scyllo}- and \emph{chiro}-inositol bound weakly, with similar binding affinities, to the peptide states of A$\beta$(16-22) considered. However, because of sequence-specific binding modes, the range of dissociation constants computed in this study is about an order of magnitude less than that of the previous study: K$_{d}$s of inositol were measured to be between 0.005 - 0.200 M for A$\beta$(16-22) versus 0.04 - 1 M for model peptides.

Both \emph{scyllo}- and \emph{chiro}-inositol bound the weakest to monomers, with K$_d$s of 179 $\pm$ 1 and 168 $\pm$ 2, respectively, at the highest molar ratio (Table 2). Because binding to the monomer is not cooperative, a predicted K$_d$ of monomeric A$\beta$42 can be obtained by linearly scaling the K$_d$ of inositol for monomeric A$\beta$(16-22) with the ratio of peptide lengths of A$\beta$(16-22) to A$\beta$(1-42).  Taking the K$_{d}$ of inositol at the highest molar ratio, this value would be 170 mM/6 = 28 mM, which is an order of magnitude higher than the highest concentration (1 mM) at which inhibition was observed in vitro.\cite{McLaurin:2000p64}  Moreover, our results indicate that the conformational equilibria of monomeric A$\beta$(16-22) is not displaced in the presence of inositol (Figure 2A). Taken together, we speculate that inositol is unlikely to act as a drug by binding to and displacing the conformational equilibria of monomers of A$\beta$42.

Similar to monomeric A$\beta$(16-22), inositol bound to small disordered oligomers weakly, with K$_d$s in the range of 30 - 40 mM for both \emph{scyllo}- and \emph{chiro}-inositol (Table 2). Moreover, monomeric A$\beta$(16-22) peptides, independently of the presence of inositol, aggregated to form a morphologically similar state with only a small amount of secondary structure. This aggregate is predominantly formed from intermolecular nonpolar contacts (Fig. 5A,B), indicating that hydrophobic association is the primary driving force for the self-assembly of monomeric A$\beta$(16-22) peptides in solution. However, inositol binds both monomers and small oligomers of A$\beta$(16-22) predominantly via hydrogen bonding interactions (Figs. 2B,4B,6C), suggesting that it is unlikely to disrupt the hydrophobic association of nonpolar groups. On the basis of these results, we speculate that inositol is unlikely to prevent early oligomer formation in the A$\beta$42 fibrillation pathway by binding to A$\beta$(16-22).

%In contrast to its binding affinities for monomers and disordered aggregates of A$\beta$(16-22), 
Inositol displays a much higher binding affinity for $\beta$-oligomers, with respective K$_d$s of 1 mM $\pm$ 1 and 5 mM $\pm$ 3 for \emph{chiro}- and \emph{scyllo}-inositol, respectively. Notably, these K$_{d}$ values are in quantitative agreement with experimental concentrations (0.5 - 1 mM) sufficient for the inhibition of A$\beta$42 fibrillation in vitro,\cite{McLaurin:2000p64} suggesting that $\beta$-oligomers may be an in vitro binding partner of inositol. Furthermore, these results suggest that morphology-specific binding modes, in addition to the presence of sequence-specific interactions, may play a role in amyloid inhibition by inositol.

The increase in inositol's binding affinity for the $\beta$-oligomer from its affinities for monomers and disordered oligomers may be explained by structural features present on the former, but not on the latter species. First, the $\beta$-oligomer has a much larger effective surface area, which can accommodate multiple bound inositol molecules (Fig. 7). Second, as a direct consequence of its morphology,  $\beta$-oligomers present grooves on its surfaces that collocate the residues (i.e. F and E) capable of high affinity interactions with inositol. Supporting our results, recent simulation studies of A$\beta$40 fibrillar fragments and NSAIDs ibuprofen and naproxen\cite{Takeda:2010p34,Raman:2009p47} suggested that their inhibitory activities may be related to their ability to bind cooperatively and to form clusters on the surface of A$\beta$40 fibrillar aggregates.

A key finding of this study is that the stereospecificity of binding by inositol stereoisomers is not differentiated by their K$_{d}$s, but instead by their binding modes with nonpolar groups of side chains with specific geometries. In particular, due to the presence of planar hydrophobic faces, \emph{scyllo}-inositol, unlike \emph{chiro}-inositol, can bind phenylalanine side chains in a planar face-to-face stacking mode (Fig. 2D). However, this difference in binding mode between \emph{scyllo}- and \emph{chiro}-inositol does not appear to influence their binding equilibria with monomers and small disordered aggregates of A$\beta$(16-22).

% TODO check that it was due to molar ratio that increased the nonpolar contacts
Instead, the stereochemistry difference between \emph{scyllo}- and \emph{chiro}-inositol modulates their binding specificity to hydrophobic surfaces on $\beta$-oligomers. Binding to $\beta$-oligomers is driven more by the hydrophobic effect, at higher molar ratios, for \emph{scyllo}- than \emph{chiro}-inositol: \emph{scyllo}-inositol is 4 times more likely than \emph{chiro}-inositol to bind via only nonpolar contacts (Fig. 5C). In support of these results, \emph{scyllo}-inositol is reported to be much less soluble experimentally than \emph{chiro}-inositol.\cite{Husson:1998wj} Because of the larger hydrophobic surface area on the $\beta$-oligomer in combination with \emph{scyllo}-inositol's higher binding specificity for phenylalanine,  \emph{scyllo}-'s propensity to adsorb onto hydrophobic groups was dramatically shifted at higher molar ratios.  By contrast, an equal fraction of \emph{chiro}-inositol was bound via only nonpolar contacts to all states considered (Figs. 2B, 4B, 6C), indicating that \emph{chiro}- binds hydrophobic groups in a nonspecific manner independently of aggregate morphology.
 
% ie. like sugars? face stacking and the glu coordination provides specificity for sc but not chiro?
Similar to our findings, a recent combined MD-simulation and biophysical study\cite{Wang:2010p204} on the polyphenolic inhibitor EGCG,\cite{Ehrnhoefer:2008p8,Wang:2010p204} showed that an increase of EGCG:A$\beta$42 molar ratio shifted the predominant binding interaction of EGCG from hydrogen-bonding to hydrophobic interactions.
% TODO ... and is likely the explanation for why at a higher molar ratio EGCG works better?? check this paper.
Taken together, our results suggest that the activity of inositol is likely to involve binding to $\beta$-sheet oligomers of A$\beta$42, where geometries of nonpolar groups (ie. aromatic versus aliphatic) of solvent-exposed side chains modulate the stereospecificity of binding.


\subsection{Similarity to Carbohydrate Binding}

A striking result of our study is the characteristically sugar-like \cite{Taroni:2000p195} binding affinities and modes of inositol. Similar to inositol, monosaccharides exhibit millimolar binding affinities for lectins, a class of sugar-binding proteins.\cite{Wohlert:2010p201,Geisler:2010p188} Furthermore, sugar binding usually involves a combination of hydrogen bonds between hydroxyl groups and charged side chains (D or E) and nonpolar stacking of aromatic moieties, residues important for the recognition and selectivity of sugar enantiomers by lectins.\cite{Sharon:2001p215}. Consistent with these observations, our results indicate that inositol have the highest binding propensities to F and E (Figure 6D-E). Moreover, from our simulations, the free energy of the binding mode with phenyl ring of F is approximately -0.5 kcal/mol, and is in agreement with that of glucose binding to the indole group of tryptophan obtained from a recent MD simulation\cite{Wohlert:2010p201} and NMR studies\cite{Kiehna:2007p163}. Finally, the shallow amphiphilic grooves found at the surface of $\beta$-oligomers is strikingly analogously to binding sites located at the surface of carbohydrate binding domains.\cite{Taroni:2000p195,Kulharia:2009p212,Weis:1996p225}  Taken together, our above results suggest that inositol may bind in carbohydrate-like binding sites on $\beta$-sheet surfaces involving A$\beta$(16-22).


\subsection{Proposed mechanism of A$\beta$ amyloid inhibition by inositol}

Mature amyloid fibrils are thought to form either by $\beta$-strand addition along the long axis of the fiber (elongation) or by lateral face-to-face association with other protofibrils.\cite{Straub:2011p174} It follows then that small molecules that can disrupt either of these interactions may inhibit fibrillation. In particular for A$\beta$42, multiple experimental studies have shown that A$\beta$(16-22) is part of the $\beta$-sheet core of fibrils of A$\beta$42\cite{Hilbich:1992vy,Gordon:2001tj,Soto:2007tm,Watanabe:2002ti,Inouye:1993ku} and is suggested to be involved in the interface which mediates the stacking of constituent protofilaments of mature fibrils.\cite{Petkova:2002p192,Petkova:2006p48,Paravastu:2006p218,Luhrs:2005p229} Consistent with these observations, the fibrillar structure of full length A$\beta$42 from a solid-state NMR study show that the protofibril has two different $\beta$-sheet faces, one of which is formed by the A$\beta$(17-22) peptide segment.\cite{Luhrs:2005p229}

% Note need to tie together the last three pieces together a bit more. The drive home point is that the affinity + cooperativity + binding in grooves to glutamate & F => binding is exactly carbohydrate like, and this is the most likely binding site for inositol.
% Stereochemistry modulate binding specificities of sugar molecules - need examples

Our results suggest that A$\beta$(16-22) is a likely binding site for inositol in the full length A$\beta$42.  We hypothesize that \emph{scyllo}-, but not \emph{chiro}-inositol, binds at the fibrillar core of A$\beta$42 because of its higher binding specificity to nonpolar groups (particular to aromatic residues) on $\beta$-sheet surfaces involving A$\beta$(16-22). Hence, we propose that \emph{scyllo}-inositol, by binding to and coating the $\beta$-sheet surfaces of protofibrils involving A$\beta$(16-22), disrupts the lateral stacking of these oligomers, which ultimately leads to the inhibition of fibril formation.

Furthermore, on the basis of our results, we hypothesize that planar nonpolar faces, with multiple hydroxyl groups in equatorial positions around the ring confer binding specificity to small molecules for the amyloidogenic core of A$\beta$, and thus are key features for their activity. Consistent with this hypothesis, in vitro studies on small molecule derivatives of \emph{scyllo}-inositol showed that the substitution of a single hydroxyl by a ketone group resulted in the loss of activity (ie. fibrils were formed).\cite{Nitz:2008p13,Sun:2008p208,McLaurin:2000p64} Furthermore, polyphenols, a class of small molecules many of which are strong in vitro inhibitors of amyloid formation, all possess planar nonpolar faces with hydroxyl groups arranged equatorially. On the basis of their structure-activity relationships, a similar hypothesis was recently put forth as a possible explanation for their effectiveness in inhibiting amyloid formation. \cite{Porat:2006p33}

There are many differences between the $\beta$-oligomer of A$\beta$(16-22) and protofibrils of  the full-length A$\beta$ peptides. Our results indicate that inositol binding depends on both the morphology, ie. fibrillar versus non-fibrillar, and the surface physiochemical properties of the aggregate in consideration. Thus, alternative binding modes and binding sites of inositol may exist on aggregate forms of the full length A$\beta$42 peptide, which cannot be ruled out from the results of this study. As part of our future directions, we will perform systematic MD simulation studies of \emph{scyllo}- and \emph{chiro}-inositol and comparatively examine the binding mechanisms specific to aggregates of the full length A$\beta$ to rationally design more efficacious therapeutics for the treatment of AD.


\section{Conclusions} % (fold)
\label{sec:conclusions}

In this study, we have examined the binding of a small molecule inhibitor \emph{scyllo}-inositol, and its inactive stereoisomer \emph{chiro}-inositol to monomers, disordered and $\beta$-sheet aggregates of A$\beta$(16-22), the peptide thought to be the core aggregation region in the A$\beta$42 peptide. Notably, the K$_{d}$ of inositol ($\sim$ 1-5 mM) with $\beta$-oligomer is commensurate with the concentration at which inhibition of amyloid formation by A$\beta$42 is observed in vitro. Although both \emph{scyllo}- and \emph{chiro}-inositol, have similar binding affinities with all peptide states considered, we have uncovered a stereospecific face-to-face stacking stacking mode of \emph{scyllo}- with the phenylalanine side chains, which suggests a molecular basis for measured differences in activity.  Cooperative binding modes of inositol at grooves on the surface of the $\beta$-oligomer of A$\beta$(16-22) suggest a possible mechanism of fibril inhibition where by inositol prevents the lateral association of protofibrillar $\beta$-sheet oligomers. Furthermore, our results suggest that the fibril core of A$\beta$ amyloid aggregates contains carbohydrate-like binding sites. As such, carbohydrate-based (e.g. mono-, di- or trisaccharride) small molecules derivatives may be a promising avenue to explore for the rational design of novel therapeutics for AD.

% section conclusions (end)
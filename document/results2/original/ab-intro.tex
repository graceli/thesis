\section{Introduction}

One in eight people over the age of 65 has Alzheimer's Disease (AD), a progressive neurodegenerative disease that currently has no cure.\cite{Citron:2010p214} The amyloid cascade hypothesis states that the extracellular neuronal deposition of A$\beta$ amyloid plaque plays a central role in the pathogensis of AD.\cite{Solomon:2010p177} A$\beta$ is a peptide proteolytically cleaved from the amyloid precursor protein (APP) and is produced as two common alloforms, A$\beta$40 or A$\beta$42, which are 40 and 42 residues in lengths, respectively. In the diseased state, A$\beta$42 levels are elevated, and the peptides deposit as extracellular A$\beta$ plaques.\cite{Haass:2007p226,Citron:1997p228}

A$\beta$40 and A$\beta$42 are intrinsically disordered peptides that self-aggregate \emph{in vitro} to form amyloid fibrils. Amyloid fibrils are protein aggregates with a characteristic cross-$\beta$ structure, which consists of in-register $\beta$-sheets with backbone hydrogen bonds running parallel to the long axis of the fibril.\cite{Petkova:2002p192} Moreover, smaller fragments of the full length A$\beta$ sequence are also found to form amyloid in vitro.\cite{Balbach:2000p49,Sawaya:2007p11} In particular, one of the shortest amyloid-forming peptides structurally characterized using solid-state NMR is KLVFFAE or A$\beta$(16-22).\cite{Balbach:2000p49} The residues LVFFA are believed to form the central hydrophobic core critical for the initiation of aggregation and fibril formation in the full length A$\beta$ peptide.\cite{Wood:1995p190} Furthermore, single-point mutations in this region greatly affect the aggregation propensity of A$\beta$: known familial mutations E22Q, E22K, and E22G, known as ``Dutch'', ``Italian'' and ``Arctic'' mutations, respectively, significantly accelerate fibril formation,\cite{Kim:2008ef} whereas the mutation F19T abolishes the formation of fibrils \emph{in vitro}.\cite{Esler:1996p288}

% check the different mutations, might be missing one
% TODO use more connectives to draw the links between the concepts introduced

Amyloid fibril formation follows a complex pathway: Prior to the appearance of fibrils in vitro, amyloidogenic monomers self-aggregate into a variety of pre-fibrillar intermediate morphologies. While the fibril is an important state implicated in AD, recent research has shown that soluble oligomers as small as dimers and tetramers play a role in neurotoxicity.\cite{Bernstein:2009p165} In recent years, drug development and research efforts have been directed towards the development of therapeutic agents to prevent the self-aggregation and amyloid formation of A$\beta$, a promising treatment approach to target the underlying disease.\cite{Masters:2006p183,Citron:2010p214,Dasilva:2010p25} As a result, many different types of \emph{in vitro} amyloid inhibitors have been discovered, including peptide molecules,\cite{EsterasChopo:2008p219,Sciarretta:2006p181,Chalifour:2003p161,Scrocchi:2002p178} immunotherapies,\cite{Janus:2000p198,Solomon:2010p177} polyphenolic molecules,\cite{Masuda:2009p205,Berhanu:2010p230,Ehrnhoefer:2008p8} and other small molecules.\cite{Hawkes:2009p189,Masuda:2009p205,Necula:2007p227,Nitz:2008p13} These approaches have been reviewed in detail elsewhere.\cite{Citron:2010p214,Dasilva:2010p25} 
% Despite experimental progress made in recent years, still lack a mechanistic understanding how these small molecules work. These approaches have been reviewed in detail elsewhere.\cite{Citron:2010p214,Dasilva:2010p25}

\emph{scyllo}-Inositol is a small-molecule A$\beta$ amyloid inhibitor developed for the treatment of AD.\cite{Dasilva:2010p25,Hawkes:2009p189,McLaurin:2000p64,Nitz:2008p13,Sun:2008p208} Inositol is a class of polyols, of which eight out of nine stereoisomers are commonly found in nature (Fig. 1). \emph{Myo}-inositol, the most common isomer, is found at high concentrations in the tissues of the human central nervous system (CNS).\cite{Fisher:2002p62} Like \emph{myo}-inositol, \emph{scyllo}-inositol is also present in the brain and can be passively and actively transported across the blood-brain barrier.\cite{Fenili:2007p182} Importantly, \emph{scyllo}-inositol was demonstrated to prevent and reverse AD-like symptoms in a transgenic mouse model of AD.\cite{McLaurin:2006p29} Because of the positive CNS bioavailability and favorable \emph{in vivo} toxicity profile of inositol, both of which are rare and essential properties of putative AD drug candidates, inositol-based therapies represent unique and promising approach for the treatment of AD. In 2007, \emph{scyllo}-inositol (ELN0005) was fast-tracked by the United States Food and Drug Administration into phase II of clinical trials in North America.

\emph{In vitro}, inositol displays stereochemistry-dependent inhibition of A$\beta$42 fibrils: \emph{myo}-, \emph{epi}- and \emph{scyllo}-inositol were shown to inhibit A$\beta$42 fibrillation at concentrations of 1 - 5 mM,\cite{McLaurin:2000p64} whereas \emph{chiro}-inositol is inactive below molar concentrations.\cite{Janus:2000p198} Moreover, upon incubation of monomeric A$\beta$42 with \emph{scyllo}-inositol, circular dichroism spectroscopy indicated the formation of $\beta$-sheet structure at an inositol:peptide molar ratio of 25:1.\cite{McLaurin:1998p176} Although inositol stereoisomers have been proposed to inhibit amyloid formation by directly interacting with either monomers or non-fibrillar aggregates to ``cap off'' fibril growth,\cite{Janus:2000p198} the molecular basis of the effect of \emph{scyllo}-inositol and its stereoisomers on A$\beta$ amyloid formation is currently not understood.

Thus far, experimental efforts to characterize the molecular structure of non-fibrillar oligomers are impeded because of their transient and disordered nature. In turn, the lack of information on the molecular structure of amyloid oligomers hampers experimental determination of the modes of action of inositol. Molecular dynamics (MD) simulations, by contrast, are well-suited for studies of disordered proteins and can provide atomic-level insight into the mechanism of peptide self-aggregation.\cite{Nikolic:2011p185,Rauscher:2006p43,Li:2012p853,Rauscher:2010p5682}

MD simulations were previously employed to examine the binding mechanism of other small molecules inhibitors such as polyphenols,\cite{Lemkul:2010p23,Wang:2010p204} non-steroidal anti-inflammatory drugs\cite{Raman:2009p47,Takeda:2010p34}, and the well-known amyloid dye thioflavin T\cite{Wu:2008ds,Wu:2011fd} to monomers\cite{Liu:2009p213}and/or fibrillar aggregates of A$\beta$. Because of the existence of multiple aggregation states, small molecule inhibitors may have multiple modes of action and can act by either binding to monomers\cite{Ehrnhoefer:2008fd} non-fibrillar or fibrillar oligomers\cite{Buell:2010p9457} in the fibrillation pathway. Furthermore, their inhibitory activity may be also affected by both concentrations of the ligand and ligand:peptide molar ratios. For example, small molecules (-)-epigallochatechin gallate (EGCG)\cite{Wang:2010p204} and ibuprofen\cite{LeVine:2005cv} have been shown to have activities modulated by ligand:peptide molar ratio.  However, thus far, few MD simulation studies have comparatively examined the effect of ligand concentration on different relevant aggregation states along the amyloid fibrillation pathway. 

% INTRO - rationale - develop
In our previous study on inositol,\cite{Li:2012p853} we systematically examined the role of backbone binding in amyloid inhibition by inositol stereoisomers, at low molar ratios, with model peptides alanine dipeptide and (GA)$_4$, a $\beta$-sheet forming peptide. Weak binding, with dissociation constants commensurate with those of osmolytes, were found for inositol with all peptides and aggregates considered, indicating that backbone binding alone is likely to be insufficient for amyloid inhibition. However, we have uncovered stereochemistry-dependent binding modes with nonpolar groups on surfaces of (GA)$_4$ fibril-like aggregates, which suggests that both aggregate morphology and sequence-specific interactions may play an important role in A$\beta$ amyloid inhibition by inositol.  

In this paper, we elucidate the role of sequence-specific interactions of inositol by examining its binding to A$\beta$(16-22), an amyloidogenic peptide that is part of the central hydrophobic core of fibrillar A$\beta$42, in three aggregation states, monomer, disordered oligomer and protofibrillar-like aggregates ($\beta$-oligomers).  Using a systematic approach, comparative MD simulation studies of each of the aforementioned states were successively carried out successively in the presence and absence of \emph{scyllo}- and its inactive stereoisomer, \emph{chiro}-inositol.  Moreover, we examine the differential effects of varying inositol:peptide molar ratios on the binding equilibria of inositol and morphologies of monomers and aggregates of A$\beta$(16-22). From our microsecond time scale simulations, we compute binding constants (K$_{d}$) and successively characterize binding modes of inositol with the peptide aggregation states considered. The results of our study have implications for both the mechanism of amyloid inhibition by small molecules and the rational design of more efficacious putative therapeutics for AD and related amyloid disorders.

% TODO should add a sentence here on why we choose the peptide KLVFFAE.  Better yet, rearrange the first reference to KLVFFAE to here and only talk about KLVFFAE in the context of full length fibril formation. ie. it is the CHC of Abeta.

% CHANGED Have I clearly explained what was done?
% Need to convey: (1) Why we picked these different states (better explain hypothesis/rationale) (2) That we copmute binding affinity
% Need to disentangle this sentence a bit more.
% Why three different types of aggregates
% Why different molar ratios, concentrations
% Binding affinities to gauge activity
% See previous writings on for a good wording so Im not rewriting the wheel again


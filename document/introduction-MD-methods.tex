\chapter{Methods}
% This isn't all methods.  The last section is a review of MD simulations of small molecule binding to amyloids.
% But I'm hesitant to call this MD simulations chapter because it isn't all about MD simulations ... or is it?

\section{Molecular Dynamics Simulations}

MD simulations employ an empirical mathematical function to describe the atomic interactions in a molecular system, and together with classical laws of Newtonian mechanics, atomic trajectories of motion are generated. 

Thermodynamic and kinetic properties can then be extracted as time averages from these trajectories and used to make a number of predictions that are often experimentally challenging to observe or measure.

MD simulation studies have been useful in studying many existing fundamental problems of biology and biochemistry, including protein dynamics and function, protein folding, biomolecular self-aggregation, and protein-ligand binding.

% Transition?

Under the Born-Oppenheimer approximation, electronic and nuclear motions are uncoupled, and can therefore be treated separately. In MD atomic nuclei are treated as classical particles, where the movement of electrons are not accounted for. Instead, the effects of electronic motion is implicitly accounted for via the use of potential energy functions.

The force on each atom in the simulation is the spatial derivative of the potential energy of the system,
\begin{equation}
F = - \nabla U
\end{equation}
where $U$ is the potential energy of the system. In MD simulations, $U$ is estimated using a molecular mechanics force field, a set of parameters which describes the interaction energies of the atoms in the system.

The acceleration $a$ of each atom is then given by Newton's second law of motion,
\begin{equation}
a = \frac{F}{m}
\end{equation}
where $m$ is the mass forces on each atom.

Dynamics of the next position of the atom at time $ t + \delta t$ is then 
\begin{equation}
x_i(t + \delta t) = x_i(t) + v_i(t)\delta t + \frac{a_i(t)\delta t^2}{2}	
\end{equation}

Once all of the new positions for all atoms are predicted, updated interatomic forces based on the coordinates in the newly generated frame are calculated, and the entire process is repeated.

% How much detail? Talk about the velocity verlet integration algorithm ?
% Marty's thesis included a little bit on the verlet algorithm.  CN did not.

For numerical stability reasons, a small integration step or timestep is used in the calculations. Typically 2 femtosecond timestep, twice the time period for the fastest vibrational motion (for bonds involving hydrogen), is used in MD simulations of biomolecular systems.

% [Ref: Chris Madill's and Tom's thesis]

Application of an empirical force field is used in MD to approximate atomic interactions in the system. A force field typically has many parameters which need to be calculated. One approach to do this is to fit to quantum mechanical calculations.  Often, force fields are iteratively improved by predicting experimentally observable quantities for small compounds, and adjusting the fit based on comparisons of these computationally predicted quantities with experimental measurements.
  
There many different force fields (AMBER, Gromos etc), each differing slightly in the potential energy function and its parameterization. In this thesis, we have chosen to use the OPLS-AA/L force field for all of our simulations.
% Note that if I say this .. I might have to justifiy this in my defense.

The general form of the force field potential energy function is 
  % \[ V(R) = bonds + angles + impropers + dihedrals + pair interactions \]
  \begin{equation}
    \begin{split}
          E = \sum_{bonds} k_b(b-b_0)^2 
          + \sum_{angles} k_{\theta}(\theta - \theta_{0})^2 \\
          + \sum_{dihedrals} k_{\chi}(1 + cos(n\chi - \delta)) 
          + \sum_{impropers} k_{\gamma}(\phi - \phi_{0})^2 \\
          + \sum_{nonbonded} \frac{q_1q_2}{er} \\
          + \sum_{nonbonded} \epsilon [(\frac{r_{min}}{r})^{12} - 2(\frac{r_{min}}{r})^6]
    \end{split}
  \end{equation}
\end{outline}

% Initial conditions -- initial positions of the atoms in the system
% Inclusion of temperature in MD -- allows for the account of entropy in our systems.  This is especially important for the determination of free energies, G.
% Note that because I talk a lot about binding ... this discussion within my methods section may be quite important for examing members that understands the physics but do not understand how MD simulations in particular is rigoroous enough to make any of those predictions.  Having those details here means that I understand how that link is established and that I'm not just making up stories based on pretty pictures.

% \subsection{Setting up a MD simulation: practical aspects}
% Here are some details to run a MD simulation of a biomolecular system.  I think I could omit this from my introduction as this is not really essential in understanding the rest of my thesis, or is it? Perhaps this should go into an appendix instead.

The following steps are often used to setup and start a MD simulation system of a protein. First, a pre-determined structure, typically a coordinate structure from X-ray crystallography or NMR, or homology-modelling data. Then a force field and solvent is chosen.

\section{Challenges and limitations of MD}

\begin{outline}
	\1 MD is computationally challenging because of limitations in length and time scales.
	
		\2 Length scale. Large systems are too complex to obtain statistics and quantitative predictions.
			
		% Evaluating the convergence of simulations is still a challenge
		\2 Time scale. Relevant biochemical reactions such as protein folding happens on the time scale of milliseconds, hours, and days. Currently with MD simulations, we are routinely able to approach the microsecond time scale, massive computing power is still insufficient to observe phenomena on the millisecond timescale. % Ref: DE Shaw Research.
		\2 Obtaining convergence. Explain why this is difficult, in particular for systems with disordered peptides.
	\1 Limitations in the accuracy of current force fields.
\end{outline}		

\section{Application of MD in structure-based drug discovery}
\begin{outline}
% \1 (Why computational?) Can help us get protein dynamics is important for understanding protein function. We want to understand protein function because we want to be able to design drugs to cure diseases.

% \1 A important application of MD simulation in biochemistry is the predicting of protein-ligand binding free energies.
	\1 A broad application of simulations of proteins is to computationally design drugs and combining that with structure-based drug design. In recent years structure-based computer modeling of protein-ligand interactions have become a core component of modern drug discovery. 

	\1 Current drug discovery platform. Typically, the first step in drug discovery is to identifying a target, a putative binding site.  Then, solve the X-ray crystal structure of the target.
  % [See Tom's thesis]
	
	\1 Ligands which may act as potential drugs typically have high binding affinity to the binding site. The goal is to find high specificity inhibitor of a protein (usually an enzyme). The binding free energy is an important quantity which can be used to evaluate how well a ligand binds. One method of estimating binding affinity is by using computational docking methods, where the binding affinity is typically estimated without taking into account of protein dynamics.  Although docking is computationally less expensive, it is inaccurate for identifying true drug candidates because binding often involves crucial changes in protein conformation.

	\1 With computer hardware becoming faster and cheaper, MD simulation and modeling can be used to rapidly prototype experimental ideas -- for example, one can perform computational alchemy, that is, ``mutate'' residues to test various hypotheses. Furthermore, simulations may be used to determine whether a chemical change will produce a more potent drug candidate.
	
	\1 Currently state of the art computational binding studies take into the account of change in protein conformation. MD simulations is an effective method, where the protein and drug is allowed to relax and freely move about in the system.

	\1 However, in the case of understanding a specific binding reaction (eg. when developing an enzyme inhibitor), the ability to observe the relevant binding events is a low probability event on the timescale achievable in our simulations. Therefore, it is  impractical in this case to solely apply brute-force sampling techniques to determine binding free energies.

	  \2 Methods used to determine binding free energies using simulations:
	  % \3 Linear interaction energy -- Out of scope
	  % \3 MM/PBSA - no explicit account for solvents -- Out of scope
	  	\3 Thermodynamic perturbation \cite{Gilson:2007hz}
			\4 thermodynamic integration
			\4 free energy perturbation
\end{outline}

\section{Review of MD studies of amyloid inhibition by small molecules}
% MD studies using brute-force sampling. Aid in medicinal chemistry by making suggestions for how to design new AD drugs.

\begin{outline}
	%  Excerpt from Transfer proposal
	\1 In recent years, molecular dynamics simulations have been intensively used to investigate the molecular basis of the structure and stability of amyloid fibrils. 
	
	\1 MD simulations of Congo red binding have been done with the protofibril-like crystal structure composed of the segment GNNQQNY.\{Wu, 2007 \#621\}
	
	\1 A recent simulation study of an N-methylated peptide with A$\beta$16-22 models of amyloid aggregates has provided insight into the possible mechanism of action of peptide inhibitors of amyloid formation.\{Soto, 2007 \#597\} This peptide inhibitor was shown to preferentially bind monomers to form dimers, possibly acting to inhibit fibril formation by sequestering monomers. However, peptide-based inhibitors have poor pharmacological profiles as they are actively broken down by proteases in the stomach and are difficult to transport across the blood-brain barrier. In addition, these peptide inhibitors specifically target A$\beta$ and thus do not have the potential to treat multiple amyloid diseases.
\end{outline}

\addcontentsline{toc}{section}{Bibliography}
\bibliographystyle{plain}
\bibliography{chapter1}

% Scrap
% Motivate the use of MD simulations
% Describe the details of molecular dynamics simulations
% Review the basic derivations of MD simulation equations and why they work

% Molecular dynamics simulations are a useful tool to study the structure, dynamics, and interaction of biomolecules. 

% MD is a numerical algorithm which solves a system of Newtonian equations of motion, and provides as output the time-trajectory of atoms with femtosecond time resolution.

% To review before my defense

% Details of the mathematics (need to review the basic theory + Taylor series expansion) - get a book - tomorrow maybe?

% Why is MD correct? Describe the fundamental assumptions of MD. Here, I want to give the readers who aren't familiar with the methodological details of MD a sense of the rigorousness of MD.

% The assumption at a hand-wave level adapted from Tom's thesis

% - Relationship between force and energy 
% - Relationship between momentum and velocity 
% - Why numerical approach must be used (no analytical solution for N > 2)
% - How is the force field plugged into the general algorithm.
Alzheimer's disease (AD) is a progressive neurodegenerative disease characterized pathologically by the presence of extracellular fibrillar deposits of $\beta$-amyloid (A$\beta$), a 38 to 42 residue protein. \emph{Scyllo}-inositol is a promising potential therapeutic compound for AD treatment that blocks the accumulation of A$\beta$ oligomeric assemblies and reduce AD-like symptoms in a mouse model of AD. \emph{In vitro}, \emph{scyllo}-inositol and its stereoisomer \emph{chiro}-inositol prevent the formation of A$\beta$(1-42) fibrils in a stereochemistry-dependent manner. 

Understanding the molecular basis of the effect of inositol on A$\beta$ aggregation will aid in the development of inhibitors of amyloid aggregation. A$\beta$ amyloid formation follows a complex aggregation pathway, where intermediate aggregate species are implicated in the disease.  To this end, extensive molecular dynamics simulations of \emph{scyllo}-inositol and \emph{chiro}-inositol were carried out to characterize the binding of inositol with different self-aggregated peptide states: single peptides, small aggregates, and ordered $\beta$-sheet aggregates. 
% By exploiting the weak binding of inositol, I obtained equilibrium binding affinities using bruteforce sampling

First, I characterized the binding equilibria of inositol with model amyloid peptides, alanine dipeptide and (Gly-Ala)$_4$.  Inositol binds predominantly to the backbone with affinities in the millimolar to molar range, indicating that backbone binding is unlikely to inhibit amyloid fibrillation. Next, I characterized the binding mechanism of inositol to monomer and aggregates of A$\beta$(16-22). Ordered $\beta$-sheet-like aggregates of A$\beta$ were identified as binding partners of inositol, where \textit{scyllo-}inositol displays higher binding specificity than \textit{chiro-}inositol for the grooves at the surface of protofibrillar oligomers. Finally, I characterized the binding of inositol to a protofibril of A$\beta$(1-42).  \textit{Scyllo}-inositol displays the highest binding specificity for the residues in the central hydrophobic core of A$\beta$42.  Together, these results suggest a molecular mechanism where \textit{scyllo-}inositol inhibits A$\beta$ amyloid formation by coating the surface of protofibrillar aggregates and disrupting their lateral stacking into fibrils.

In Chapter 6, extensive simulations of PgaB, a key protein in the export of polysaccharides in the formation of bacterial biofilm, were conducted successively in the presence of two monosaccharides that are components of the polymeric substrate, PNAG. Understanding the molecular basis of PgaB-PNAG binding is required for the rational design of inhibitors of biofilm production by bacteria. Here, the putative binding modes and sites of PNAG predicted, in combination with experimental studies, led to a mechanism for the export of PNAG by PgaB. Significantly, the results here demonstrate that the central methodology of this thesis is broadly applicable for characterizing protein-carbohydrate binding.
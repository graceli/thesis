Alzheimer's disease (AD) is a progressive neurodegenerative disease characterized pathologically by the presence of extracellular fibrillar deposits of $\beta$-amyloid (A$\beta$), a 38 to 42 residue protein that is produced normally as part of the cellular metabolism. \emph{Scyllo}-inositol is a promising potential therapeutic compound for AD treatment, which has been shown to effectively block the accumulation of A$\beta$ oligomeric assemblies and reduce AD-like symptoms in a mouse model of AD. Furthermore, \emph{in vitro}, \emph{scyllo}-inositol and its stereoisomer \emph{chiro}-inositol, have been demonstrated to prevent the formation of A$\beta$42 fibrils in a stereochemistry-dependent manner. However, the specific molecular interactions of inositol and its effect on A$\beta$ aggregation at the molecular level are not known. In order to elucidate the molecular basis of the structure-activity relationship of inositol, I performed extensive molecular dynamics simulations to determine its effect on the structure and thermodynamics of the amyloid aggregation pathway.  Simulations of \textit{scyllo-} and \textit{chiro-}inositol with (1) single peptides; (2) small aggregates; (3) large ordered aggregates of various amyloidogenic peptides were performed. In addition, control simulations in the absence of inositol were also performed.

First, the binding equilibria of inositol was characterized with model amyloidogenic peptides.  Inositol binds weakly with milimolar to molar affinities, suggesting that exclusively binding to the backbone is unlikely to lead to the inhibition of amyloid fibrillation. Next, I examined binding of inositol to peptide and aggregates of A$\beta$(16-22), a fibril-forming fragment important for initiating amyloid formation in the full-length peptide (Chapter 4).  Based on predicted binding affinities and modes, ordered $\beta$-sheet-like aggregates of A$\beta$ were identified as putative binding partners. Furthermore, \textit{scyllo-}inositol displays higher binding specificity than \textit{chiro-}inositol for the grooves at the surface of protofibrillar oligomers.

In the final part of my inositol studies, I characterized the binding of inositol stereoisomers with a solid-state NMR model of the protofibril of A$\beta$42. Due its stereochemistry, \textit{scyllo}-inositol displays the highest binding specificity for the residues in the central hydrophobic core of A$\beta$42, consistent with our previous study in Chapter 4. Taken together, our above results suggest that \textit{scyllo-}inositol inhibits amyloid formation by coating the surface of protofibrillar aggregates of A$\beta$ and disrupting their lateral stacking into fibrils.

The generality of my central methodological framework was demonstrated in chapter 6. Simulations of PgaB, a key protein in the export of polysaccharides in the formation of bacterial biofilm, were conducted successively in the presence of two monosaccharides, which are components of the polymeric substrate, PNAG. Understanding the molecular basis of PgaB-PNAG binding is required for the rational design of inhibitors of biofilm production by bacteria. Putative binding modes for PNAG were predicted on the basis of my results, which in combination with experimental studies, led to a mechanism for the export of PNAG by PgaB.

In summation, my work provides insight into developing novel, higher-efficacy derivatives with the ability to effectively prevent the onset and progression of AD.  Moreover, the central methodology of this thesis is broadly applicable for characterizing protein-carbohydrate binding in general.




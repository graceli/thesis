MD is a molecular mechanics simulation technique.

% I like these analogies and I think it will help make my methods sections less dry
In many respects, Molecular Dynamics simulations are very similar to real experiments. When we perform a real experiment, we first prepare a sample of the material that we wish to study. This sample is then connected to a measuring instrument from which we can measure the property of interest during a certain time interval. If our measurements are subject to statistical noise (as most measurements are), then the longer we average, the more accurate our measurement becomes. 

The same approach is applied when performing a Molecular Dynamics simulation. First, a model system consisting of N particles is selected and we record the dynamics of this system (via solving Newton's equation of motion) until the properties of the system no longer change with time (we equilibrate the system).  After equilibration, we perform the actual measurement. In fact, some of the most common mistakes that can be made when performing a computer experiment are very similar to the mistakes that can be made in real experiments (e.g., the sample is not prepared correctly, the measurement is too short, the system undergoes an irreversible change during the experiment, or we do not measure what we think). (Copied and adapted from Molecular Simulations book)

MD simulations employ an empirical mathematical function to describe the atomic interactions in a molecular system, and together with classical laws of Newtonian mechanics, to simulate systems undergoing thermal motion from which atomic trajectories are generated. 

Thermodynamic and kinetic properties can then be extracted as time averages from these trajectories and used to make a number of predictions that are often experimentally challenging to observe or measure.

MD simulations have been used in recent years to gain insight into many fundamental problems in biology and biochemistry, including protein dynamics and function, protein folding, self-aggregation, and protein-ligand binding. This classical mechanics approach has been shown to be able to provide a valid approximation of many biological systems thus far. REFs

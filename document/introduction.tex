\chapter{Introduction}

% Organizational detail: AD and amyloid (which one should go first?). Center it on Amyloid, and then bring in AD as a central motivation for doing amyloid science.

% I think my focus should still be on Alzheimer's disease ... the practical purpose of my work is to understand how inositol works
% I am not trying to cure all amyloid diseases.
\section{Amyloid}
Many diseases, most of them neurodegenerative, have been linked with the presence of amyloid: Prion-related diseases, Parkinson's, Huntington's disease, Type II diabetes etc.

\subsection{Involvement in disease}
	% Importance of understanding amyloid.  Role of amyloid in the human body.

\subsubsection{Alzheimer's Disease}
% Here, my intention is to lead into the discussion of amyloid by giving a historical perspective and overview of AD
% And use AD as a motivation for why so much work has gone into studying amyloids.
\begin{outline}[enumerate]
	\1 More than a hundred years have pass since Dr. Alois Alzheimer first presented the connection between the presence of neuronal plaques and the clinical symptoms of presenile dementia characteristic of Alzheimer's disease (AD).

	\1 Today, AD is known to be the most common cause of dementia in persons of age 65 or older. With the increasing longevity of our population, AD is approaching epidemic proportions with no cure or preventative therapy available.\cite{Blennow:2006wd}

	% Elaborate on happened between the step above and the amyloid hypothesis?
	\1 The discovery of amyloid plaque deposits in brains of deceased dementia patients led to the formulation of the amyloid hypothesis, which posits that amyloid aggregates initiates the pathogenesis of AD and that the other pathological symptoms such as neurofibrillary tangles are secondary.

% \2 \abeta is produced by intramembrane proteolytic cleavage of the larger amyloid-$\beta$ precursor protein (APP) by $\beta$-secretase, and is produced constitutively as part of the normal cellular metabolism.\{Selkoe, 2002 \#222\} Depending on the position of the cleavage, \abeta peptides of lengths varying from 38 to 43 residues can be produced. However, the peptides spanning residues 1--40 (\abeta40) or 1--42 (\abeta42) are predominantly found AD-associated plaques.
\end{outline}

\subsubsection{Toxicity}
% I think outline some of the ideas / hypothesis about the link between amyloid and disease, but don't go into what people speculate or data on toxicity. It is related, but this is out of the scope of your thesis.
 \begin{outline}
 	% Key question in the field: What is the toxic species?
 	\1 Multiple lines of research have identified oligomers as a likely causative agent for neuronal cell death. By contrast, the monomeric and fibril forms are thought to be less toxic than oligomers. It is hypothesized that soluble oligomers may cause toxicity by perturbing the integrity of cellular membranes through binding and disrupting the lipid bilayer (perhaps by making them ion permeable). \cite{Walsh:2007fu}
 	\1 Include a paragraph about amyloid formation and lipid membranes (?)
 	% Understanding the toxicity or finding out whether there is a toxic species in part validates the amyloid hypothesis. 
 \end{outline}

\subsection{Structure and Formation}
  % In this section I will talk about how amyloid aggregation is thought to work. Introduce the thermodynamic model for understanding fibril formation. I will now broadly introduce to amyloid.  
	Finding a treatment for AD and other fatal neurodegenerative diseases motivated many biochemical and biophysical studies of the amyloid state.  

  \begin{outline}[enumerate]
	\1 In vitro, a variety of proteins and peptides, folded or intrinsically disordered, have been shown to be able to aggregate to form amyloid fibrils under certain solution conditions. 

	\1 Amyloid fibrils are formed via a complex aggregation pathway. Initially, monomers aggregate to form oligomers with different morphologies which exists in equilibrium with amyloid fibrils. Some of these oligomers are on-pathway to fibril formation, while others themselves may be end-points of the aggregation pathway. Biochemically, fibrils are protease resistant and are insoluble in the presence of SDS.

   \1 Currently, it is thought that amyloid fibrillar state may be the globally stable folded state for all proteins.
  \end{outline}

 \subsubsection{Fibrils}
    \subsubsubsection{Structure} % (fold)
    \label{ssub:structure}
    % Add details on the definition of the cross-$\beta$ structure and its significance. 

Early X-ray diffraction studies show that fibrils have a regular structure, which is defined by a characteristic 4.8\angstrom\ interpeptide, and 10\angstrom\ intersheet spacing. Biophysicists have adopted this as the definition of the cross-$\beta$ amyloid fiber structure. (Figure~\ref{fig:fibril_diffraction})
\begin{figure}
  \centering
  \includegraphics[width=6in]{figures/introduction/fibril_structure_diffraction.pdf}
  \caption[Characteristic cross-$\beta$ spacings from X-ray fibre diffraction studies of amyloid fibrils]{This is adapted from Eisenberg, 2012}
  \label{fig:fibril_diffraction}
\end{figure}

After staining amyloid fibers are visible as long unbranched fibrils under the transmission Electron microscope (TEM). Figure~\ref{fig:fibril_TEM_SSNMR}
\begin{figure}
  \centering
  \includegraphics[width=6in]{figures/introduction/fibril_TEM_SSNMR.pdf}
  \caption[Characteristic cross-$\beta$ spacings from X-ray fibre diffraction studies of amyloid fibrils]{A Example EM images of oligomers.  Adapted from Bitan G. et al. 2003 and Walsh D. 1999 C TEM image of fibrils D SSNMR model proposed by Tycko et al.  }
  \label{fig:fibril_TEM_SSNMR}
\end{figure}
% Describe the molecular structure of \abeta\ amyloid fibrils. 
% Briefly mention the techniques that can be used to obtain structural information of amyloid fibrils.

% Since the discovery of the cross-$\beta$ several biophysical techniques have been applied to uncover the molecular structure of fibrils. 

In recent years, advances in SSNMR and X-ray crystallography have recently made major contributions to our knowledge of the structures of amyloid fibrils.

% SSNMR
Both A$\beta$40 and A$\beta$42 amyloid fibrils have been studied in detail using solid-state NMR. Figure~\ref{fig:fibril_TEM_SSNMR}
% \begin{figure}
%   \centering
%   \includegraphics[width=6in]{figures/introduction/fibril_structure_NMR.jpg}
%   \caption[Blah]{NMR model of the core \abeta\ amyloid fibril consistent with MPL from EM, STEM and the cross-$\beta$ structure}
%   \label{fig:fibril_SSNMR_model}
% \end{figure}

% X-ray structures
Furthermore, small peptide fragments that have characteristics of amyloid fibrils, which are also amenable to single crystal X-ray diffraction analysis have demonstrated similar type structures from those studied using SSNMR.  These structures obtained by X-ray crystallography have been described to have a dry interface with stacked sheets. (Figure~\ref{fig:fibril_xray_model})

\begin{figure}
  \centering
  \includegraphics[width=6in]{figures/introduction/fibril_xray_model.pdf}
  \caption[Characteristic cross-$\beta$ spacings from X-ray fibre diffraction studies of amyloid fibrils]{This is adapted from Eisenberg, 2012}
  \label{fig:fibril_xray_model}
\end{figure}


% Fibrils all have the cross beta structure in common
The ubiq- uitous presence of a cross-β structure strongly supports the view that the physicochemical properties of the polypeptide chain are the major determinants of the fibrillar structure in each case. Moreover, several of the proposed structures, despite very different sequences of their component polypeptides, suggest that the core region is composed of two to four sheets that interact closely with each other. An interesting feature of these sheets is that they appear to be much less twisted than ex- pected from the analysis of the short arrays of β-strands that form β-sheets in globular protein structures. This feature was first pro- posed from cryo-EM and has been supported by Fourier transform infrared (FTIR) analy- ses (48, 61).

% But fibrils vary in the length of the beta-strand involved, side chain orientation

% Even fibrils formed from a single peptide can exhibit polymorphism due to the different experimental conditions under which they are formed.
Amyloid Fibrils have been found to exhibit polymorphism at the molecular level, but all have similar ultrastructures.  % (Figure~\ref{fig:fibril_diffraction})

		% \1 Decades after the initial discovery by Alois Alzheimer, \abeta, the central protein component of neuronal plaques, was synthetically produced in the laboratory. In vitro, \abeta\ was found to precipitate out of solution almost immediately. 

  \subsubsection{Structure of Non-fibrillar oligomers}
   Due to their structural disorder and their insolubility, structural determination of oligomers have posed challenging experimentally.
	
  \subsubsection{Kinetics of aggregation}
  Amyloid fibrils have been observed to form via a two-step nucleation-polymerization process. In the nucleation phase, energetic barriers of aggregation must be overcome to form the initial aggregation nucleus or seed.  Following nucleation, free monomers bind to the nucleated aggregates and polymerize into mature fibrils.\cite{Murphy:2002fe}


\section{Amyloid Inhibition: A promising treatment for amyloid disorders}
% Cure, method of prevention; is there hope?
\begin{outline}
	\1 In this section, I will provide an overview of some of the challenges to overcome when developing a small molecule therapeutic for Alzheimer's disease.  Furthermore, using this information, I will motivate why inositol is an exciting avenue to explore.
	
	  \2 scyllo-Inositol is able to cross the blood brain barrier. It has high bioavailability. Because it is not broken down in the gut, it can be taken orally.
	  
	  \2 Inositol is not toxic to the human body.  Inositol is used in signaling pathways.

	\1 Briefly mention non-small molecule putative therapies which also acts via amyloid inhibition. The focus of this thesis will be on small-molecule amyloid inhibition.
\end{outline}

\subsection{Molecular mechanisms of amyloid inhibition 
            \\ by small molecules}
\begin{outline}[enumerate]
    \1 Amyloid inhibition as a treatment for Alzheimer's disease and related amyloid disorders. Amyloids are attractive drug targets. Small molecules may be one effective way to develop a treatment for amyloid disorders because they have the potential to be able to treat the underlying disease. Through in vitro screening, many small molecules have been found to effect the amyloid aggregation pathway.  Some were demonstrated to inhibit amyloid fibrils, where as others were shown to arrest or reduce oligomer formation.   
      % Here I can take a cue from Justin Lemkul`'s recent review paper.
      % Talk about the different kinds of small molecules that have been found to inhibition amyloid formation.  Here I will also provide a summary of what people know about the mechanism by which they inhibit amyloid formation.
      
      \2 Pharmacological perspective of the challenge of developing an Alzheimer's drug. In order to effectively treat Alzheimer's and other neurodegenerative diseases, small molecule drug candidates must pass the blood brain barrier at sufficient concentrations for inhibition.  This is difficult to achieve.
      
      \2 In vitro screening has led to the discovery of a large number of small-molecules which were found to affect the amyloid aggregation pathway. Many of these small molecules are thought to act by directly binding to amyloidogenic peptides and aggregates.

      \begin{figure}
        \centering
        \includegraphics[width=3in]{figures/introduction/dyes.png}
        \caption[Small molecule binders]{Amyloid binding dyes Congo Red and Thioflavin T}
        \label{fig:amyloid_dyes}
      \end{figure}
            
        \3 Thioflavin T and Congo red are dye molecules used to identify the presence of amyloid.  Both molecules bind to mature amyloid fibrils and have been shown to affect fibril formation.(Fig.~\ref{fig:amyloid_dyes})

        \begin{figure}
          \centering
          \includegraphics[width=6in]{figures/introduction/polyphenols.png}
          \caption[Small molecule binders]{Polyphenols}
          \label{fig:polyphenols}
        \end{figure}
        
        \3 Polyphenols,  is a large group of natural and synthetic molecules.  (−)-epigallocatechin-3-gallate, curcumin, and a polyphenolic grape seed extract, known for their anti-oxidant properties,  were recently discovered to be capable of affecting amyloid formation.(Fig.~\ref{fig:polyphenols})
      
      \2 Small molecule inhibitors share common chemical features and groups.  They are typically planar in geometry, have many aromatic rings, and polar functional groups (hydroxyl groups) around the edge of these aromatic rings.
    
    	\2 Mechanism of action. Some small molecules inhibit fibril formation, where as others may prevent oligomerization, but not fibrillation. A high concentration is often required to observe activity (micromolar to millimolar), which suggests that they may be non-specific inhibitors. EGCG, one such polyphenol, is known to have the lowest IC50.
    	% IC50 -- This quantitative measure indicates how much of a particular drug or other substance (inhibitor) is needed to inhibit a given biological process (or component of a process, i.e. an enzyme, cell, cell receptor or microorganism) by half.
      % EC50 -- The term half maximal effective concentration (EC50) refers to the concentration of a drug, antibody or toxicant which induces a response halfway between the baseline and maximum after some specified exposure time.[1] It is commonly used as a measure of drug's potency.
      % Ref: wikipedia
      
      % Review of what is known about amyloid fibril ligand binding, specifically dyes.	
	    \3 Molecular mechanism of binding of dye molecules. Thought to bind flat on on the surface grooves of amyloid fibrils where they interact with hydrophobic groups exposed at the surface. 
      % Doesn't explain why the dye molecules are also able to suppress fibril formation.
      % Can the birefringence be explained by these binding modes? -- this is out of the scope of my thesis.  Don't put this in my thesis but I should be able to coherently explain this during my defense.
      
	\subsection{Inositol molecules}
	
	\begin{figure}
    \centering
    \includegraphics[width=6in]{figures/introduction/inositol.png}
    \caption[Inositol]{Inositol stereoisomers}
    \label{fig:inositols}
  \end{figure}
  
		\2 Inositol stereoisomers.(Fig.~\ref{fig:inositols}) Role of inositol in the human body.

			\3 \emph{myo}-inositol.  Use the physiological role of myo-inositol as a lead to transition into the 

		\2 Where is inositol found. Present in human body tissues. Myo- is present in certain grains, grape fruit, but scyllo- is only found in small quantities in food sources.
	
		\2 Role of inositol in amyloid inhibition. Include the background on how inositol was discovered as an \abeta\ amyloid fibril inhibitor.
		
			\3 In vitro and In vivo studies (mouse)
			
			\3 Include some data on human clinical trials (?)
			
\end{outline}

\section{Analogy to Sugar-protein binding}
% Does this section fit here? Where should I put it?
\subsection{Experimental techniques to study sugar-binding modes}
\subsection{Sugar Binding modes}

% This section provides a nice lead in to the methods section
\section{Protein-ligand interactions}
\subsection{Forces involved in binding}
% Note that I may end up introducing the forces up in the earlier section -- reorganize as needed
\begin{outline}
	\1 Protein-ligand non-covalent interactions that are important for ligand binding and recognition
		\2 Electrostatic interactions. Polar (hydrogen bonding) and charge-charge interactions
		 % Here, it will benefit me to read Sarah's appendix C carefully.
		\2 Nonpolar (hydrophobic) interactions
		  \3 Van der Waals
\end{outline}

\subsection{Binding equilibria}
% subsection protein_ligand_binding_theory (end)
% Below is a summary of an excerpt from Tom's thesis on structure-based drug discovery.
% Design of antibiotics 
% 1) Target determination (biochemical)
% 2) Structural determination (Xray, NMR, or homology); active site identified; Here would be useful to get the holo structure of the protein
% 3) Screen for inhibitors against a chemical library or in silico docking.
\begin{outline}
	\1 Enzyme and its putative ligand typically bind specifically (high affinity binding).  We want to optimize binding specificity to increase the efficacy of the putative drug, and decrease adverse side effects (toxicity) in the human body.

	\1 The dissociation constant, $K_d$, is a measure of the affinity of a ligand for its binding site on the host protein. Pharmacologically, it can be interpreted as the concentration at which 50\% of the drug is bound to the protein. In experimental studies, $K_d$ is often used to quantitatively screen for potential drug candidates. 
  % A small $K_d$ suggests that the ligand may bind tightly to the protein.

	\1 Binding equilibrium

    \begin{equation}
      \left[ Protein\cdot Inositol \right] 
      \rightleftharpoons 
      \left[ Protein \right]+\left[ Inositol \right]
    \end{equation}
  
    % \2 Absolute binding free energy
    % \2 Relative binding free energy
    
	\1 The binding free energy of a ligand to a protein is directly related to its dissociation constant, $K_d$, the equilibrium constant of the above reaction

     \begin{equation}
        K_{d} = f_{ub}\frac{\left[ Protein \right]\left[ Inositol \right]}{\left[Protein \cdot Inositol\right]},
     \end{equation}
     
     % Add equation converting binding constant to gibbs free energies.
	\1 Experimental techniques for estimating $K_d$
		\2 What experimental techniques are used to estimate binding affinity? (May need to study up on this)
		\2 Isothermal titration calorimetry (ITC) is a technique which can be used to measure energetics of ligand binding to peptides.
\end{outline}

% \subsection{Role of chirality in drug binding}
% Stereoisomerism is important to the activity of molecules.  It modulates binding to proteins.
% Two types of stereochemistry
% Constitutional isomers - differs in bonding sequences and connectivity
% Stereoisomers - differs in orientation of their atoms in space, but no connectivity differences.
% Definition of chirality [Add schematic] ... etc
% Molecules with chirality have a non-superimposable mirror image, called an enantiomer.
% A carbon molecule with four different groups has chirality.

\section{Thesis objectives and rationale}
% Understanding amyloid inhibition in the context of the framework of traditional enzyme inhibition mechanism
\subsection{Challenges of amyloid inhibition}
\begin{outline}
       % However, most of these studies were focused on A$\beta$ and large A$\beta$ aggregates,\{Fawzi, 2008 \#553;Esposito, 2008 \#567;Sgourakis, 2007 \#609;Wei, 2006 \#656;Tarus, 2006 \#628 Karsai, 2006 \#658\} and thus, were computationally limited by the complexity of the molecular systems.

    \1 The protein-ligand binding model developed to understand enzyme inhibition cannot be directly applied to understand the molecular mechanism of amyloid inhibition by small molecules. 
    
      \2 Amyloid inhibitors are found to be very weak binders. How do non-specific inhibitors act as a drug? And how do we approach this with MD simulations?
      
      \2  Because the A$\beta$ amyloid aggregate pathway encompasses a variety of species, some of which has no folded structure, a single conformation cannot be assumed for binding. Furthermore, structural information of amyloidogenic species lags behind those of enzymes, which tends to be globular proteins amenable for X-ray crystallography. This means that the putative binding sites are not known.
      
      \2 The structural disorder of the peptides involved poses a challenge for obtaining converged properties from MD simulations. 
    
    	\1 A$\beta$ peptides are completely disordered.  We also do not know what the binding site looks like, where it is located on these structures.
    	
    \1 To date, few studies have attempted to provide statistically meaningful results pertaining to general mechanisms of protein self-aggregation and amyloid formation. Furthermore, despite the abundance of MD studies of A$\beta$, few studies have systematically examined the mechanism of action of small molecule inhibitors of amyloids

    \1 In AD, there is the added challenge of the drug being able to cross the brain barrier, while remaining non-neurotoxic.  What kind of drugs cross the BBB?  Typically hydrophobic drugs.
\end{outline}    

\subsection{Study Design and Rationale}
\begin{outline}
	\1 Here describe in detail how I designed my study to circumvent the challenges presented by the amyloid inhibition problem, and the limitations  of MD simulations. At this point, clearly explain and discuss my study design and rationale. (Fig.~\ref{fig:rationale})

  \begin{figure}
    \centering
    \includegraphics[width=6in]{figures/introduction/matrix.pdf}
    \caption[Rationale]{Shows the progression from small, model systems to larger and structurally more complex systems involving the full-length A$\beta$42 peptide.}
    \label{fig:rationale}
  \end{figure}

	\1 Beginning with the simplest model systems for an amyloidogenic peptide, the alanine dipeptide, we systematically examine binding of inositol with systems of both increasing sequence and structural complexity.

	% Use brute force simulations
	\1 We exploit conventional MD simulation techniques because simulation approaches used for understanding enzyme-ligand binding is not applicable. 
	
	\1 Instead, we use conventional MD simulations and repeats of independent simulations to determine the binding modes, and binding equilibria of inositol with amyloidogenic peptides and aggregates of A$\beta$.
\end{outline}

\section{Thesis Organization}
The first chapter introduces the thesis in the context of the field.  The second chapter introduces the main methods used in the work in the thesis. Chapters 3, 4, and 5 are the results of simulations of inositol with amyloid like peptides and aggregates. Chapter 6 shows work of the general applicability of our methods developed throughout this thesis to MD simulations to understand protein - carbohydrate binding. Chapter 7 provides discussion, suggestions for future work, and perspectives.

\addcontentsline{toc}{section}{Bibliography}
\bibliographystyle{plain}
\bibliography{chapter1}
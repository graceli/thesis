Because numerous diseases share the amyloid plaque pathology, fibrils were initially hypothesized to be the toxic species in these diseases.\cite{Hardy:2002dh} However, recent research has implicated non-fibrillar oligomers as the more likely toxic agent in the cause of several neurodegenerative diseases (such as Parkinson's, Alzheimer's and Huntington's diseases) and type II diabetes (see Table~\ref{tbl:amyloid_peptides}).\cite{Haass:2007db,Xue:2009da,Berthelot:2013fs} % (Baglioni 2006)
Currently, the mechanism of toxicity of amyloid oligomers has not been determined, and is an area under intensive research. Oligomers formed from a variety of peptides, including those not implicated in amyloid disorders (e.g. lysozyme, $\beta$2-microglobulin, transthyretin) all exhibited toxicity, suggesting that the toxicity of amyloid oligomers may be independent of the peptide sequence.\cite{Fandrich:2012kb,Kayed:2003en} Experimental evidence widely supports the hypothesis that amyloid toxicity is based on a generic mechanism that involves the interactions of oligomers with cellular membranes.\cite{Martins:2008bz,Walsh:2007fu} Specifically, it is hypothesized that oligomeric aggregates may ultimately induce cell death by interacting with and disrupting the integrity of the cellular membrane.\cite{Fandrich:2012kb} Moreover, the aggregation of amyloidogenic peptides was found to occur more rapidly in the presence of membrane surfaces,\cite{McLaurin:1997wm,Kayed:2004ul,Yip:2002vx} leading to the claim that membrane-catalyzed fibril formation may induce cellular toxicity.\cite{Yip:2001tl}

% Some studies have suggested that A$beta$ oligomers can increase membrane conductance without the formation of channels.
% Exam question: How does inositol eliminate toxicity of oligomers?

% Table of amyloid diseases 
\begin{table}\footnotesize\centering
    % \begin{center}
    \vspace{10pt}
    \caption{Amyloid-forming peptides and proteins and their associated diseases}
    \label{tbl:amyloid_peptides}
      \begin{tabular}{| c | c |}
        \hline
        Disease & Peptide \\
        \hline
        \hline
	Alzheimer's & A$\beta$40 and A$\beta$42 \\   
        \hline
	Parkinson's & $\alpha$-synuclein \\
        \hline
        Huntington's & poly-glutamine \\
        \hline
        Amyotrophic lateral sclerosis & Superoxide dismutase 1 \\
        \hline
        Spongiform encephalopathies & Prion protein or fragments thereof \\
        \hline
        Type II Diabetes & islet amyloid polypeptide (or Amylin) \\
        \hline
      \end{tabular}
    % \end{center}
    \label{tbl:amyloid_peptides}
  \end{table}

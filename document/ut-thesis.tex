%%%%%%%%%%%%%%%%%%%%%%%%%%%%%%%%%%%%%%%%%%%%%%%%%%%%%%%%%%%%%%%%%%%%%%
%%  
%%  UT-THESIS.TEX
%%
%% This program can be redistributed and/or modified under the terms
%% of the LaTeX Project Public License Distributed from CTAN archives
%% in directory CTAN:/macros/latex/base/lppl.txt.
%%
%%  Copyright (c) 1999 by Francois Pitt
%%  Last Update: 1999 May 13
%%  
%%%%%%%%%%%%%%%%%%%%%%%%%%%%%%%%%%%%%%%%%%%%%%%%%%%%%%%%%%%%%%%%%%%%%%
%%  
%%  This file is distributed in the hope that it will be useful but
%%  without any warranty (without even the implied warranty of
%%  fitness for a particular purpose).  For a description of this
%%  file's purpose, and instructions on its use, see below.
%%  
%%  Feel free to copy and redistribute this file, as long as this
%%  copyright notice remains intact and this file is distributed
%%  along with the companion file `ut-thesis.cls'.
%%  
%%  (Thanks to Robert Bernecky for his suggestions on improving the
%%  usefulness and readability of this file.)
%%  
%%  Send all bugs, questions, comments, suggestions, etc. to the
%%  author, at <fpitt@cs.utoronto.ca>.
%%  
%%%%%%%%%%%%%%%%%%%%%%%%%%%%%%%%%%%%%%%%%%%%%%%%%%%%%%%%%%%%%%%%%%%%%%
%%  
%%  Skeleton LaTeX2e file for the preparation of theses at UofT;
%%  conforms to the School of Graduate Studies' guidelines of 07/97.
%%  To be used in conjunction with class file `ut-thesis.cls', whose
%%  features it illustrates.
%%  
%%  To comment out parts of a file, use the macro \ignore{...}
%%  around the entire block of text you want to ignore.
%%  
%%  To explicitly set the pagestyle of any inserted blank page when
%%  \cleardoublepage occurs, use one of \clearemptydoublepage or
%%  \clearplaindoublepage instead.
%%  
%%  For single-spaced quotes or quotations, use the `longquote' and
%%  `longquotation' environments.  For single-spaced, 1 1/2-spaced,
%%  or double-spaced paragraphs, use one of the environments
%%  `singlespaced', `oneandahalfspaced', or `doublespaced'.  More
%%  generally, for paragraphs with a line spacing of `n', use
%%  `\begin{newspacing}{n}...\end{newspacing}'.
%%  
%%  All other environments, commands, and options provided by the
%%  `ut-thesis' class will be described below, at the point where
%%  they should appear in the document.
%%  
%%  See the companion file `ut-thesis.cls' for more details.
%%  
%%%%%%%%%%%%%%%%%%%%%%%%%%%%%%%%%%%%%%%%%%%%%%%%%%%%%%%%%%%%%%%%%%%%%%


%%%%%%%%%%%%         PREAMBLE         %%%%%%%%%%%%

%% Default settings format a final copy (12pt font, single-sided,
%% double-spaced, normal margins, single-spaced notes).  For a rough
%% copy (10pt font, double-sided, double-spaced, normal margins, with
%% the word "DRAFT" printed at each corner of every page), use the
%% `draft' option.  The default line spacing can be changed with one
%% of the following options: `singlespaced', `oneandahalfspaced', or
%% `doublespaced'.  The notes are always single-spaced by default, but
%% can be made to have the same spacing as the rest of the document by
%% using the option `spacednotes'.  The size of the margins can be
%% changed with one of the following options: `narrowmargins' (1 1/4"
%% left, 3/4" others), `normalmargins' (1 1/4" left, 1" others),
%% `widemargins' (1 1/4" all), `extrawidemargins' (1 1/2" all).  Any
%% other standard option for the `report' document class can be used
%% to override the default or draft settings.

%% ***   Add any desired options.   ***
\documentclass{ut-thesis}

%% ***   Add \usepackage declarations here.   ***


%% The line spacing of the document should be specified using one of
%% the document options given above, but if you need a line spacing
%% that is not provided by the options, you can override the default
%% line spacing for the entire document with the command
%%   `\linespacing{...}'.
%% Note that in order to get the correct appearance, the argument to
%% `\linespacing' must be equal to 1/3 + 2/3 times the desired line
%% spacing (for example, single-spaced = \linespacing{1},
%%                        1 1/2-spaced = \linespacing{1.33}, and
%%                       double-spaced = \linespacing{1.66}).

%% ***   Uncomment and fill in a value, if needed.    ***
%% ***   REMEMBER: You should NOT need to use this.  Use one of   ***
%% ***   the document class options mentionned above instead.     ***
%\linespacing{}

%%%%%%%%%%%%%%%%%%%%%%%%%%%%%%%%%%%%%%%%%%%%%%%%%%%%%%%%%%%%%%%%%%%%%%
%%                                                                  %%
%%                  ***   I M P O R T A N T   ***                   %%
%%                                                                  %%
%%  Fill in the following fields with the required information:     %%
%%   - \degree{...}       name of the degree obtained               %%
%%   - \department{...}   name of the graduate department           %%
%%   - \gradyear{...}     year of graduation                        %%
%%   - \author{...}       name of the author                        %%
%%   - \title{...}        title of the thesis                       %%
%%%%%%%%%%%%%%%%%%%%%%%%%%%%%%%%%%%%%%%%%%%%%%%%%%%%%%%%%%%%%%%%%%%%%%

%% ***   Change this example to appropriate values.   ***
\degree{Doctor of Philosophy}
\department{Biochemistry}
\gradyear{2012}
\author{Grace Li}
\title{Molecular mechanism of amyloid inhibition and protein denaturation}

%% ***   NOTE   ***
%% Put here all other formatting commands that belong in the preamble.


%% For example, to list only down to subsections in table of contents
%% (-1=part, 0=chapter, 1=section, 2=subsection, 3=subsubsection,
%%  4=paragraph, 5=subparagraph, 6=subsubparagraph).
%
\setcounter{tocdepth}{2}


%%%%%%%%%%%%      MAIN  DOCUMENT      %%%%%%%%%%%%

\begin{document}

%% ***   NOTE   ***
%% You should put all of your `\newcommand', `\newenvironment', and
%% `\newtheorem's (in other words, all the global definitions that
%% you will need throughout your thesis) in a separate file and use
%% "\input{filename}" to input it here.


%% This sets the page style and numbering for preliminary sections.
\begin{preliminary}

%% This generates the title page from the information given above.
\maketitle

%% There should be NOTHING between the title page and abstract.

%% This generates the abstract page, with the line spacing adjusted
%% according to SGS guidelines.
\begin{abstract}
%% ***   Put your Abstract here.   ***
%% (At most 150 words for M.Sc. or 350 words for Ph.D.)
% TODO: This is my Biophysical society abstract.  MUST REWORD and readapt
Alzheimer's disease (AD) is a severe neurodegenerative disease with no cure. Currently, one method of targeting the underlying disease is to prevent or reverse the amyloid formation of A$\beta$42, a key pathological hallmark of AD. A small-molecule novel drug candidate, Scyllo-inositol, is a polyol small-molecule that exhibits stereochemistry dependent inhibition of the formation of fibrils in vitro.  Furthermore, recently completed phase II clinical trials demonstrated that scyllo-inositol achieved target drug levels in the cerebral spinal fluid (CSF) of Alzheimer’s patients, a main challenge for AD drug candidates to to overcome. 
Despite its promise as a therapeutic for AD, the mechanism of action of scyllo-inositol at the molecular level is currently not understood.  We perform extensive microsecond timescale atomistic explicit solvent molecular dynamics simulations of scyllo-inositol and its inactive stereoisomer, chiro-inositol with the full length A$\beta$42 protofibril.  From our simulations, we predict binding affinities and characterize the binding modes of inositol and their stereochemistry-dependent effect on the structure of A$\beta$42 protofibrils.  Our results provide molecular insight for the rational design of small-molecule inhibitors of A$\beta$42 and other amyloid-based diseases.
\end{abstract}

%% Anything placed between the abstract and table of contents will
%% appear on a separate page since the abstract ends with \newpage
%% and the table of contents starts with \clearpage.

%% This generates a "dedication" section, if needed.
%% (uncomment to have it appear in the document)
%\begin{dedication}
%% ***   Put your Dedication here.   ***
%\end{dedication}

%% The `dedication' and `acknowledgements' sections do not create new
%% pages so if you want the two sections to appear on separate pages,
%% you should put an explicit \newpage between them.

%% This generates an "acknowledgements" section, if needed.
%% (uncomment to have it appear in the document)
\begin{acknowledgements}
% ***   Put your Acknowledgements here.   ***
First and foremost, I am grateful to my supervisor, Regis Pomes for his wisdom, insight and guidance for the past 6 years.

Secondly, I thank my entire lab, past and present, for their moral support and help with my work. Nilu Chakrabarti, David Caplan, John Holyoake, Loan Huynh, Kethika K. Chris Madill, Chris Neale,  Tom Rodinger ...

Finally, I thank my family, mom, grandparents for their support over the years.
\end{acknowledgements}  

%% This generates the Table of Contents (on a separate page).
\tableofcontents

%% This generates the List of Tables (on a separate page), if needed.
%% (uncomment to have it appear in the document)
%\listoftables

%% This generates the List of Figures (on a separate page), if needed.
%% (uncomment to have it appear in the document)
%\listoffigures

%% End of the preliminary sections: reset page style and numbering.
\end{preliminary}

%%%%%%%%%%%%%%%%%%%%%%%%%%%%%%%%%%%%%%%%%%%%%%%%%%%%%%%%%%%%%%%%%%%%%%
%%  Put your Chapters here; the easiest way to do this is to keep   %%
%%  each chapter in a separate file and `\include' all the files    %%
%%  right here.  Note that each chapter file should start with the  %%
%%  line "\chapter{ChapterName}".  Note that using `\include'       %%
%%  instead of `\input' makes each chapter start on a new page.     %%
%%%%%%%%%%%%%%%%%%%%%%%%%%%%%%%%%%%%%%%%%%%%%%%%%%%%%%%%%%%%%%%%%%%%%%

%% ***   Include chapter files here.   ***
\chapter{Introduction}

Lorem ipsum dolor sit amet, consectetur adipiscing elit. Proin tellus nunc, accumsan sit amet blandit nec, accumsan a libero. Praesent blandit erat ut tellus congue eu facilisis leo malesuada. Nullam sit amet fermentum erat. Nunc mi leo, rutrum in tincidunt sit amet, imperdiet a nisl. Donec pulvinar, risus vel hendrerit elementum, lacus eros fringilla est, sed condimentum massa libero ac velit. Donec non sapien nunc, eget pellentesque lectus. Maecenas est quam, convallis non sagittis at, accumsan id risus. Duis gravida faucibus nisi auctor rutrum. Vestibulum rutrum massa in orci iaculis a condimentum nisl accumsan. Donec placerat, turpis et rutrum facilisis, nisl turpis condimentum nibh, a adipiscing massa elit non lorem. Quisque et nisl vel lectus tincidunt pulvinar quis eget arcu. Donec erat neque, malesuada sed euismod vel, tempus quis purus. Mauris dictum mauris ante. Class aptent taciti sociosqu ad litora torquent per conubia nostra, per inceptos himenaeos.

Pellentesque vitae nulla a nisi tincidunt scelerisque et in justo. Aenean et urna sed enim luctus interdum sed et magna. Phasellus eget orci ac nulla consectetur porta a eget felis. Aenean egestas lorem vel nulla suscipit mollis. Cras interdum, elit at aliquam commodo, purus risus aliquam mi, sit amet tempus neque mi a dui. Nulla ante est, porta eget adipiscing sit amet, mollis a purus. Etiam sit amet ultrices nisi. Lorem ipsum dolor sit amet, consectetur adipiscing elit. Nunc ultricies odio non dolor condimentum viverra.

Maecenas ac lectus sed velit convallis auctor. Nullam ultrices augue ut arcu rhoncus pellentesque rutrum lectus blandit. Maecenas lorem elit, interdum eget egestas ac, ultrices a enim. Sed sodales nunc a lectus interdum non faucibus nunc ornare. Curabitur varius porttitor quam, congue elementum urna tristique in. In et magna libero, at pretium augue. Nunc scelerisque egestas justo non mollis.

Fusce in augue lacus, vitae rhoncus nisl. Nulla mattis imperdiet luctus. Praesent est magna, eleifend nec accumsan vel, sodales in leo. Maecenas augue urna, aliquam a iaculis eu, tincidunt ac dolor. Integer vitae augue tellus, in porta diam. Cras sit amet condimentum justo. Etiam mollis leo vel tortor dignissim ac varius est rutrum. In quis libero purus, quis sagittis risus. Cras imperdiet dapibus posuere. Nulla gravida molestie ligula eget mollis. Vivamus facilisis viverra leo, in tincidunt urna imperdiet et. Aenean turpis eros, pharetra eget faucibus at, eleifend eget neque. Duis augue nisi, hendrerit in malesuada nec, aliquet quis dolor. Donec ut condimentum est. Nunc ac lorem at diam dictum sodales ac vitae diam.

Quisque sollicitudin nisl ac turpis consequat ut ultrices massa adipiscing. Ut tempor ligula ut est elementum et elementum nulla imperdiet. Aliquam id gravida ligula. Praesent vel nibh felis. Aliquam mollis nisl in nibh dictum at mollis turpis tristique. Curabitur vitae risus sapien, et fermentum lectus. Mauris eleifend ornare neque sit amet condimentum. Proin turpis nisi, mollis eu pretium ac, dictum et eros. Proin eu lorem quis leo cursus pretium. Praesent nunc metus, euismod at lacinia id, pharetra adipiscing dui. In hac habitasse platea dictumst.

Lorem ipsum dolor sit amet, consectetur adipiscing elit. Proin tellus nunc, accumsan sit amet blandit nec, accumsan a libero. Praesent blandit erat ut tellus congue eu facilisis leo malesuada. Nullam sit amet fermentum erat. Nunc mi leo, rutrum in tincidunt sit amet, imperdiet a nisl. Donec pulvinar, risus vel hendrerit elementum, lacus eros fringilla est, sed condimentum massa libero ac velit. Donec non sapien nunc, eget pellentesque lectus. Maecenas est quam, convallis non sagittis at, accumsan id risus. Duis gravida faucibus nisi auctor rutrum. Vestibulum rutrum massa in orci iaculis a condimentum nisl accumsan. Donec placerat, turpis et rutrum facilisis, nisl turpis condimentum nibh, a adipiscing massa elit non lorem. Quisque et nisl vel lectus tincidunt pulvinar quis eget arcu. Donec erat neque, malesuada sed euismod vel, tempus quis purus. Mauris dictum mauris ante. Class aptent taciti sociosqu ad litora torquent per conubia nostra, per inceptos himenaeos.

Pellentesque vitae nulla a nisi tincidunt scelerisque et in justo. Aenean et urna sed enim luctus interdum sed et magna. Phasellus eget orci ac nulla consectetur porta a eget felis. Aenean egestas lorem vel nulla suscipit mollis. Cras interdum, elit at aliquam commodo, purus risus aliquam mi, sit amet tempus neque mi a dui. Nulla ante est, porta eget adipiscing sit amet, mollis a purus. Etiam sit amet ultrices nisi. Lorem ipsum dolor sit amet, consectetur adipiscing elit. Nunc ultricies odio non dolor condimentum viverra.

Maecenas ac lectus sed velit convallis auctor. Nullam ultrices augue ut arcu rhoncus pellentesque rutrum lectus blandit. Maecenas lorem elit, interdum eget egestas ac, ultrices a enim. Sed sodales nunc a lectus interdum non faucibus nunc ornare. Curabitur varius porttitor quam, congue elementum urna tristique in. In et magna libero, at pretium augue. Nunc scelerisque egestas justo non mollis.

Fusce in augue lacus, vitae rhoncus nisl. Nulla mattis imperdiet luctus. Praesent est magna, eleifend nec accumsan vel, sodales in leo. Maecenas augue urna, aliquam a iaculis eu, tincidunt ac dolor. Integer vitae augue tellus, in porta diam. Cras sit amet condimentum justo. Etiam mollis leo vel tortor dignissim ac varius est rutrum. In quis libero purus, quis sagittis risus. Cras imperdiet dapibus posuere. Nulla gravida molestie ligula eget mollis. Vivamus facilisis viverra leo, in tincidunt urna imperdiet et. Aenean turpis eros, pharetra eget faucibus at, eleifend eget neque. Duis augue nisi, hendrerit in malesuada nec, aliquet quis dolor. Donec ut condimentum est. Nunc ac lorem at diam dictum sodales ac vitae diam.

Quisque sollicitudin nisl ac turpis consequat ut ultrices massa adipiscing. Ut tempor ligula ut est elementum et elementum nulla imperdiet. Aliquam id gravida ligula. Praesent vel nibh felis. Aliquam mollis nisl in nibh dictum at mollis turpis tristique. Curabitur vitae risus sapien, et fermentum lectus. Mauris eleifend ornare neque sit amet condimentum. Proin turpis nisi, mollis eu pretium ac, dictum et eros. Proin eu lorem quis leo cursus pretium. Praesent nunc metus, euismod at lacinia id, pharetra adipiscing dui. In hac habitasse platea dictumst.

% \chapter{Chapter 2}

Lorem ipsum dolor sit amet, consectetur adipiscing elit. Proin tellus nunc, accumsan sit amet blandit nec, accumsan a libero. Praesent blandit erat ut tellus congue eu facilisis leo malesuada. Nullam sit amet fermentum erat. Nunc mi leo, rutrum in tincidunt sit amet, imperdiet a nisl. Donec pulvinar, risus vel hendrerit elementum, lacus eros fringilla est, sed condimentum massa libero ac velit. Donec non sapien nunc, eget pellentesque lectus. Maecenas est quam, convallis non sagittis at, accumsan id risus. Duis gravida faucibus nisi auctor rutrum. Vestibulum rutrum massa in orci iaculis a condimentum nisl accumsan. Donec placerat, turpis et rutrum facilisis, nisl turpis condimentum nibh, a adipiscing massa elit non lorem. Quisque et nisl vel lectus tincidunt pulvinar quis eget arcu. Donec erat neque, malesuada sed euismod vel, tempus quis purus. Mauris dictum mauris ante. Class aptent taciti sociosqu ad litora torquent per conubia nostra, per inceptos himenaeos.

Pellentesque vitae nulla a nisi tincidunt scelerisque et in justo. Aenean et urna sed enim luctus interdum sed et magna. Phasellus eget orci ac nulla consectetur porta a eget felis. Aenean egestas lorem vel nulla suscipit mollis. Cras interdum, elit at aliquam commodo, purus risus aliquam mi, sit amet tempus neque mi a dui. Nulla ante est, porta eget adipiscing sit amet, mollis a purus. Etiam sit amet ultrices nisi. Lorem ipsum dolor sit amet, consectetur adipiscing elit. Nunc ultricies odio non dolor condimentum viverra.

Maecenas ac lectus sed velit convallis auctor. Nullam ultrices augue ut arcu rhoncus pellentesque rutrum lectus blandit. Maecenas lorem elit, interdum eget egestas ac, ultrices a enim. Sed sodales nunc a lectus interdum non faucibus nunc ornare. Curabitur varius porttitor quam, congue elementum urna tristique in. In et magna libero, at pretium augue. Nunc scelerisque egestas justo non mollis.

Fusce in augue lacus, vitae rhoncus nisl. Nulla mattis imperdiet luctus. Praesent est magna, eleifend nec accumsan vel, sodales in leo. Maecenas augue urna, aliquam a iaculis eu, tincidunt ac dolor. Integer vitae augue tellus, in porta diam. Cras sit amet condimentum justo. Etiam mollis leo vel tortor dignissim ac varius est rutrum. In quis libero purus, quis sagittis risus. Cras imperdiet dapibus posuere. Nulla gravida molestie ligula eget mollis. Vivamus facilisis viverra leo, in tincidunt urna imperdiet et. Aenean turpis eros, pharetra eget faucibus at, eleifend eget neque. Duis augue nisi, hendrerit in malesuada nec, aliquet quis dolor. Donec ut condimentum est. Nunc ac lorem at diam dictum sodales ac vitae diam.

Quisque sollicitudin nisl ac turpis consequat ut ultrices massa adipiscing. Ut tempor ligula ut est elementum et elementum nulla imperdiet. Aliquam id gravida ligula. Praesent vel nibh felis. Aliquam mollis nisl in nibh dictum at mollis turpis tristique. Curabitur vitae risus sapien, et fermentum lectus. Mauris eleifend ornare neque sit amet condimentum. Proin turpis nisi, mollis eu pretium ac, dictum et eros. Proin eu lorem quis leo cursus pretium. Praesent nunc metus, euismod at lacinia id, pharetra adipiscing dui. In hac habitasse platea dictumst.

Lorem ipsum dolor sit amet, consectetur adipiscing elit. Proin tellus nunc, accumsan sit amet blandit nec, accumsan a libero. Praesent blandit erat ut tellus congue eu facilisis leo malesuada. Nullam sit amet fermentum erat. Nunc mi leo, rutrum in tincidunt sit amet, imperdiet a nisl. Donec pulvinar, risus vel hendrerit elementum, lacus eros fringilla est, sed condimentum massa libero ac velit. Donec non sapien nunc, eget pellentesque lectus. Maecenas est quam, convallis non sagittis at, accumsan id risus. Duis gravida faucibus nisi auctor rutrum. Vestibulum rutrum massa in orci iaculis a condimentum nisl accumsan. Donec placerat, turpis et rutrum facilisis, nisl turpis condimentum nibh, a adipiscing massa elit non lorem. Quisque et nisl vel lectus tincidunt pulvinar quis eget arcu. Donec erat neque, malesuada sed euismod vel, tempus quis purus. Mauris dictum mauris ante. Class aptent taciti sociosqu ad litora torquent per conubia nostra, per inceptos himenaeos.

Pellentesque vitae nulla a nisi tincidunt scelerisque et in justo. Aenean et urna sed enim luctus interdum sed et magna. Phasellus eget orci ac nulla consectetur porta a eget felis. Aenean egestas lorem vel nulla suscipit mollis. Cras interdum, elit at aliquam commodo, purus risus aliquam mi, sit amet tempus neque mi a dui. Nulla ante est, porta eget adipiscing sit amet, mollis a purus. Etiam sit amet ultrices nisi. Lorem ipsum dolor sit amet, consectetur adipiscing elit. Nunc ultricies odio non dolor condimentum viverra.

Maecenas ac lectus sed velit convallis auctor. Nullam ultrices augue ut arcu rhoncus pellentesque rutrum lectus blandit. Maecenas lorem elit, interdum eget egestas ac, ultrices a enim. Sed sodales nunc a lectus interdum non faucibus nunc ornare. Curabitur varius porttitor quam, congue elementum urna tristique in. In et magna libero, at pretium augue. Nunc scelerisque egestas justo non mollis.

Fusce in augue lacus, vitae rhoncus nisl. Nulla mattis imperdiet luctus. Praesent est magna, eleifend nec accumsan vel, sodales in leo. Maecenas augue urna, aliquam a iaculis eu, tincidunt ac dolor. Integer vitae augue tellus, in porta diam. Cras sit amet condimentum justo. Etiam mollis leo vel tortor dignissim ac varius est rutrum. In quis libero purus, quis sagittis risus. Cras imperdiet dapibus posuere. Nulla gravida molestie ligula eget mollis. Vivamus facilisis viverra leo, in tincidunt urna imperdiet et. Aenean turpis eros, pharetra eget faucibus at, eleifend eget neque. Duis augue nisi, hendrerit in malesuada nec, aliquet quis dolor. Donec ut condimentum est. Nunc ac lorem at diam dictum sodales ac vitae diam.

Quisque sollicitudin nisl ac turpis consequat ut ultrices massa adipiscing. Ut tempor ligula ut est elementum et elementum nulla imperdiet. Aliquam id gravida ligula. Praesent vel nibh felis. Aliquam mollis nisl in nibh dictum at mollis turpis tristique. Curabitur vitae risus sapien, et fermentum lectus. Mauris eleifend ornare neque sit amet condimentum. Proin turpis nisi, mollis eu pretium ac, dictum et eros. Proin eu lorem quis leo cursus pretium. Praesent nunc metus, euismod at lacinia id, pharetra adipiscing dui. In hac habitasse platea dictumst.


%% This adds a line for the Bibliography in the Table of Contents.
\addcontentsline{toc}{chapter}{Bibliography}
%% ***   Set the bibliography style.   ***
%% (change according to your preference)
\bibliographystyle{plain}
%% ***   Set the bibliography file.   ***
%% ("thesis.bib" by default; change if needed)
\bibliography{thesis}

%% ***   NOTE   ***
%% If you don't use bibliography files, comment out the previous line
%% and use \begin{thebibliography}...\end{thebibliography}.  (In that
%% case, you should probably put the bibliography in a separate file
%% and `\include' or `\input' it here).

\end{document}

%%%%%%%%%%%%%%%%%%%%%%%%%%%%%%%%%%%%%%%%%%%%%%%%%%%%%%%%%%%%%%%%%%%%%%
%%  End of UT-THESIS.TEX
%%%%%%%%%%%%%%%%%%%%%%%%%%%%%%%%%%%%%%%%%%%%%%%%%%%%%%%%%%%%%%%%%%%%%%

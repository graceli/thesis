My project started off because was interest in how inositol drug works as a potential AD therapeutic

Studies of Abeta of phospholipid with inositol headgroups was able 

Hypothesis if we just took the head of the lipids would these molecules also inhibit fibril formation.
They tried with phosphoinositols which didn't work
Then they tried with scyllo epi, myo. Myo worked but it was too weak. Scyllo and epi appeared to have stronger effects. They were able to inhibit 
Epi and scyllo worked well in mice. 


Mark Nitz started to modify scyllo-inositol in an attempt to improve its mechanism of action


In general terms what's teh difference between specific and non-specific binding?
Specific binding 
Non-specific binding - when you have enough of the solute around your protein, you get interaction that isn't neccessarily the type of interaction that explains the activity of the small molecule.

Using molecular simulations, could describe the different ways to look specific binding or non-specific binding.
For specific binding, 

Because inositol binds very weakly, we couldn't approach the problem using the approach typically used to study protein-ligand binding.

Which is easier, determining specific or non-specific binding.  

Is there anything else that you would like to say about why it is particularly difficult for determining free energies of binding and binding modes for non-specific binders?


what kind of results can you get from simulations that you can't get from experiments. In what ways? and Why?

can you speculate why one can't get crystal structures of Abeta?

You said you looked fragments of Abeta. How did you select this fragment?

What is the normal funciton of Abeta


Why is inositol particularly exciting?

Why is myo-inositol interesting? What does it do?
	- Suggestion: Myo-inositol's regular function in the body might be a lead-in to BBB and the excitingness for why scyllo-inositol is a good candidate for an AD drug.
		      Also they feed it to the mice.
Does our body make myo-inositol or is it taken in via nutrition or both?

What's the relationship between phosphotidylinositol 


Have people looked at scyllo-inositol as an imaging agent?
	- scyllo-inositol has a weak binding constant not a good imagining agent.

What is the difference between a sugar monosaccharide and a inositol molecule
	- Stereochemistry is different.  Subtle differences in stereochemistry 
		- Binding of different sugar groups to proteins produced by bacteria
			- Binding by a sugar polymer versus monosaccharride binding
			- tell me about polyvalency and avidity


What are fibrils?
	- Discovery of fibrils
	- Where fibrils are useful, fibrils of amyloid etc (2-3 pages with figures)

Do things other than amyloid form fibrils?
- Actin and this other polymerizing protein.


What is a computer simulation?
	- A computer program that solves a system of equations iteratively
	- takes
What are the limitations of computer simulations?
Computer simulations are limited by length scale, incompleteness of the model
Time scale - taking a very small time step for numerical stability and physical purposes.
You are not approaching timescales at which interesting events are happening in your simulations.

You can get precise answers.  The resolution of at which you can obtain for the given model.  Experiments are often missing data.

Tell me about force fields that are available for sugar, protein, and inositol.
 - how did you obtain parameters for inositol

You said that in the binding modes that the hydrogen bonds are important in the binding mode interaction
Is there anything explicit in the force field that is giving rise to the hydrogen bonds? 
In nature, hydrogen bonds aren't just about the distances between atoms, but also the angle formed between the donors and acceptor.
I know that simulations can often recapitulated in the simulations ... and without an explicit hydrogen bonding term, how simulations could possibly get that right.

Here have a section (could be a big deal) on hydrogen bonds, and hydrophobic interactions (and hydrophobic stacking), london dispersion forces etc


What's chirality?
  - 






			




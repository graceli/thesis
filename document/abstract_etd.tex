% Directly lifted from my Ph.D thesis proposal objectives
% Molecular dynamics simulations can provide the atomistic level of detail necessary to determine the interaction of inositol with specific intermediates in the pathway.  

Alzheimer's Disease (AD) is a progressive neurodegenerative disease characterized pathologically by the presence of extracellular fibrillar deposits of &beta;-amyloid (A&beta;), a 38 to 42 residue protein that is produced normally as part of the cellular metabolism. <italic>scyllo</italic>-Inositol is a promising potential therapeutic compound for AD treatment, which is currently in phase two of clinical trials. Both <italic>scyllo</italic>-inositol and one of its stereoisomers, <italic>epi</italic>-inositol, have been shown to effectively block the accumulation of A&beta; oligomeric assemblies and reduce AD-like symptoms in a transgenic mouse model of AD. Furthermore, emph{in vitro}, <italic>scyllo</italic>-inositol and its stereoisomers <italic>myo</italic>-inositol, <italic>epi</italic>-inositol, and <italic>chiro</italic>-inositol, have been demonstrated to stabilize A&beta;42 oligomers, disassemble preformed A&beta;42 fibrils, and reduce A&beta;42-induced neurotoxicity in a stereochemistry-dependent manner. However, the specific molecular interactions of inositol and its effect on A&beta; aggregation at the molecular level are not known. At present, experimental approaches lack the ability to determine the precise mechanistic mode(s) of action of inositol as the molecular structures of various intermediates in the amyloid aggregation pathway are not known. Moreover, observed intermediates morphologically distinct from fibrils are experimentally very difficult to detect and isolate.

The primary objective of my research is to use molecular dynamics simulations to elucidate the molecular basis of the structure-activity relationship of inositol by determining its effect on the structure, thermodynamics of the amyloid aggregation pathway.  In order to achieve the sampling neccessary to obtain meaningful statistics from my simulations, I perform thousands of independent MD simulations of <italic>scyllo</italic>-inositol and textit{chiro}-inositol with (1) single peptides; (2) small aggregates; (3) large ordered aggregates. In addition, I also perform control simulations in the absence of inositol. 

I first began my studies by characterizing the binding equilibria of inositol with model amyloidogenic peptides.  Inositol was found to interact weakly with milimolar to molar affinities, suggesting that binding to the backbone is unlikely to lead to inhibition of amyloid fibrillation. Next, I characterized the binding of inositol with the peptide KLVFFAE or A&beta;(16-22), a short fibril-forming fragment thought to be important for initiating amyloid formation in the full-length protein.  Based on predicted binding affinities and modes of inositol, I identified putative binding partners of inositol:  inositol acts on ordered aggregates of A&beta; and does not bind the monomer of KLVFFAE, and therefore, not likely to act as a drug on the monomer of A&beta;. Importantly, <italic>scyllo</italic>-inositol displays higher binding specificity than <italic>chiro</italic>-inositol for the grooves at the surface of protofibrillar oligomers. 

In the final part of my inositol studies, I characterized the binding of inositol stereoisomers with a model of the protofibril of A&beta;42. A key finding was that due its stereochemistry, <italic>scyllo</italic>-inositol displays the highest binding specificity for the residues in the central hydrophobic core of A&beta;42. Taken together, our above results suggest that <italic>scyllo</italic>-inositol inhibits amyloid formation by coating the surface of protofibrillar aggregates of A&beta; and disrupting their lateral stacking into fibrils.

Finally, I demonstrate the generality of my central methodological framework. Simulations of PgaB, a key protein in the export of polysaccharides in the formation of bacterial biofilm were conducted successively in the presence of two monosaccharides, which are components of the polymeric substrate, PNAG.   Understanding the molecular basis of PgaB-PNAG binding is needed for the rational design of inhibitors of biofilm production by bacteria.   Putative binding modes for PNAG were predicted on the basis of my results, which in combination with experimental studies, led to a mechanism for the export of PNAG by PgaB.

In summation, my work provides insight into developing novel, higher-efficacy derivatives with the ability to effectively prevent the onset and progression of AD.  Moreover, the central methodology of this thesis is broadly applicable for characterizing protein-carbohydrate binding in general.

% Furthermore, we have predicted equilibrium binding affinities and generated multiple experimentally-testable hypotheses that are currently being tested in the laboratory of Dr. Simon Sharpe at SickKids.





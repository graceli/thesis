%\author{Grace Li}
%\author{Christopher Ing}
% \author{Dustin Little}
% \author{P. Lynne Howell}
% \email{howell@sickkids.ca}
% \affiliation[University of Toronto, Biochemistry]
% {Department of Biochemistry, University of Toronto, 27 King's College Circle, Toronto, Ontario, Canada M5S 1A1}
% \alsoaffiliation[Hospital for Sick Children]
% {Molecular Structure and Function, The Hospital for Sick Children, 555 University Avenue, Toronto, Ontario, Canada M5G 1X8}
% 
% \author{Mark Nitz}
% \email{nitz@chem.utoronto.ca}
% \affiliation[University of Toronto, Chemistry]
% {Department of Chemistry, University of Toronto, 80 St. George Street, Toronto, Ontario, Canada M5S 3H6}

%\author{R\'{e}gis Pom\`{e}s}
%\email{pomes@sickkids.ca}
%\phone{416 813 5686}
%\fax{416 813 5022}
%\affiliation[University of Toronto, Biochemistry]
%{Department of Biochemistry, University of Toronto, 27 King's College Circle, Toronto, Ontario, Canada M5S 1A1}
%\alsoaffiliation[Hospital for Sick Children]
%{Molecular Structure and Function, The Hospital for Sick Children, 555 University Avenue, Toronto, Ontario, Canada M5G 1X8}

\chapter[MD simulations of PgaB-glucosamine binding]{Molecular Dynamics simulations of PgaB and monosaccharides of N-acetyl-glucosamine}
% The contents of this section were adapted from an article published in the \emph{Journal of Physical Chemistry}.

\emph{Reference}: Part of this work is published in ``PgaB Contains a Carbohydrate Binding Domain Required for Modification and Export of Poly-$\beta$-1,6-N-acetyl-D-glucosamine"
\\
\\
\emph{Contributions}:
Grace Li conducted the MD simulation part of the research and wrote the section. Dustin Little conducted and interpreted the experimental results. Chris Ing parameterized the partial charges for glucosamine, a key part of the MD simulation study. Rgis Poms, Lynne Howell, Mark Nitz provided editorial input and guidance.

\newpage

\section{Summary}
% Note that this summary was taken from a draft of the paper written by Dustin from April 5th, 2013
%Bacteria embedded in a self-produced matrix of exopolymeric substance, or biofilm, are tolerant to antibiotics, protected from the environment, and isolated from the innate immune system. Production and de-N-acetylation of the exopolysaccharide poly-$\beta$-1,6-N-acetyl-D-glucosamine (PNAG) is important for biofilm formation in Escherichia coli. PgaB is essential for the partial de-N-acetylation of PNAG (dPNAG); a process required for polymer export and subsequent biofilm formation. Here we report 1.9 Å crystal structures of PgaB’s isolated C-terminal domain (PgaB-CT) and complex with glucosamine. The structure of PgaB-CT has structural difference from the previously reported PgaB42-655 structure (\textbf{What are these differences?}). Characterization of PgaB-CT using tryptophan fluorescence quenching assays shows binding to PNAG oligomers with ~1-4 mM affinity. These data in combination with molecular dynamics simulations of PgaB with N-acetylglucosamine and glucosamine suggest PNAG de-N-acetylation occurs first, with subsequent binding of dPNAG to the C-terminal domain. We believe this concerted action plays a pivotal role in targeting dPNAG for export through the outer membrane porin PgaA.
%
%From the book chapter XXX, PNAG is a homopolymer of beta(1,6) glcnac but with 20\% of the monomers deacetylated.  But then what would be the difference between dPNAG and PNAG? Or is it just a terminology difference between the two?

\section{Introduction}
% \textbf{In the second paper, experiments were performed to characterize the ability of PgaB-CT domain to bind carbohydrate substrates, which are short oligosaccharides of \pnag. The results of this study suggests a mechanism by which polymer substrates are bound and synthesized onto the protein.  Furthermore, a proposed mechanism of export is discussed.}

% \textbf{commentary: From reading what Dustin has written in the current introduction, I think their experimental data is quite weak.  Their key result is:  ``We show PgaB-CT binds PNAG oligomers with low affinity (what's the range?) using tryptophan fluorescence quenching, and have determined the structure of PgaB-CT in complex with glucosamine.'' I think MD simulations will be their best leverage for the proposed mechanism of export.}

% PgaB - description of the study, motivation and what has been done thus far
PgaB is a key protein that is responsible for the transport of the functionally relevant form of the exopolysaccharride poly-$\beta$-1,6-$N$-acetylglucosamine (PNAG), which is required for biofilm formation in a variety of bacterial systems. A crystal structure of PgaB was recently solved in the laboratory of Dr. Lynne Howell.\cite{Little:2012dp} Structural and functional characterization have shown that PgaB is composed of two domains, an N-terminal de-N-acetylase domain, and a C-terminal domain with structural homology to glycosyl hydrolases.\cite{Little:2012dp}
% PgaB structure has been deposited in the PDB -- I did not find this June 30th 2012

Currently, it is not known whether PgaB binds polymeric GlcNac or its de-N-acetylated end products, or both. Furthermore, the binding mode and the length of the putative substrate polymer of PgaB are not known. Crystallization of bound structures of PgaB have been impeded by the insolubility of long sugar polymer chains (eg. those longer than a pentamer), and the weak binding of short sugar polymers (eg. di- and tri-saccharides). However, MD simulations are not impeded by these challenges, and are well-suited for examining the weak interactions characterizing carbohydrate-interactions.

In this study, we perform unbiased molecular dynamic simulations of the full-length PgaB with GlcNAc and GlcNH3+, the monosaccharide components of \pnag, to examine the putative binding site and modes of PNAG.  By exploiting the weak binding affinities of N-acetyl-glucosamine (GlcNac) and glucosamine monosaccharides, a fragment-based approach to predict putative binding modes and sites of PNAG with the full-length PgaB protein. This fragment-based methodology of using monosaccharides to probe carbohydrate binding sites has been employed successfully in our previous studies to examine the binding mechanism of inositol, a carbohydrate-like amyloid inhibitor, with amyloidogenic peptides and their aggregates.\cite{Li:2012bx,Li:2013Abreview}

% PgaB is hypothesized to bind sugar polymers (most likely a 15-mer) across its two domains on its surface. The polymer is hypothesized to extend out from the catalytic binding site (N-terminal domain) onto the charged grooves of the C-terminal domain.
% Experimentally it is known that shallow binding pockets exist at the surface of the proteins. REF?
% To our knowledge, this is the only study thus far in literature which employs large-scale MD simulations to predict carbohydrate-protein interactions. -- this is just not true.
% The objective of my work is to predict PNAG binding sites at the surface of PgaB using unrestrained molecular dynamics simulations in the presence of GlcNac. This methodology was used in earlier inositol studies to predict inositol binding sites on amyloid peptides and aggregates.
% \textbf{The directionality of the polymer binding.  This is currently not experimentally determined.} 


\section{Material and Methods}
% Modelling details
PgaB with a loop spanning residues 613 to 619 (numbered according to the chimeric PDB structure; check with Dustin again) in the N-terminal domain modelled into the truncated crystal structure, was used in our simulations. The initial crystal structure has Ni(II) bound at its enzymatic active site. Ni(II) is tetrahedrally coordinated with surrounding residues.  The acetate ion bound near the active site, an artifact of crystallization, was removed in the simulation. Crystal waters were removed from the initial PDB structure. Histidine protonation states were assigned based on predicted pKa values using the web software PROPKA (\url{http://propka.ki.ku.dk}), and histidine hydrogen-bonding geometries in the initial crystal structure.

Protein and ions were modelled using the AMBER99 force field.\cite{Cornell:1995td} Parameters for Ni(II) was approximated using the parameters for the magnesium ion (Mg$^{2+}$). After assigning protonation states of histidines, the net charge of the protein was -10e. The final simulation system comprised of 11 sodium (Na$^{+}$) counterions, and either 45 molecules of free $\beta$-N-acetyl-glucosamine (GlcNac) or glucosamine, at a concentration of 100 mM (Figure~\ref{fig:nag}). 19533 and 19991 water molecules were present in the GlcNac and glucosamine simulations, respectively. The initial volume of the simulation box is 713.6 nm$^{3}$.  To mimic experimental conditions, 100 mM of salt was added to the aqueous solution in glucosamine simulation systems.

% Note: To generate this structure from Glycam builder, choose the beta-pyranose ring and then acetyl-glucosamine
GlcNac molecules were generated using the web-based Glycam Biomolecule Builder (\url{http://glycam.ccrc.uga.edu/ccrc/biombuilder/biomb_index.jsp}). The GLYCAM06 force field for carbohydrates\cite{Kirschner:2008ii} was used to model GlcNac and glucosamine. A PDB of glucosamine-$\textrm{NH}_{3}^{+}$ was obtained from the ZINC database.\cite{Irwin:2005kx} Energy minimization was performed using the software GAUSSIAN-09.\cite{g09} The minimized glucosamine structure was consistent with the GLYCAM force field, and new RESP-derived partial atomic charges were computed for glucosamine (a net charge of -1e) by fitting to a single HF/6-31G* molecular electrostatic potential (MEP) with a restraint weight of 0.01. MEPs were computed using the CHELPG methodology\cite{Breneman:1990ue} with the R.E.D. III software package.\cite{Dupradeau:2010bb} The partial charges were assigned so that the HCNH3 group summed to a net charge of +1.164e, and the rest of the molecule summed to a net charge of -0.164e. Aliphatic hydrogen atoms were fitted with a zero partial charge in order to be compatible with GLYCAM06.
% How to cite the web builder - Carbohydrate Builder Woods Group. (2005-XXXX) GLYCAM Web. Complex Carbohydrate Research Center, University of Georgia, Athens, GA. (http://www.glycam.com) XXXX = current year

% Note put the partial charges used for glucosamine in an appendix. Ask Chris Ing to add to this methods description.  

The TIP3P water model was used to represent the solvent. Version 4.5.5 of the GROMACS software package\cite{Pronk:2013ef,Hess:2008p5353} was used to perform unrestrained all-atom MD simulations with the stochastic dynamics algorithm using an integration timestep of 2 femtoseconds.

% Better organize below to make clear that I was using different integrators for equilibration and production dynamics.
% I still need to remind myself what rcoulomb and rlist corresponds to physically in the MD algorithm
Electrostatic interactions were calculated using Particle Mesh Ewald (PME) summation with a grid size of 0.12 nm and a Coulombic real-space cutoff of 1.1 nm. The Lennard-Jone potential was computed up to 1.2 nm using the GROMACS twin-range cutoff function with a short-range cut-off of 1.1 nm. Covalent bonds involving hydrogens were constrained using the LINCS algorithm. The simulation system was first subjected to energy minimization followed by a 1 ns equilibration in the NVT ensemble using Berendsen temperature coupling at 300 K with a coupling constant of 2.0.

A second equilibration was performed for 1 ns in the NpT ensemble with isotropic pressure coupling. Temperature and pressure for equilibration were controlled at 300 K and at 1 atm, respectively, using Berendsen thermostat and pressure coupling schemes. Production simulations were performed using the stochastic dynamics (sd) integrator and the Parrinello-Rahman barostat for pressure coupling.
% Look at /mnt/scratch_mp2/pomes/ligrace1/pgab/protein_sugar/params for fill in the blank parameters here

13 independent MD simulations were performed for each of GlcNac and glucosamine, respectively. In total, XXX $\mu$s of sampling was performed.

\subsection*{Analysis Protocol}
To compute spatial binding probability densities of GlcNac and glucosamine, frames from our simulations were first fitted via RMSD alignment of the protein backbone atoms to a crystal structure that was energy minimized and equilibrated. The density map corresponds to the fractional atomic occupancy of GlcNac or glucosamine molecules binned using a grid with a resolution of 1 \angstrom.  The density maps involving carbohydrates were computed using  $\sim$ 900,000 time frames. The Visual Molecular Dynamics (VMD) software package\cite{Humphrey:1996to} was used to calculate and graphically render the densities depicted in Figures~\ref{fig:pgab_density}, \ref{fig:pgab_binding_sites}, and XXX.
% By examining the structure - does the sugars arrange into predicted binding site (s)?

\section{Results and Discussion}

\subsection{Protein Dynamics}
PgaB did not show any gross rearrangements or large perturbations during the simulations.
RMSD of the entire protein
RMSD of just the C-terminal domain
Fluctuation of the loop region (Need a control simulation of just PgaB without any sugars).

\subsection{GlcNac and glucosamine binding}
GlcNac molecules preferentially bind to residues in the N-terminal domain (Figure~\ref{fig:pgab_density}B and D). The spatial binding probability density of bound GlcNac, depicted in Figure~\ref{fig:pgab_density} suggests that its binding sites are localized to three main sites on PgaB (Figure~\ref{fig:pgab_binding_sites}).  Notably, the binding densities for GlcNac (A, B in Figure~\ref{fig:pgab_binding_sites}A,B} are predominantly located in proximity to the active site, consistent with the function of PgaB as a PNAG de-acetylase. In particular, the binding densities located in the crevice found beneath the loop spanning residue numbers 309 to 314 may be involved in substrate binding. The role of this region of the protein has not yet been determined experimentally.

% in the inter-domain groove found directly beneath a loop connecting the two domains (Figure~\ref{fig:pgab_density}), and on the surface directly below the active site, which contains surface exposed hydrophobic residues.

% Moreover, our simulations have identified binding sites of GlcNac that were not predicted from solely examining the static crystal structure. 

% When comparing the densities of PNAG and glucosamine (depicted in Figures \ref{fig:pgab_density} and XXX, respectively) it can be seen that whereas PNAG predominantly binds the N-terminal domain, but glucosamine instead bind to the C-terminal domain, predominantly in the groove capped by the loop spanning residues XXX to YYY.

\textbf{The spatial distribution of bound glucosamine, as depicted in Figure~\ref{fig:pnag_nag_overlapped_zoomedout}, indicates that although glucosamine molecules are predominantly bound to the C-terminal domain. Specifically, clusters of glucosamine molecules are bound in the groove lined with acidic (Asp and Glu), and aromatic residues Trp613, Trp552 (Figure~\ref{fig:nag_cterminal_zoomedin})}.  The C-terminal domain of PgaB is found to be homologous with glucosidase hydrolases, where their enzymatic active sites are located in grooves with a similar geometry. Taken together,  our results suggest that this C-terminal groove of PgaB is a binding site for the de-acetylated PNAG (dPNAG).   Notably, the results from simulation indicates that both GlcNAc and glucosamine possess  binding sites at Trp613, and Trp552, consistent with the crystal structure of PgaB-CT with bound glucosamine (Figure~\ref{fig:nag_cterminal_zoomedin}).

Furthermore, as depicted by the spatial distribution of bound glucosamine in Figure~\ref{fig:},  the entire length of the groove can bind glucosamine molecules, suggesting that this groove may accommodate a linear polymer of glucosamine.  In support of this result, glucosamine molecules were observed to bind in the putative binding groove of PgaB-CT by forming linear hydrogen-bonded chains (Figure~\ref{fig:directionality}).  Taken together, the morphology of the overall binding density suggests that PNAG may bind to PgaB by wrapping around the protein (Figure~\ref{fig:pgab_density}A and B).  
% Based on our results, GlcNac may be more likely than glucosamine to bind at this loop binding site because the the binding densities for GlcNAc, but not for glucosamine persist at higher isovalues (0.2). -- Needs to be verified.
In contrast to the spatial distributions of glucosamine and GlcNac, salt ions do not significant bind the protein, and are uniformly distributed over the surface of the protein at low occupancies (isovalue of approximately XXXX) (Figure~\ref{fig:salt_density_distribution}).   
% This result further supports the ability of MD simulations and our methodology to identify carbohydrate-protein binding sites. 


% with the fact that these residues are conserved across PgaB  homologues. \textbf{See Dustin's first pgab paper} REF
% Notably, in our sim- ulations, glucosamine was found to adopt a binding mode consistent with the crystal structure
% Save this for the discussion section? Moreover, our simulations identified carbohydrate binding to Trp613, a key conserved residue across PgaB homologues that is found in the C-terminal domain loop spanning residues 613 to 619.
% TODO: Make a figure from MD showing glucosamine binding to compare to X-ray.
% TODO: Note that I should redo these plots with the correct density maps where I’ve correctly trjcatted the PgaB trajectories
% TODO: Note that here Kd calculations might help with hypothesizing which glcnac or glucosamine that binds there with high Kds.
% Because of the mobility of this loop region, we hypothesize that this loop may be involved in the export process whereby the opening and closing of this loop over the binding groove facilitate the export and stabilize the polymer for binding at the enzymatic active site.
% Loop clamping. Perhaps the loop may periodically open and clamp down (ie. bind particular sugars with higher affinity) on particular forms of the sugar on the PNAG.  It may clamp down for glucosamine or GlcNac. 

 


When a vector is drawn from the CX carbon to the nitrogen atom on the $NH_{3}^{+}$ group of glucosamine, we observe that these vectors point towards the N-terminal domain. Based on this result, we speculate that the (reducing or non-reducing end of the polymer?) to (reducing/non-reducing) end of the polymer extends from the N-terminal towards the C-terminal of PgaB.  This hypothesized polymer directionality is consistent with our proposed mechanism of synthesis and export whereby de-acetylation in PgaB-NT occurs prior to the binding of de-acetylated-PNAG by PgaB-CT.  Note that the hydrogen-bonded chain spanning the first three monomers in the groove shown in Figure~\ref{fig:} mimics the linkage pattern of PNAG, except you would have to rotate the schematic of PNAG by 180 degrees. See PNAG clipped from Little 2012, Figure S7.

% I don't fully understand the directionality business.  Here is a hack job at summarizing what it is from readings.
% From wikipedia -- An oligosaccharide has both a reducing and a non-reducing end. The reducing end of an oligosaccharide is the monosaccharide residue with hemiacetal functionality, thereby capable of reducing the Tollens’ reagent, while the non-reducing end is the monosaccharide residue in acetal form, thus incapable of reducing the Tollens’ reagent.[2] The reducing and non-reducing ends of an oligosaccharide are conventionally drawn with the reducing-end monosaccharide residue furthest to the right and the non-reducing (or terminal) end furthest to the left.[2]

% directionality paper
% http://www.jbc.org/content/274/37/26557.full

This study demonstrates the utility of large-scale MD simulations in predicting carbohydrate-protein binding modes, when combined with a force field that can with relative accuracy reproduce subtleties in sugar-protein interactions (for example, GLYCAM). Molecular simulations of carbohydrates and the accuracy of various sugar force fields has been recently reviewed here.\cite{Fadda:2010p5889}  Moreover, we demonstrate the utility of computing binding densities to map out sugar binding sites. A recent study suggests that by building 3D density binding maps and combining this data with additional algorithms can increase the accuracy of identifying putative carbohydrate binding sites.\cite{Tsai:2012bj} As part of future work, it may be interesting to employ a similar approach to investigate the binding of polysaccharides to PgaB.  

%%%%%%%%%%%%%%%%%%%%%% Future work %%%%%%%%%%%%%%%%%%%%%%
% What is the name for positively charged glucosamine molecules? Is there a 3-letter abbrev. for it?
% This paragraph can be part of the discussion.

% Various analyses (eg. time evolution of RMSD, RMSF of the protein, principle component analysis, etc.) can be done to quantify protein dynamics, especially the mobility of the two domains. Furthermore, it will be interesting to investigate how dynamics may be correlated to the function of PgaB.  Finally, it will be interesting to predict GlcNac and GlcNH$_3^+$ binding modes and binding constants from MD simulations so that they can be compared with corresponding future experimental results.

% \subsection{New written up results to incorporate}
% These are results -- writing them here first 
% GlcNac binding. Moreover, GlcNac interacts with conserved hydrophobic residues X, Y, Z in CE4 (Figure S1 in Little, JBC, 2012). Figure caption (overlapped zoomed out densities): The magenta colored densities over PgaB-NT cover conserved residues in CE4.  These residues are highlighted in magenta in Little et. al. in Figure S1. 

% \subsection{From dustin's committee report (Jan 2013)}
% Significant efforts to obtain a co-crystal structure of PgaB (or PgaB-CT) in complex with a PNAG oligomer have failed.
% Description of glucosamine binding site on CT domain: This hypothesis (CT is a CBM) is further supported by a co-crystal structure of PgaB-CT in complex with glucosamine (Figure 4). Attempts to obtain PgaB-CT in complex with N-acetylglucosamine (GlcNAc) under equivalent crystallization conditions have been unsuccessful. PgaB-CT crystallized in the presence of 200 mM glucosamine yielded two large peaks (3?) in the unbiased |Fo – Fc| difference electron density map. The two density peaks could not be accounted for by side chain or buffer molecules, and were large enough to accommodate a glucosamine molecule. The first molecule of glucosamine was refined unambiguously and is coordinated by Asp472 and Trp552 (Figure 4). The second molecule was refined ambiguously keeping stereochemistry in the most preferred orientation, and is coordinated by Tyr432, Asp466, and Trp613 (Figure 4)

\section{Acknowledgements}
This work was done in collaboration with Dustin Little, Dr. Lynne Howell and Dr. Mark Nitz.

\section*{Figures}

\begin{figure}[htbp]
\centering
\includegraphics[height=1.5in, width=3in]{figures/results4/figure_pgab_sugars.png}
\caption[NAG]{$\beta$-N-acetyl-glucosamine (GlcNac) and glucosamine (GlcNH$_2$) monosaccharides.}
\label{fig:nag}
\end{figure}

\begin{figure}[htbp]
\centering
\includegraphics[height=4.25in, width=6in]{figures/results4/figure_pgab_density.png}
\caption[NAG binding density]{Spatial binding probability density map of bound GlcNac around PgaB.  (A) An example snapshot of PgaB (grey) shown using a surface representation with bound GlcNac binding density depicted in purple at 10\% occupancy (iso-contour of 0.1). Binding densities of GlcNac overlapped with a cartoon representation of PgaB at iso-contour levels of (B) 0.1 (C) 0.2 (D) 0.25. In our coloring scheme, residue numbers 43 to 310 and numbers 311 to 667 represent N- (green) and C-terminal (cyan) domains, respectively.}
\label{fig:pgab_density}
\end{figure}

\begin{figure}[htbp]
\centering
\includegraphics[height=6.29in, width=6.12in]{figures/results4/figure_pgab_binding_sites.png}
\caption[GlcNac binding sites]{High probability GlcNac binding sites at an iso-contour value of 0.25 (D). Insets (A to C) show detailed views of the binding sites and the residues involved in binding. \textbf{This figure probably isn't that useful}}
\label{fig:pgab_binding_sites}
\end{figure}

\begin{figure}[htbp]
\centering
\includegraphics[width=7in]{figures/results4/pnag_nag_sdf_zoomedout.png}
\caption{Comparisons of the distributions of bound PNAG and NAG. Note that the magenta colored densities over PgaB-NT cover conserved residues in CE4.  (These residues are highlighted in magenta in Little et. al. in Figure S1. Id the residues and list them in the text and in the caption.)}
\label{fig:pnag_nag_overlapped_zoomedout}
\end{figure}

\begin{figure}[htbp]
\centering
\includegraphics[width=6.23in]{figures/results4/nag_cterminal_zoomedin.png}
\caption{Distribution of NAG in the binding groove in the C-terminal domain.}
\label{fig:nag_cterminal_zoomedin}
\end{figure}

\begin{figure}[htbp]
\centering
\includegraphics[width=7in]{figures/results4/salt_distribution_kitten.png}
\caption[Ionic distribution]{\textbf{Not yet rendered}. Spatial probability distribution of salt ions around \pgab}
\label{fig:salt_density_distribution}
\end{figure}

\begin{figure}[htbp]
\centering
\includegraphics[width=6in]{figures/results4/glucosamine_binding_direction_suggestive.png}
\caption[Polymer directionality]{A bound linear hydrogen-bonded chain of glucosamine in the putative carbohydrate-binding groove of the C-terminal domain.  The yellow arrows are drawn as a guide for the directionality of the chain. The bottom panel shows a pentamer PNAG.}
\label{fig:directionality}
\end{figure}

\begin{singlespace}
\addcontentsline{toc}{section}{Bibliography}
\bibliographystyle{elsart-num}
\bibliography{/Users/grace/github/thesis/document/results4/results4}
\end{singlespace}

% Objective of this work: 
% Binding surface - Map out a binding surface for both domains - We don't know what the C-Terminal domain is responsible for.  
% Compare and predict binding affinities.  Dustin is trying to get experimental data and co-crystal structures.

% Dustin -- But to get back to your main idea - i'm blabbering on here, other then the density map and binding constants. I guess it would be nice to see where the most flexible parts of the protein are - and what this might correlate too. What other quantitative values can you obtain?

% protein motion -- dynamics of the protein - can it be related to how it might function?
% \begin{itemize}
% 	\item Rmsd vs. time for the entire protein
% 	\item Rmsd vs. time for each of the individual domains
% 	\item Rmsf of the protein
% \end{itemize}

% % FIGURE
% \begin{figure}[pgab_rmsd]
% \centering
% \includegraphics[height=4.1in, width=6.23in]{figures/pgab_rmsd.jpg}
% \caption[PgaB protein dynamics]{Time evolution of RMSD of PgaB from its crystal state}
% \label{fig:pgab_rmsd}
% \end{figure}
% 
% % FIGURE
% \begin{figure}[pgab_rmsf]
% \centering
% \includegraphics[height=4.1in, width=6.23in]{figures/pgab_rmsf.jpg}
% \caption[PgaB protein dynamics]{Time evolution of RMSD of PgaB from its crystal state}
% \label{fig:pgab_rmsf}
% \end{figure}

% FIGURE
% \begin{figure}[pgab_domains]
% \centering
% \includegraphics[height=4.1in, width=6.23in]{figures/pgab_domains.jpg}
% \caption[PgaB protein dynamics]{Time evolution of RMSD of PgaB from its crystal state}
% \label{fig:pgab_domains}
% \end{figure}
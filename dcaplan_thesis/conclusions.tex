\chapter{Conclusions and future directions}

The new unconstrained simulation model of C\emph{c}O embedded in a solvated lipid bilayer constructed in this thesis has already proven to be valuable. It was used to show, for the first time, that the K-channel can become fully hydrated through simulation. Equilibrium hydration was studied in both proton uptake channels and the results were compared to earlier work by Henry et al., where a simpler, finite-size model of C\emph{c}O was used \cite{Henry:2011p10221}. In addition, the favourable orientations of the putative gating residue N139 were determined and compared to previous work. Finally, the D-channel's affinity for \ce{K^+} cations was studied after the cation bound spontaneously to the vestibule region of the channel. The results predict a significant loss of proton selectivity in the D-channel of the N139D single mutant (and possibly in the double mutant). This finding suggests that experimentally-observed decoupling and/or inhibition of certain mutants may be an artifact of the experimental protocols used in such studies, which is a testable hypothesis to be investigated in the lab of our collaborator Mårten Wikström (University of Helsinki, Finland).

Elucidating the interplay between conformational fluctuations, hydration, proton relay, and now cation inhibition is required to understand molecular mechanisms of proton movement in hydrated channels and cavities, especially in a pump which requires kinetic as well as thermodynamic control of proton movement. While important mechanistic questions remain, this work provides a new simulation model which can be used to study the molecular mechanism of proton pumping in C\emph{c}O via the systematic examination of proton pathways throughout the enzyme interior, including the binuclear centre as well as the putative proton loading site and proton exit pathways. In the near future, the model may be used to study equilibrium hydration in other redox states of the enzyme, as well as the mechanism of decoupling by other D-channel mutants, paving the way to detailed mechanistic studies of the energetics and kinetics of proton binding and translocation throughout the protein's proton pathways.


Electron and proton transfer are coupled as electron transfer to the active site is always accompanied by net uptake of one proton for charge compensation \cite{Gennis:2004p10239}. Because of this, experimentalists can use techniques such as photoreduction to study specific proton transfer reactions within the enzyme. Controlled electron transfer steps can be monitored by UV-vis spectroscopy and then correlated with proton transfer reactions.



% experimental clues
%  the E286Q mutant does not proceed beyond Pr when when O2 reacts with the fully reduced enzyme.
%  the Pr→F step is associated with a net proton uptake in the wild type oxidase, no proton uptake is associated with this step with the D132N mutant (61)
%  The sequential Pr→F and F→O transitions both require protons delivered via E286, with the requirement that E286 be reprotonated from D132 in between these steps. The D-channel has an internal proton capacity (at least the1 proton on E286) allowing the Pr→F transition to proceed even when the channel entrance is blocked.


%Blocking the entrance of the K-channel (E101II mutants), in contrast, dramatically inhibits electron transfer from heme a to the oxidized heme a3 /CuB center (16)

%
% EXPERIMENTAL TRICKS
%

% 1. pH-dependence of electron transfer reactions.
% 2. Solvent kinetic isotope effects on individual electron transfer reactions. The kH/kD ratio is determined by running the reaction in H2O or D2O. This can be used to identify electron transfer reactions that are rate-limited by coupled proton transfers (46, 47).
% 3. Proton release from oxidase reconstituted in phospholipid vesicles (48).
% 4. Net proton uptake/release from oxidase in solution (detergent-solubilized) (49).
% 5. Electrometric measurement of charge separation across the membrane by oxidase reconstituted in vesicles (24, 29, 34, 50, 51). This time-resolved voltage measurement monitors electrical work done by moving charges across the membrane.
% 6. FTIR difference spectroscopy. This method can identify individual residues, particularly glutamates and aspartates, that undergo changes in hydrogen bonding or protonation (52-57).



% Cyanide, sulfide, azide, and carbon monoxide all bind to cytochrome c oxidase, thus competitively inhibiting the protein from functioning.
% These states, including the nature of the intermediate compounds, have been defined experimentally using a series of time-resolved spectroscopic methods.
% the enzyme can be fully oxidized by ...
% the enzyme can be fully reduced by the addition of ascorbate. Ascorbate is a mild reducing agent and acts as an antioxidant.

%% Reaction of O2 with the Fully Reduced Enzyme (R4)- the Flow-Flash Method
%% The protein is prepared in the fully reduced R4 state which is bound to CO. After rapid mixing of O2, the reaction is initiated by the photolysis of CO. Optical monitoring of the hemes reveals steps: R4→A→Pr→F→O with approximate time constants of 8 μsec, 30 μsec, 120 μsec and 1 msec (R. sphaeroides enzyme)\cite{Gennis:2004p10239}

%% Photoreduction of O, E, Pm and F
%%%%%% Individual steps can be examined in isolation by using a light-generated reductant, usually a derivative of Ruthenium (e.g., Ruthenium bis bipyridine, RuBpy), which is non-covalently bound to the enzyme (24, 33, 34). Methods have been devised to trap most of the enzyme in states E, Pm or F. The isolated one-electron steps correspond to those shown in Figure 4A: E→R2, Pm→F and F→O.

% Generation of Pm
%% The fully oxidized enzyme is reduced by CO, a 2- electron reductant, in the presence of O2. After the reaction of CO, two electrons are present in the enzyme (see Section 2.4.4), so the subsequent reaction with O2 forms the product Pm.(see Figure 4A). Photoreduction results in the Pm→F transition.

% Generation of F
%% Exposure of the oxidized enzyme to H2O2 at alkaline pH results in the generation of state F. This allows one to examine the F→O transition by photoreduction. The peroxide reaction is actually a steady state cycle (35, 36).

% Generation of E
%% Once state F is obtained by reacting with peroxide, the peroxide can be eliminated using catalase and the enzyme reduced by CO (29, 37). This 2-electron reduction brings the enzyme through O and to E directly. Photoreduction allows one to examine the E→R2 transition (29). The CO that is present binds to and traps the product in which both electrons are at the heme a3 /CuB center, so the actual transition studied is E→R2(CO).

% O→R4, reduction of the fully oxidized enzyme
% The anaerobic addition of a chemical reductant such as Ruthenium hexammine, dithionite or ascorbate/TMPD to the fully oxidized enzyme using a stopped flow rapid mixer allows one to time-resolve the rates of reduction of heme a and heme a3 (38). Proton uptake (about 2.4 protons, depending on the pH) accompanies full reduction of the oxidase (40).

% END OF EXPERIMENTAL TRICKS

% During turnover, four electrons are utilized to reduce O2 to H2O followed by the sequential transfer from reduced cytochrome c to Cu$_{A}$, heme a, and finally to catalytic site (heme a-Cu$_{B}$). Each of the transitions between the intermediates accompanies proton and electron transfer to catalytic site and proton pumping, typically in the microsecond time scale. Characterizing the details of each intermediate is important to understand the proton pumping mechanism. The catalytic cycle and the nature of intermediates have been defined by a series of time-resolved spectroscopic methods (32-38). The flow-flash technique has been frequently used. In the flow-flash method, the oxidase is first fully reduced and a CO ligand is bound to ferrous heme a$_{3}$, at the site where O2 binds at the catalytic site. Then, the anaerobic-CO solution
% 
% is mixed with an O2-containing solution. Upon a few nanosecond laser flash the CO ligand is removed, producing O2 bound reduced enzyme, which is linked to oxygen reduction.
% The catalytic cycle (Figure 1. 5) initiates with fully oxidized enzyme, O state, where \ce{Cu_B}in the catalytic site has high electron affinity and produces the one-electron reduced enzyme by forming the E state. During this step, one electron is taken up along with 2 protons from the negative side (N-side). Next, the E state to R state transition translocates a second electron to the binuclear center and two protons are taken up as well.
% The reduced catalytic center now binds an O2 molecule in the ferrous form of heme a$_{3}$ forming the A state in a time scale of 10 μs. After the A state either PM or PR states can be formed based on the availability of a third-electon. PM state is formed (300 μs) when there is no electron available. The third electron is transferred from an external electron donor to the catalytic site. The electron is coupled to proton transfer. The O-O bond is broken in the PM state and the active tyrosine is thought to form a neutral radical (41). This step is not coupled to proton pumping. The PR state is formed by rapidly accepting the third electron from heme a with the time constant of 30-50 μs, which is faster than that of the proton transfer to the catalytic site (100 μs) and forms F state. The final transition, F state to fully oxidized O state takes a proton from bulk solution and pumps one proton in time scale of 1.2 ms (Figure 1.7) (39, 40). -->
% Depending on the step in the mechanism, a proton may be taken up \emph{via} either one of the two known proton channels, and may end up either in the BNC or the proton-loading site (PLS).
